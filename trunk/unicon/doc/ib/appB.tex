\chapter{Virtual Machine Instructions}

This appendix lists all the Icon virtual machine instructions. For
instructions that correspond to source-language operators, only the
corresponding operations are shown. Unless otherwise specified,
references to the stack mean the interpreter stack.


arg n\ \ Push a variable descriptor pointing to argument n.


asgn\textit{\ \ expr1 }:= \textit{exp12}


bang \textit{lexpr}


bscan\ \ Push the current values of \&subject and \&pos. Convert the
descriptor prior to these two descriptors into a string. If the
conversion cannot be performed, terminate execution with an error
message. Otherwise, assign it to \&subject and assign 1 to \&pos. Then
suspend. If resumed, restore the former values of \&subject and \&pos
and fail.


cat\textit{\ \ expr1 {\textbar}{\textbar}} \textit{exp12}


ccase\ \ Push a copy of the descriptor just below the current expression frame.


chfail n Change the failure ipc in the current expression frame marker to n.


coact\ \ Save the current state information in the current
co-expression block, restore state information from the co-expression
block being activated, perform a context switch, and continue
execution.


cofail\ \ Save the current state information in the current
co-expression block, restore state information from the co-expression
block being activated, perform a context switch, and continue
execution with co-expression failure signal set.


compl \~{}\textit{expr}


caret\ \ Save the current state information in the current
co-expression block, restore state information from the co-expression
block being activated, perform a context switch, and continue
execution with co-expression return signal set.


create\ \ Allocate a co-expression block and a refresh block. Copy the
current procedure frame marker, argument values, and local identifier
values into the refresh block. Place a procedure frame for the current
procedure on the stack of the new co-expression block.


cset a\ \ Push a descriptor for the cset block at address a onto the stack.


diff\textit{\ \ expr1 -{}-expr2}


div\textit{\ \ expr1 }/ \textit{expr2}


dup\ \ Push a null descriptor onto the stack and then push a copy of
the descriptor that was previously on top of the stack.


efail\ \ If there is a generator frame in the current expression
frame, resume its generator. Otherwise remove the current expression
frame. If the ipc in its marker is nonzero, set ipc to it. If the
failure ipc is zero, repeat efail.


eqv\textit{\ \ expr1 }=== \textit{expr2}


eret\ \ Save the descriptor on the top of the stack. Unwind the C
stack. Remove the current expression frame from the stack and push the
saved descriptor.


escan\ \ Dereference the top descriptor on the stack if
necessary. Copy it to the place on the stack prior to the saved values
of \&subject and \&pos (see bscan). Exchange the current values of
\&subject and \&pos with the saved values on the stack. Then
suspend. If resumed, restore the values of \&subject and \&pos from
the stack and fail.


esusp\ \ Create a generator frame containing a copy of the portion of
the stack that is needed if the generator is resumed.


field n\ \ Replace the record descriptor on the top of the stack by a
descriptor for field n of that record.


global n Push a variable descriptor pointing to global identifier n.


goto n\ \ Set ipc to n.


init n\ \ Change init instruction to goto.


int n\ \ Push a descriptor for the integer n.


inter\textit{\ \ expr1 }** \textit{exp'2}


invoke n \textit{exp1lJ(expr1, exp'2, }..., \textit{exprn)}


keywd n Push a descriptor for keyword n.


lconcat\textit{\ \ expr1 {\textbar}{\textbar}{\textbar}} \textit{expr2}


lexeq\textit{\ \ expr1 }== \textit{expr2}


lexge\textit{\ \ expr1 }{\guillemotright}= \textit{expr2}


lexgt\textit{\ \ expr1 }{\guillemotright} \textit{expr2}


lexle\textit{\ \ expr1 }{\guillemotleft}= \textit{expr2}


lexlt\textit{\ \ expr1 }{\guillemotleft} \textit{expr2}


lexne\textit{\ \ expr1 }{}-== \textit{expr2}


limit\ \ Convert the descriptor on the top of the stack to an
integer. If the conversion cannot be performed or if the result is
negative, terminate execution with an error message. If it is zero,
fail. Otherwise, create an expression frame with a zero failure ipc.


line n\ \ Set the current line number to n.


llist n\textit{\ \ [expr1,expr2,...,exprn]}


local n\ \ Push a variable descriptor pointing to local identifier n.


lsusp\ \ Decrement the current limitation counter, which is
immediately prior to the current expression frame on the stack. If the
limit counter is nonzero, create a generator frame containing a copy
of the portion of the interpreter stack that is needed if the
generator is resumed. If the limitation counter is zero, unwind the C
stack and remove the current expression frame from the stack.


mark\ \ Create an expression frame whose marker contains the failure
ipc corresponding to the label n, the current efp, gfp, and ilevel.


mark0\ \ Create an expression frame with a zero failure ipc.


minus\textit{\ \ expr1 -expr2}


mod\textit{\ \ expr1 }\% \textit{expr2}


mult\textit{\ \ expr1 }* \textit{expr2}


neg \textit{{}-expr}


neqv\textit{\ \ exprl }{}-=== \textit{expr2}


nonnull \textit{{\textbackslash}expr}


null \textit{/expr}


number\textit{\ \ +expr}\ \ .


numeq\textit{\ \ expr1 }= \textit{expr2}


numge\textit{\ \ exprl }{\textgreater}= \textit{expr2}


numgt\textit{\ \ expr1 }{\textgreater} \textit{expr2}


numle\textit{\ \ exprl }{\textless}= \textit{expr2}


numlt\textit{\ \ exprl }{\textless} \textit{expr2}


numne\textit{\ \ exprl }{}-= \textit{expr2}


pfail\ \ If \&trace is nonzero, decrement it and produce a trace
message. Unwind the C stack and remove the current procedure frame
from the stack. Then fail.


plus\textit{\ \ exprl }+ \textit{expr2}


pnull\ \ Push a null descriptor.


pop\ \ Pop the top descriptor.


power\textit{\ \ exprl }\~{} \textit{expr2}


pret\ \ Dereference the descriptor on the top of the stack, if
necessary, and copy it to the place where the descriptor for the
procedure is. If \&trace is nonzero; decrement it and produce a trace
message. Unwind the C stack and remove the current procedure frame
from the stack.


psusp\ \ Copy the descriptor on the top of the stack to the place
where the descriptor for the procedure is, dereferencing it if
necessary. Produce a trace message and decrement \&trace if it is
nonzero. Create a generator frame containing a copy of the portion of
the stack that is needed if the procedure call is resumed.


push1\ \ Push a descriptor for the integer 1.


pushn1\ \ Push a descriptor for the integer -1.


quit\ \ Exit from the interpreter.


random \textit{?expr}


rasgn\textit{\ \ expr1 }{\textless}- \textit{expr2}


real a\ \ Push a descriptor for the real number block at address a onto the stack.


refresh \textit{\~{}expr}


rswap\textit{\ \ expr1 }{\textless}-{\textgreater} \textit{expr2}


sdup\ \ Push a copy of the descriptor on the top of the stackr


sect\textit{\ \ expr1 [} \textit{expr2: expr3}\ \ ]


size\ \ *expr


static n Push a variable descriptor pointing to static identifier n.


str n, a Push a descriptor for the string of length n at address a.


subsc \textit{expr1[expr2]}


swap\textit{\ \ expr1 }:=: \textit{exp12}


tabmat \textit{=exp'}


toby\textit{\ \ expr1 }to \textit{expr2 }by \textit{expr3}


unions\textit{\ \ expr1 }++ \textit{expr2}


unmark\ \ Remove the current expression frame from the stack and
unwind the C stack.


value\ \ \ \ \textit{.expr}
