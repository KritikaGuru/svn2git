\chapter{Networking, Messaging and the POSIX Interface}

Unicon's system interface is greatly enriched compared with Icon,
primarily in that it treats Internet connections and Internet-based
applications as ubiquitous, and extends the file type with
appropriately high-level capabilities.  Fundamental TCP and UDP
connections are a breeze using the networking facilities, and common
application-level protocols are supported via the messaging facilities
(see also the X11 graphics facilities and the SQL/ODBC database
facilities for examples where application-level networking is provided
in Unicon). Portions of this chapter related to the messaging
facilities were contributed by their author, Steve Lumos.

\section[27.1 Networking Facilities]{27.1 Networking Facilities}

\subsection[27.1.1 Sockets in the Unicon File Functions]{27.1.1 Sockets in the Unicon File Functions}

UNIX file abstractions allow certain low-level file I/O
operations to be performed across either files or network sockets.
Unicon's treatment of network connections as a special type of file
uses the same abstraction, but Icon's existing file operations are
written in terms of the higher-level buffered file I/O in ANSI C.

In general, the file operations have to call different underlying C calls 
depending on what kind of file is being used.  File values in Unicon
have a status word, and that word has various bits to indicate whether
the file is really a network connection, graphics window, database, etc.

\subsection[27.1.2 Simultaneous Input from Different File Types]{27.1.2 Simultaneous Input from Different File Types}

Modern applications may need to allow asynchronous input from all
types of files. Unicon's select() originally used UNIX C's select()
function as-is to handle input from various network connections,
including X Windows GUI input.  MS Windows required an elaborate
alternative to be developed, since window input events do not
arrive on file descriptors.  Eventually, the UNIX implementation
was also migrated off the straight C select(), for greater flexibility
in handling other file types.  The general approach is to break the
input list of input file handles out into sublists of each type, and
handle them with separate calls, which must be repeated until some
input is found.  This polling approach can waste CPU. A small delay
is included in the loop to prevent the CPU burning.


\section[27.2 Messaging Facilities]{27.2 Messaging Facilities}
\subsection[27.2.1 The Transfer Protocol Library]{27.2.1 The Transfer
Protocol Library}

All of the message facilities are handled by the transfer protocol
library (libtp). This library provides an abstraction of the many
different protocols (HTTP, SMTP, etc) into a clear and consistent
API. Ease of adding support for new protocols and porting the entire
library to new operating system interfaces were primary design
goals. These goals are both accomplished by using the AT\&T Labs
discipline and method (DM) architecture described below.

\subsection[27.2.2 Libtp Architecture]{27.2.2 Libtp Architecture}

The key feature of the DM architecture is that it makes explicit two
interfaces in the library: \textit{disciplines} which hold system
resources and define routines to acquire and manipulate them, and
\textit{methods} which define the higher-level algorithms used to
access these resources. This model fits the problem of Internet
transfer protocols nicely; the discipline abstracts the operating
system interface to the network, and there is a method for each
protocol that defines communication with a server only in terms of the
discipline.

This architecture makes porting easy because you need only create a
discipline for the new system, which means writing 9 functions. The
only currently-existing discipline handling both the Berkeley Socket
and WINSOCK APIs is only 400 lines long. Once a discipline exists, the
new system immediately gains all of the supported protocols.

\subsection[27.2.3 The Discipline]{27.2.3 The Discipline}

The discipline is a C structure whose members are pointers to functions:

\bigskip

{\ttfamily\mdseries
typedef struct \_tp\_disc\_s\ \ Tpdisc\_t; \ \ /* discipline */}


\bigskip

{\ttfamily\mdseries
typedef int\ \ \ \ \ \ (*Tpconnect\_f)(char* host, u\_short port, Tpdisc\_t* disc);}

{\ttfamily\mdseries
typedef int\ \ \ \ \ \ (*Tpclose\_f)(Tpdisc\_t* disc);}

{\ttfamily\mdseries
typedef ssize\_t\ \ (*Tpread\_f)(void* buf, size\_t n, Tpdisc\_t* disc);}

{\ttfamily\mdseries
typedef ssize\_t\ \ (*Tpreadln\_f)(void* buf, size\_t n, Tpdisc\_t* disc);}

{\ttfamily\mdseries
typedef ssize\_t\ \ (*Tpwrite\_f)(void* buf, size\_t n, Tpdisc\_t* disc);}

{\ttfamily\mdseries
typedef void*\ \ \ \ (*Tpmem\_f)(size\_t n, Tpdisc\_t* disc);}

{\ttfamily\mdseries
typedef int\ \ \ \ \ \ (*Tpfree\_f)(void* obj, Tpdisc\_t* disc);}

{\ttfamily\mdseries
typedef int\ \ \ \ \ \ (*Tpexcept\_f)(int type, void* obj, Tpdisc\_t* disc);}

{\ttfamily\mdseries
typedef Tpdisc\_t* (*Tpnewdisc\_f)(Tpdisc\_t* disc);}


\bigskip

{\ttfamily\mdseries
struct \_tpdisc\_s}

{\ttfamily\mdseries
\{}

{\ttfamily\mdseries
\ \ Tpconnect\_f\ \ connectf;\ \ /* establish a connection */}

{\ttfamily\mdseries
\ \ Tpclose\_f\ \ \ \ closef;\ \ /* close the connection */}

{\ttfamily\mdseries
\ \ Tpread\_f\ \ \ \ readf;\ \ /* read from the connection */}

{\ttfamily\mdseries
\ \ Tpreadln\_f\ \ readlnf;\ \ /* read a line from the connection */}

{\ttfamily\mdseries
\ \ Tpwrite\_f\ \ \ \ writef;\ \ /* write to the connection */}

{\ttfamily\mdseries
\ \ Tpmem\_f\ \ \ \ \ \ memf;\ \ /* allocate some memory */}

{\ttfamily\mdseries
\ \ Tpfree\_f\ \ \ \ freef;\ \ /* free memory */}

{\ttfamily\mdseries
\ \ Tpexcept\_f\ \ exceptf;\ \ /* handle exception */}

{\ttfamily\mdseries
\ \ Tpnewdisc\_f\ \ newdiscf;\ \ /* deep copy a discipline */}

{\ttfamily\mdseries
\ \ int\ \ \ \ \ \ \ \ type;\ \ /* (not used currently) */}

{\ttfamily\mdseries
\};}


\bigskip


These functions define a complete API for acquiring and manipulating
all of the system resources needed by all of the methods and (it is
hoped) any conceivable method. By convention, every discipline
function takes a pointer to the current discipline as its last
argument. (Every method function takes a library handle which contains
a pointer to the current discipline, so the discipline functions are
always available when needed.) The Tpdisc\_t is an abstract
discipline. In practice, a new discipline will extend Tpdisc\_t by at
minimum adding some system dependent data such as a Unix file
descriptor or Windows SOCKET*. Here is the
{\textquotedbl}Unix{\textquotedbl} discipline (it would be better
called the socket discipline since it works for the Berkeley Socket
API and WINSOCK on multiple systems):

{\ttfamily\mdseries
struct \_tpunixdisc\_s}

{\ttfamily\mdseries
\{}

{\ttfamily\mdseries
\ \ \ Tpdisc\_t tpdisc;}

{\ttfamily\mdseries
\ \ \ int fd;}

{\ttfamily\mdseries
\}}

\subsection[27.2.4 Exception Handling]{27.2.4 Exception Handling}

The DM archtecture defines a very useful convention for exception
handling. Exceptions are passed as integers to the exceptf function
along with some exception-specific data. The function can do arbitrary
processesing and then return \{-1, 0, 1\}, which instructs the library
to retry the operation (1), return an error to the caller (-1), or
take some default action (0). Libtp uses constants TP\_TRYAGAIN,
TP\_RETURNERROR, and TP\_DEFAULT.

Although not as powerful as languages with true exceptions, the DM
exception handling definitely serves to make the code more
readable. In the Unix discipline, exceptf is used to aggregate all of
the many, sometimes transient errors that can occur in network
programming. For example, the Unix discipline's readf function is:

{\ttfamily\mdseries
ssize\_t unixread(void* buf, size\_t n, Tpdisc\_t* tpdisc)}

{\ttfamily\mdseries
\{}

{\ttfamily\mdseries
\ \ Tpunixdisc\_t* disc = (Tpunixdisc\_t*)tpdisc;}


\bigskip

{\ttfamily\mdseries
\ \ size\_t \ nleft;}

{\ttfamily\mdseries
\ \ ssize\_t nread;}

{\ttfamily\mdseries
\ \ char* \ \ ptr = buf;}


\bigskip

{\ttfamily\mdseries
\ \ nleft = n;}

{\ttfamily\mdseries
\ \ while (nleft {\textgreater} 0) \{}

{\ttfamily\mdseries
\ \ \ \ if ((nread = read(disc-{\textgreater}fd, ptr, nleft)) {\textless}= 0) \{}

{\ttfamily\mdseries
\ \ \ \ \ \ int action = tpdisc-{\textgreater}exceptf(TP\_EREAD, \&nread, tpdisc);}

{\ttfamily\mdseries
\ \ \ \ \ \ if (action {\textgreater} 0) \{}

{\ttfamily\mdseries
\ \ \ \ \ \ \ \ nread = 0;}

{\ttfamily\mdseries
\ \ \ \ \ \ \ \ continue;}

{\ttfamily\mdseries
\ \ \ \ \ \ \}}

{\ttfamily\mdseries
\ \ \ \ \ \ else if (action == 0) \{}

{\ttfamily\mdseries
\ \ \ \ \ \ \ \ break;}

{\ttfamily\mdseries
\ \ \ \ \ \ \}}

{\ttfamily\mdseries
\ \ \ \ \ \ else \{}

{\ttfamily\mdseries
\ \ \ \ \ \ \ \ return (-1);}

{\ttfamily\mdseries
\ \ \ \ \ \ \}}

{\ttfamily\mdseries
\ \ \ \ \}}


\bigskip

{\ttfamily\mdseries
\ \ \ \ nleft -= nread;}

{\ttfamily\mdseries
\ \ \ \ ptr += nread;}

{\ttfamily\mdseries
\ \ \}}

{\ttfamily\mdseries
\ \ return (n -- nleft);}

{\ttfamily\mdseries
\}}


The Unix read() system call can return a positive number, indicating
the number of bytes read, a negative number, indicated error, or zero,
if end-of-file is reached (or a network connection is closed by the
remote host). We consider the latter two cases exceptional, and ask
exceptf what we should do. An exceptf function is normally a large
switch with one case for each exception. For TP\_EREAD, it says:

{\ttfamily\mdseries
\ \ \ case TP\_EREAD:}

{\ttfamily\mdseries
\ \ \ \ \ \ if (errno == EINTR) \{}

{\ttfamily\mdseries
\ \ \ \ \ \ \ \ \ return TP\_TRYAGAIN;}

{\ttfamily\mdseries
\ \ \ \ \ \ \ \ \ \}}

{\ttfamily\mdseries
\ \ \ \ \ \ else \{}

{\ttfamily\mdseries
\ \ \ \ \ \ \ \ \ ssize\_t nread = (*(ssize\_t*)obj);}

{\ttfamily\mdseries
\ \ \ \ \ \ \ \ \ if (nread == 0) \{ /* EOF */}

{\ttfamily\mdseries
\ \ \ \ \ \ \ \ \ \ \ \ return TP\_DEFAULT;}

{\ttfamily\mdseries
\ \ \ \ \ \ \ \ \ \ \ \ \}}

{\ttfamily\mdseries
\ \ \ \ \ \ \ \ \ else \{}

{\ttfamily\mdseries
\ \ \ \ \ \ \ \ \ \ \ \ return TP\_RETURNERROR;}

{\ttfamily\mdseries
\ \ \ \ \ \ \ \ \ \ \ \ \}}

{\ttfamily\mdseries
\ \ \ \ \ \ \ \ \ \}}


This may not seem very revolutionary, after all the code that calls
exceptf and branches on its result is just as long as the exception
handler itself. We aren't even gaining much code-reuse over the
conventional method, which wraps system calls in another function with
names like Read(). The real win here lies in the ability of the
caller to replace or extend exceptf at runtime. You may have noticed
that there is no code above to output an error message, unixread()
simply returns -1 on errors. In fact, \textit{the} standard and
expected way to output errors is to override exceptf. The wtrace
example shown [XXX: at the end somewhere?] uses the following:

{\ttfamily\mdseries
Tpexcept\_f tpexcept;}

{\ttfamily\mdseries
Tpdisc\_t disc;}


\bigskip

{\ttfamily\mdseries
int exception(int e, void* obj, Tpdist\_t* disc)}

{\ttfamily\mdseries
\{}

{\ttfamily\mdseries
\ \ \ int rc = tpexcept(e, obj, disc);}

{\ttfamily\mdseries
\ \ \ if (rc == TP\_RETURNERROR) \{}

{\ttfamily\mdseries
\ \ \ \ \ \ if (errno != 0) \{}

{\ttfamily\mdseries
\ \ \ \ \ \ \ \ \ perror(url);}

{\ttfamily\mdseries
\ \ \ \ \ \ \ \ \ \}}

{\ttfamily\mdseries
\ \ \ \ \ \ else \{}

{\ttfamily\mdseries
\ \ \ \ \ \ \ \ \ switch (e) \{}

{\ttfamily\mdseries
\ \ \ \ \ \ \ \ \ \ \ \ case TP\_HOST:}

{\ttfamily\mdseries
\ \ \ \ \ \ \ \ \ \ \ \ \ \ \ fputs(url, stderr);}

{\ttfamily\mdseries
\ \ \ \ \ \ \ \ \ \ \ \ \ \ \ fputs({\textquotedbl}: Unknown host{\textbackslash}n{\textquotedbl}, stderr);}

{\ttfamily\mdseries
\ \ \ \ \ \ \ \ \ \ \ \ \ \ \ break;}

{\ttfamily\mdseries
\ \ \ \ \ \ \ \ \ \ \ \ default:}

{\ttfamily\mdseries
\ \ \ \ \ \ \ \ \ \ \ \ \ \ \ fputs(url, stderr);}

{\ttfamily\mdseries
\ \ \ \ \ \ \ \ \ \ \ \ \ \ \ fputs({\textquotedbl}: Error connecting{\textbackslash}n{\textquotedbl}, stderr);}

{\ttfamily\mdseries
\ \ \ \ \ \ \ \ \ \ \ \ \ \ \ \}}

{\ttfamily\mdseries
\ \ \ \ \ \ \ \ \ \}}

{\ttfamily\mdseries
\ \ \ \ \ \ exit(1); }

{\ttfamily\mdseries
\ \ \ \ \ \ \}}

{\ttfamily\mdseries
\ \ \ else \{}

{\ttfamily\mdseries
\ \ \ \ \ \ return rc;}

{\ttfamily\mdseries
\ \ \ \ \ \ \}}

{\ttfamily\mdseries
\}}


\bigskip


Then instead of the usual:

{\ttfamily\mdseries
tp = tp\_new({\textless}uri{\textgreater}, {\textless}method{\textgreater}, TpdUnix);}


wtrace copies TpdUnix, saves and replaces the default exception
handler, and then uses the copied discipline:

{\ttfamily\mdseries
disc = tp\_newdisc(TpdUnix);}

{\ttfamily\mdseries
tpexcept = disc-{\textgreater}exceptf;}

{\ttfamily\mdseries
disc-{\textgreater}exceptf = exception;}


\bigskip

{\ttfamily\mdseries
tp = tp\_new({\textless}uri{\textgreater}, {\textless}method{\textgreater}, disc);}


In the same way, wtrace also overrides all of the read and write
functions to provide a trace log of HTTP communications.
