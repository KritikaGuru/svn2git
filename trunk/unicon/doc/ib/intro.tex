\clearpage\setcounter{page}{1}\pagestyle{ClintStyleii}

\section{Compendium Introduction}

The implementation of the Icon programming language is venerable.
The implementation documentation traditionally revolved around
the virtual machine and its runtime system; other parts of the
implementation were documented in scattered technical reports or not
at all, other than the source code. This volume changes all that, by
bringing all the implementation documents together in a single volume.

Icon's public-domain implementation is fairly efficient; for example
at one point Keith Waclena of the University of Chicago documented a
factor of 4 or more speed advantage of Icon versus Python on multiple
benchmarks, and that was for the Icon virtual machine interpreter; the
Icon optimizing compiler adds another factor of 2-5 or more in faster
execution speed. The design decisions that achieve Icon's very-high
level language features (such as generators and goal-directed
evaluation) with acceptable performance make for an interesting
study. This book is intended for those wanting to learn the
implementation in order to add features, improve performance, learn
about compilers in general, or glean ideas for their own independent
programming language efforts.

Icon traditionally consisted of a virtual machine translator, a
linker, and a virtual machine interpreter. The translator and linker
were merged long ago, but other tools have been added. The big added
components are the optimizing compiler written by Ken Walker, and the
Unicon translator written by Clint Jeffery. These additions are now a
large part of the story. In the future you might hope to see merger of
components, and gradual replacement of C-based components with ones
written in Unicon.


\subsection*{How Many Compilers?}

The figure below shows two symmetrically-organized sets of tools.
The tools on the left are the compilers end-users employ to translate
Icon or Unicon into executable machine code, while the tools on the
right show how the underlying run-time system needed in order to
execute those programs is built. Of the six rectangles, four are
compilers that perform distinct tasks specific to this programming
language family. The front-end translation tool, named unicon, is a
preprocessor that translates Unicon code into Icon code. Its primary
functions are to translate object-orientation (classes, single and
multiple inheritance, and packages) down to underlying imperative
constructs. Unicon is written in Unicon. Icont and iconc compile Icon
code down to virtual machine and C code, respectively. They share a
few common front-end components, but are largely independent. Iconx is
the name of the Icon (and Unicon) virtual machine, which mostly
consists of a large collection of complex high-level data structure
and I/O facilities which are built-in to these languages. Most of the
source code for iconx is also used in rt.a, the runtime library that
is linked to Icon programs compiled with iconc.

\begin{picture}(400,100)
\thicklines

\put(20,50){\framebox(50,20){unicon}}
\put(111,71){\framebox(50,20){icont}}
\put(111,29){\framebox(50,20){iconc}}
\put(70,60){\vector(2,1){40}}
\put(70,60){\vector(2,-1){40}}

\put(220,50){\framebox(50,20){rtt}}
\put(311,71){\framebox(50,20){iconx}}
\put(311,29){\framebox(50,20){rt.a}}
\put(270,60){\vector(2,1){40}}
\put(270,60){\vector(2,-1){40}}

\end{picture}

{\center Figure CI-1: three compilers for users (left), one (rtt) for
the language implementors}

\clearpage
\bigskip


\bigskip
