\chapter{Preface}

This book is a compendium of all documents that describe the
implementation of the Icon and Unicon programming languages, an
implementation that started with Icon version 3 on a PDP-11 sometime
near the year 1980.

\section*{Organization of This Book}

This book consists of four parts. The first part, Chapters 1-12,
present the core of the implementation, focusing on the Icon virtual
machine interpreter and runtime system. This material was formerly
published as the Implementation of the Icon Programming Language, by
Ralph and Madge T. Griswold; at that time it documented Icon Version
6. Many of the details in this book became obsolete with the rewriting
of the runtime system for Icon Version 8. After long consideration, I
have elected to preserve the authors' style and intent, while updating
it to document Icon Version 9.5 and Unicon Version 12. Blue-colored
text indicates when necessary Unicon issues and differences, so that
Part I remains useful to people who prefer to use the Icon
implementation, not just those working with Unicon.

Part II, in Chapters 13-19, describes the optimizing compiler, iconc,
and the structuring of the runtime system to support it. This work is
the brainchild of Ken Walker, whose dissertation is presented here,
along with his technical reports describing the runtime language RTL
and its translator, rtt. Ken's compiler has been enhanced
significantly by Anthony Jones' B.S. Honors thesis at UTSA on space
reduction techniques that reduce the space cost of type inferencing by
2/3rds, and Mike Wilder's M.S. thesis at NMSU and follow-on work at
Idaho on adapting iconc to support Unicon. These contributions belong
logically to Part II.

Part III describes the implementation of Unicon and the many
extensions that transformed the language from a
string-and-list-processing language into a modern object-oriented,
network-savvy, graphics-rich applications language.  Part IV consists
of essential reference material presented in several Appendixes.

\section*{Acknowledgments}

This book would not be possible without the generous contributions and
consent of the primary authors of the Icon language implementation
documents, Ralph and Madge Griswold, and Kenneth Walker. Ralph
Griswold re-scanned and corrected his Icon implementation book
manuscript in order to place it in the public domain on the web, a
large, selfless, thankless, and valuable undertaking. Ken Walker found
and shared his original nroff dissertation source files.

Susie Jeffery provided crucial assistance in the OCR reconstruction of
Icon implementation book manuscript from the public domain scanned
images. Mike Kemp was a valuable volunteer proofreader in that
effort. Responsibility for remaining typographical errors rests with
me.

Thanks to the rest of the people who contributed code to the Icon and
Unicon Projects over a period of many years, and to those who
contributed while obtaining many Ph.D. and M.S. degrees.

The editor wishes to acknowledge generous support from the National
Library of Medicine. This work was also supported in part by the
National Science Foundation under grants CDA-9633299, \ EIA-0220590
and EIA-9810732, and the Alliance for Minority Participation.


Clinton Jeffery, Moscow ID, Dec 2016

\subsection*{Acknowledgments for Chapters 1-12}


The implementation of Icon described in Part I owes much to previous
work and in particular to implementations of earlier versions of
Icon. Major contributions were made by Cary Coutant, Dave Hanson, Tim
Korb, Bill Mitchell, a Steve Wampler. Walt Hansen, Rob McConeghy, and
Janalee O'Bagy also made significant contributions to this work.


The present system has benefited greatly from persons who have
installed Icon on a variety of machines and operating systems. Rick
Fonorow, Bob Goldberg, Chris Janton, Mark Langley, Rob McConeghy, Bill
Mitchell, Janal O'Bagy, John Polstra, Gregg Townsend, and Cheyenne
Wills have made substantial contributions in this area.

The support of the National Science Foundation under Grants MCS7
01397, MCS79-03890, MCS81-0l916, DCR-8320138, DCR-840183I, at
DCR-8502015 was instrumental in the original conception of Icon and
has bee invaluable in its subsequent development.

A number of persons contributed to this book. Dave Gudeman, Dave
Hanson, Bill Mitchell, Janalee O'Bagy, Gregg Townsend, and Alan Wendt
contributed to the exercises that appear at the ends of chapters and
the projects given in Appendix E. Kathy Cummings, Bill Griswold, Bill
Mitchell, Katie Morse, Mike Tharp, and Gregg Townsend gave the
manuscript careful readings and made numerous suggestions. Janalee
O'Bagy not only read the manuscript but also supplied concepts for
presenting and writing the material on expression evaluation.

Finally, Dave Hanson served as an enthusiastic series editor for this
book. His perceptive reading of the manuscript and his supportive
and constructive suggestions made a significant contribution to the
final result.

\hfill {\em Ralph and Madge Griswold}

\subsection*{Acknowledgments for Chapters 13-24}

I would like to thank Ralph Griswold for acting as my research advisor. He provided the balance of guidance, support,
and freedom needed for me to complete this research. From him I learned many of the technical writing skills I needed
to compose this dissertation. I am indebted to him and the other members of the Icon Project who over the years have
contributed to the Icon programming language that serves as a foundation of this research. I would like to thank Peter
Downey and Saumya Debray for also serving as members on my committee and for providing insightful criticisms and
suggestions for this dissertation. In addition, Saumya Debray shared with me his knowledge of abstract interpretation,
giving me the tool I needed to shape the final form of the type inferencing system. 


I have received help from a number of my fellow graduate students both while they were still students and from some
after they graduated. Clinton Jeffery, Nick Kline, and Peter Bigot proofread this dissertation, providing helpful
comments. Similarly, Janalee O'Bagy, Kelvin Nilsen, and David Gudeman proofread earlier reports that served as a basis
for several of the chapters in this dissertation. Janalee O'Bagy's own work on compiling Icon provided a foundation for
the compiler I developed. Kelvin Nilsen applied my liveness analysis techniques to a slightly different implementation
model, providing insight into dependencies on execution models. 


