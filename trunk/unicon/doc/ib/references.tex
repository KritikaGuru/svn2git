\begin{noIndex}
\clearpage\section{References}

[ASU86] Aho, Alfred; Sethi, Ravi; and Ullman, Jeffrey. Compilers,
Principles Techniques and Tools. Addison-Wesley, 1986.

\noindent
[Andrews88] Gregory R. Andrews, Ronald A. Olsson et al. An Overview of the SR
Language and Implementation. TOPLAS 10:1, January 1988, pp 51-86.

\noindent
[Bartlett 89] J. Bartlett. SCHEME-{\textgreater}C a Portable
Scheme-to-C Compiler. Research Report 89/1. DEC Western Research
Laboratory, January 1989.

\noindent
[Foley82] Foley, J.D; and A.Van Dam. Fundamentals of Interactive
Computer Graphics. Reading, MA: Addison-Wesley Publishing Company,
1982.

\noindent
[Griswold93] Griswold, William G.; and Townsend, Gregg M.
The design and implementatoin of dynamic hashing for sets
and tables in icon. Software: Pracice and Experience 23:4,
April 1993, pp. 351-367.

\noindent
[Griswold96] Griswold, Ralph E and Griswold, Madge T. The Icon
Programming Language, Third Edition. San Jose, CA: Peer-To-Peer
Communications, 1996.

\noindent
[Griswold98] Griswold, Ralph E.; Jeffery, Clinton L.; and Townsend,
Gregg M. Graphics Programming in Icon. San Jose, CA: Peer-To-Peer
Communications, 1998.

\noindent
[Griswold71] Griswold, Poage, and Polonsky . The SNOBOL 4 Programming
Language, 2nd ed. Englewood Cliffs, N.J.  Prentice-Hall, Inc. 1971.

\noindent
[Jeffery99] Clinton L. Jeffery. Program Monitoring and Visualization:
An Exploratory Approach.  Springer-Verlag, New York, NY. 1999.

\noindent
[Jeffery04] Jeffery, Clinton; Mohamed, Shamim; Pereda, Ray; and
Parlett, Robert. Programming with Unicon. Draft manuscript from
http://unicon.org

\noindent
[LeHors96] Arnaud LeHors. XPM Manual. Groupe Bull, Koala
Project, INRIA, France, 1996. Technical report at
\url{www.xfree86.org/current/xpm.pdf}

\noindent
[Nye88] Adrian Nye, editor. Xlib Reference Manual. O'Reilly \&
Associates, Inc., Sebastopol, California, 1988.

\noindent
[OpenGL99] OpenGL Architecture Review Board; Woo, Mason; Neider,
Jackie; Davis, Tom; Shreiner, Dave. OpenGL Programming Guide: the
Official Guide to Learning OpenGL, Third Edition. Reading, MA:
Addison-Wesley Publishing Company, 1999.

\noindent
[OpenGL00] OpenGL Architecture Review Board; Shreiner, Dave. OpenGL
Programming Guide: the Official Reference Document to OpenGL, Third
Edition. Upper Saddle Reading, MA: Addison-Wesley Publishing Company,
2000.

\noindent
[Rees 86] Jonathan Rees, William Clinger. et al. Revised Report on the
Algorithmic Language Scheme. SIGPLAN Notices, 21:12, December 1986.

\noindent
[TGJ96] Gregg M. Townsend, Ralph E. Griswold, and Clinton
L. Jeffery. Configuring the Source Code for Version 9 of Icon;
Technical Report IPD238c, Department of Computer Science, University
of Arizona, April 1996.
\url{www.cs.arizona.edu/icon/docs/ipd238.htm}.

\noindent
[TGJ98] Gregg M. Townsend, Ralph E. Griswold, and Clinton
L. Jeffery. Installing Version 9 of Icon on UNIX Platforms; Technical
Report IPD243e, Department of Computer Science, University of Arizona,
February 1998.
\url{www.cs.arizona.edu/icon/docs/ipd243.htm}.

\noindent
[Uhl88] Stephen A. Uhler. MGR --- C Language Application
Interface. Technical report, Bell Communications Research, July 1988.

\noindent
[Walker94] Kenneth Walker. The Run-Time Implementation Language for Icon;
\url{www.cs.arizona.edu/icon/ftp/doc/ipd261.pdf}. Technical
Report IPD261, Department of Computer Science, University of Arizona,
June 1994.

\noindent
[Yuasa] T. Yuasa and M. Hagiya. Kyoto Common Lisp Report. Research
Institute for Mathematical Sciences, Kyoto University

\noindent
[Weiner] J.L. Weiner and S. Ramakrishnan. A Piggy-back Compiler for
Prolog. Proceeding of the 1988 Conference on Programming Language
Design and Implementation, SIGPLAN Notices 23:7, July 1988,
pp. 288-295.

\noindent
[Stroustrup 86] B. Stroustrup. The C++ Programming
Language. Addison-Wesley, 1986.

\noindent
[peephole] Andrew S. Tanenbaum, Hans van Staveren, and Johan
W. Stevenson. Using Peephole Optimization on Intermediate Code. TOPLAS
4:1, January 1982.

\noindent
[Wulf] William A. Wulf, Richard. K. Johnsson, Charles. B. Weinstock,
Steven. O. Hobbs, Charles. M. Geschke. The Design of an Optimizing
Compiler. American Elsevier Pub. Co., New York, 1975.

\noindent
[denote] M. J. C. Gordon. The Denotational Description of Programming
Languages, An Introduction. Springer, 1979.

\noindent
[Stoy] J. E. Stoy. Denotational Semantics: The Scott-Strachey Approach
to Programming Language Theory. MIT Press, Cambridge, 1977.

\noindent
[ansi-c] American National Standard for Information
Systems. Programming Language - C, ANSI X3.159-1989. American National
Standards Institute, New York, 1990.

\noindent
[Prabhala] Bhaskaram Prabhala and Ravi Sethi. Efficient Computation of
Expressions with Common Subexpressions. Fifth Annual ACM Symposium on
Principles of Programming Languages, pp. 222-230, January 1978.

\noindent
[Nilsson] J{\o}rgen Fischer Nilsson. On the Compilation of a
Domain-Based Prolog. Information Processing; Richard Edward Allison
Mason ed., North-Holland, 1983, pp. 293-299.

\noindent
[Martinek] John Martinek and Kelvin Nilsen. Code Generation for the
Temporary-Variable Icon Virtual Machine. Technical Report 89-9,
Department of Computer Science, Iowa State University, December 1989.

\noindent
[pntstr] David R. Chase, Mark Wegman, and F. Kenneth Zadeck. Analysis
of Pointers and Structures. Proceeding of the 1990 Conference on
Programming Language Design and Implementation, SIGPLAN Notices 25:6,
June 1990, pp. 296-310.

\noindent
[depptr] Susan Horwitz, Phil Pfeiffer, and Thomas Reps. Dependence
Analysis for Pointer Variables. Proceeding of the 1989 Conference on
Programming Language Design and Implementation, SIGPLAN Notices 24:7,
July 1989, pp. 28-40.

\noindent
[smltlk type] Norihisa Suzuki. Inferring Types in Smalltalk. Eighth
Annual ACM Symposium on Principles of Programming Languages,
pp. 187-199, January 1981.

\noindent
[Milner] Robin Milner. A Theory of Type Polymorphism in
Programming. Journal of Computer and System Sciences. 17:3, December
1978, pp. 348-375.

\noindent
[unify] J. A. Robinson, A Machine-Oriented Logic Based on the
Resolution Principle. JACM, 12:1, January 1965, pp.  23-41.

\noindent
[ianl1] Ralph E. Griswold and Madge T. Griswold. The Icon Analyst \#1,
August 1990.

\noindent
[johnk] John Kececioglu. Private Communication. November 1990.

\noindent
[debray apr91] Saumya K. Debray. Private Communication. April 1991.

\noindent
[wam] D. H. D. Warren. An Abstract Prolog Instruction Set. Technical
Note 309, SRI International, Menlo Park, CA, October 1983.
\end{noIndex}
