\chapter{The Implementation of Type Inferencing}

Chapter 15 develops a theoretical type inferencing model for Icon,
called Model 3. The purpose of that chapter is to explain the
relationship between type inferencing and the semantics of Icon; for
simplicity, some features of the language along with certain questions
of practical importance are ignored in that chapter. This chapter
describes the implementation of the type inferencing system used in
the Icon compiler. The implementation is based on Model 3. This
chapter describes solutions to those issues that must be addressed in
developing a complete practical type inferencing system from Model
3. The issues include:

\liststyleLxxi
\begin{itemize}
\item 
the representation of types and stores 
\item 
the development of a type system for the full Icon language 
\item 
the handing of procedure calls and co-expression activation 
\item 
the determination of edges in the flow graph 
\item 
the computation of a fixed point for type information 
\end{itemize}

In addition, performance of the abstract interpretation must be
considered. This includes both speed and memory usage.


\section[19.1 The Representation of Types and Stores]{19.1 The Representation of Types and Stores}

A type consists of a set of basic types (technically, it is a union of
basic types, but the constituents of the basic types are not
explicitly represented). The operations needed for these sets are: add
a basic type to a set, form the union of two sets, form the
intersection of two sets, test for membership in a set, and generate
members of a subrange of basic types (for example, generate all
members that are list types). A bit vector is used for the set
representation, with a basic type represented by an integer index into
the vector. The required operations are simple and efficient to
implement using this representation. Unless the sets are large and
sparse, this representation is also space efficient. While the sets of
types are often sparse, for typical programs, they are not large.

A store is implemented as an array of pointers to types. A mapping is
established from variable references to indexes in the store. In the
type inferencing model, Model 3, presented in Chapter 3, there is one
kind of store that contains all variables. In the actual
implementation, temporary variables need not be kept in this
store. The purpose of this store is to propagate a change to a
variable to the program points affected by the change. A temporary
variable is set in one place in the program and used in one place;
there is nothing to determine dynamically. It is both effective and
efficient to store the type of a temporary variable in the
corresponding node of the syntax tree.

Another level of abstraction can be developed that requires much less
memory than Model 3, but it must be modified to produce good
results. This abstraction abandons the practice of a store for every
edge in the flow graph. Instead it has a single store that is a merger
of all the stores in Model 3 (the type associated with a variable in a
merged store is the union of the types obtained for that variable from
each store being merged). For variables that are truly of one type
throughout execution, this abstraction works well. Named variables do
not have this property. They have an initial null value and usually
are assigned a value of another type during execution. Because
assignments to named variables are treated as strong updates, Model 3
can often deduce that a variable does not contain the null type at
specific points in the flow graph.

For structure variables this further abstraction does work well in
practice. These variables are initialized to the empty type (that is,
no instances of these variables exist at the start of program
execution), so the problem of the initial null type does not
occur. Sometimes instances of these variables are created with the
null type and later changed. However, the fact that assignments to
these variables must be treated as weak updates means that type
inferencing must assume that these variables can always retain their
earlier type after an assignment. Propagating type information about
structure variables through the syntax tree does not help much in
these circumstances. While it is possible to construct example
programs where Model 3 works better for structure variables than this
further abstraction, experiments with prototype type inferencing
systems indicate that the original system seldom gives better
information for real programs [.tr88-25.].

Type inferencing in the compiler is implemented with two kinds of
stores: local stores that are associated with edges in the flow graph
and contain named variables (both local and global variables) and a
global store that contains structure variables (in the implementation,
the global store is actually broken up by structure-variable type into
several arrays).


\section[19.2 A Full Type System]{19.2 A Full Type System}

Model 3 from Chapter 3 includes no structure type other than lists,
nor does it consider how to handle types for procedure and
co-expression values to allow effective type inferencing in their
presence. This section develops a complete and effective type system
for Icon.

Most of the structure types of Icon are assigned several subtypes in
the type inferencing system. As explained for lists in Chapter 3,
these subtypes are associated with the program points where values of
the type are created. The exception to this approach is records. One
type is created per record declaration. While it is possible to assign
a subtype to each use of a record constructor, in practice a given
kind of record usually is used consistently with respect to the types
of its fields throughout a program. The extra subtypes would require
more storage while seldom improving the resulting type information.

For efficiency, the size of the bit vectors representing types and the
size of the stores need to remain fixed during abstract
interpretation. This means that all subtypes must be determined as
part of the initialization of the type inferencing system. In order to
avoid excessive storage usage, it is important to avoid creating many
subtypes for program points where structures are not created. The
invocation optimization described in Chapter 6 helps determine where
structures can and cannot be created by determining for most
invocations what operation is used. The type inferencing system
determines what structures an operation can create by examining the
abstract type computations associated with the operation in the data
base. A new clause in an abstract type computation indicates that a
structure can be created. The following example is the abstract type
computation from the built-in function list. It indicates the the
function creates and returns a new list with elements whose type is
the same as that of the parameter x (the second parameter).

{\ttfamily\mdseries
\ \ \ abstract \{}

{\ttfamily\mdseries
\ \ \ \ \ \ return new list(type(x))}

{\ttfamily\mdseries
\ \ \ \ \ \ \}}

This is the clause as written by the programmer developing the
run-time library; it is translated into an internal form for storage
in the data base.

Invocation optimizations may not identify the operation associated
with some invocations. The initialization phase of type inferencing
skips these invocations. Type inferencing may later discover that one
of these invocations can create a structure. Each structure type is
given one subtype that is used for all of these later
discoveries. While it is possible for there to be several of these
creation points representing logically distinct subtypes, this seldom
occurs in practice. If it does happen, type inferencing produces a
correct, but less precise, result.

The type system contains representations for all run-time values that
must be modeled in the abstract interpretation.  These run-time values
can be divided into three categories, each of which is a superset of
the previous category:

\liststyleLxxii
\begin{itemize}
\item 
the first-class Icon values 
\item 
the intermediate values of expression evaluation 
\item 
the values used internally by Icon operations 
\end{itemize}

Just as these categories appear in different places during the
execution of an Icon program, the corresponding types appear in
different places during abstract interpretation. If certain types
cannot appear as the result of a particular type computation, it is
not necessary to have elements in the bit vector produced by the
computation to represent those types. It is particularly important to
minimize the memory used for stores associated with edges of the flow
graph (this is discussed more in the last section of this
chapter). These stores contain only the types of the smallest set
listed above: the first-class values.

Types (or subtypes) are allocated bit vector indexes. The first-class
types may appear as the result of any of the three classes of
computation. They are allocated indexes at the front of the bit
vectors. If they are the only types that can result from an abstract
computation, the bit vector for the result has no elements beyond that
of the last first-class types. The first-class types are:

\liststyleLxxiii
\begin{itemize}
\item 
null 
\item 
string 
\item 
cset 
\item 
integer 
\item 
real 
\item 
file 
\item 
list subtypes 
\item 
set subtypes 
\item 
table subtypes 
\item 
record subtypes 
\item 
procedure subtypes 
\item 
co-expression subtypes 
\end{itemize}

The list subtypes are allocated with 

\liststyleLxxiv
\begin{itemize}
\item 
one for the argument to the main procedure 
\item 
one for each easily recognized creation point 
\item 
one representing all other lists 
\end{itemize}

The set subtypes are allocated with 

\liststyleLxxv
\begin{itemize}
\item 
one for each easily recognized creation point 
\item 
one representing all other sets 
\end{itemize}

The table subtypes are allocated with 

\liststyleLxxvi
\begin{itemize}
\item 
one for each easily recognized creation point 
\item 
one representing all other tables 
\end{itemize}

The record subtypes are allocated with one for each record
declaration. The procedure subtypes are allocated with

\liststyleLxxvii
\begin{itemize}
\item 
one for each procedure 
\item 
one for each record constructor 
\item 
one for each built-in function 
\item 
one representing operators available for string invocation 
\end{itemize}

Note that procedure subtypes are allocated after most procedure and
function values have been eliminated by invocation optimizations (the
procedures and functions are still there, they are just not Icon
values). Therefore, few of these subtypes are actually allocated. The
co-expression subtypes are allocated with

\liststyleLxxviii
\begin{itemize}
\item 
one for the main co-expression 
\item 
one for each create expression 
\end{itemize}

The bit vectors used to hold the intermediate results of performing
abstract interpretation on expressions must be able to represent the
basic types plus the variable reference types. Variable reference
types are allocated bit vector indexes following those of the basic
types. The bit vectors for intermediate results are just long enough
to contain these two classifications of types. The variable reference
types are

\liststyleLxxix
\begin{itemize}
\item 
integer keyword variable types 
\item 
\&pos 
\item 
\&subject 
\item 
substring trapped variable types 
\item 
table-element trapped variable types 
\item 
list-element reference types 
\item 
table assigned-value reference types 
\item 
field reference types 
\item 
global variable reference types 
\item 
local variable reference types 
\end{itemize}

\&random and \&trace behave the same way from the perspective of the
type inferencing system, so they are grouped under one type as integer
keyword variables. The fact that \&pos can cause assignment to fail is
reflected in the type inferencing system, so it is given a separate
type. \&subject is the only string keyword variable so it is in a type
by itself.

Substring trapped variables and table-element trapped variables are
{\textasciigrave}{\textasciigrave}hidden'{}' structures in the
implementation of Icon. They appear as intermediate results, but are
only indirectly observable in the semantics of Icon. In order to
reflect these semantics in the type inferencing system, there are
substring trapped variable types and table-element trapped variable
types. These are structure types similar to sets, but are variable
reference types rather than first-class types. The substring trapped
variable types are allocated with

\liststyleLxxx
\begin{itemize}
\item 
one for each easily recognized creation point 
\item 
one representing all other substring trapped variables 
\end{itemize}

The table-element trapped variable types are allocated with 

\liststyleLxxxi
\begin{itemize}
\item 
one for each easily recognized creation point 
\item 
one representing all other table-element trapped variables 
\end{itemize}

List elements, table assigned-values, and record fields are all
variables that can appear as the intermediate results of expression
evaluation. The type system has corresponding variable reference types
to represent them. The list-element reference types are allocated with
one for each list types. The table assigned-value reference types are
allocated with one for each table type. The field reference types are
allocated with one for each record field declaration.

Identifiers are variables and are reflected in the type system. The
global variable reference types are allocated with

\liststyleLxxxii
\begin{itemize}
\item 
one for each global variable (except those eliminated by invocation optimizations). 
\item 
one for each static variable 
\end{itemize}

The local variable reference types are allocated with one for each
local variable, but the bit vector indexes and store indexes are
reused between procedures. The next section describes why this reuse
is possible.

Icon's operators use a number of internal values that never
{\textasciigrave}{\textasciigrave}escape'{}' as intermediate
results. Some of these internal values are reflected in the type
system in order to describe the semantics of the operations in the
abstract interpretation. The full set of types that can be used to
express these semantics are presented in the syntax of the abstract
type computations of the run-time implementation language; see
Appendix A. In addition to the types of intermediate results, these
types include

\liststyleLxxxiii
\begin{itemize}
\item 
set-element reference types 
\item 
table key reference types 
\item 
table default value reference types 
\item 
references to the fields within substring trapped variables that reference variables 
\item 
references to fields within table-element trapped variables that reference tables 
\end{itemize}

These types are allocated bit vector indexes at the end of the type
system. The only bit vectors large enough to contain them are the
temporary bit vectors used during interpretation of the abstract type
computations of built-in operations.

Set elements, table keys, and table default values do not appear as
variable references in the results of expression evaluation. However,
it is necessary to refer to them in order to describe the effects of
certain Icon operations. For this reason, they are included in the
type system. The set-element reference types are allocated with one
for each set type. The table key reference types are allocated with
one for each table type. The table default value reference types are
allocated with one for each table type.

Substring trapped variable types contain references to the variable
types being trapped and table-element trapped variable types contain
references to the table types containing the element being
trapped. These references are fields within these trapped variable
types. There is one field reference type for each trapped variable
type.


\section[19.3 Procedure Calls and Co-Expression Activations]{19.3 Procedure Calls and Co-Expression Activations}

A type inferencing system for the full Icon language must handle
procedures and co-expressions. As noted above, each procedure and each
create expression is given its own type. This allows the type
inferencing system to accurately determine which procedures are
invoked and what co-expressions might be activated by a particular
expression.

The standard semantics for procedures and co-expressions can be
implemented using stacks of procedure activation frames, with one
stack per co-expression. The first frame, on every stack except that
of the main co-expression, is a copy of the frame for the procedure
that created the co-expression. The local variables in this frame are
used for evaluating the code of the co-expression. The type
inferencing system uses a trivial abstraction of these procedure frame
stacks, while capturing the possible transfers of control induced by
procedure calls and co-expression activations.

The type inferencing system, in effect, uses an environment that has
one frame per procedure, where that frame is a summary of all frames
for the procedure that could occur in a corresponding environment of
an implementation of the standard semantics. The frame is simply a
portion of the store that contains local variables. Because no other
procedure can alter a local variable, it is unnecessary to pass the
types of local variables into procedure calls. If the called procedure
returns control via a return, suspend, or fail, the types are
unchanged, so they can be passed directly across the call. This allows
the type inferencing system to keep only the local variables of a
procedure in the stores on the edges of the flow graph for the
procedure, rather than keeping the local variables of all procedures.
Global variables must be passed into and out of procedure
calls. Because static variables may be altered in recursive calls,
they must also be passed into and out of procedure calls.

A flow graph for an entire program is constructed from the flow graphs
for its individual procedures and co-expressions.  An edge is added
from every invocation of a procedure to the head of that procedure and
edges are added from every return, suspend, and fail back to the
invocation. In addition, edges must be added from an invocation of a
procedure to all the suspends in the procedure to represent
resumption. When it is not possible to predetermine which procedure is
being invoked, edges are effectively added from the invocation to all
procedures whose invocation cannot be ruled out based on the naive
invocation optimizations. Edges are effectively added between all
co-expressions and all activations, and between all
activations. Information is propagated along an edge when type
inferencing deduces that the corresponding procedure call or
co-expression activation might actually occur. The representation of
edges in the flow graph is discussed below.

Type inferencing must reflect the initializations performed when a
procedure is invoked. Local variables are initialized to the null
value. On the first call to the procedure, static variables are also
initialized to the null value. The initialization code for the
standard semantics is similar to

{\ttfamily\mdseries
\ \ \ \textit{initialize locals}}

{\ttfamily\mdseries
\ \ \ if (first\_call) \{}

{\ttfamily\mdseries
\ \ \ \ \ \ initialize statics}

{\ttfamily\mdseries
\ \ \ \ \ \ user initialization code}

{\ttfamily\mdseries
\ \ \ \ \ \ \}}

In type inferencing, the variables are initialized to the null type
and the condition on the if is ignored. Type inferencing simply knows
that the first-call code is executed sometimes and not others. Before
entering the main procedure, global variables are set to the null type
and all static variables are set to the empty type. In some sense, the
empty type represents an impossible execution path. Type inferencing
sees paths in the flow graph from the start of the program to the body
of a procedure that never pass through the initialization
code. However, static variables have an empty type along this path and
no operation on them is valid. Invalid operations contribute nothing
to type information.


\section[19.4 The Flow Graph and Type Computations]{19.4 The Flow Graph and Type Computations}

In order to determine the types of variables at the points where they
are used, type inferencing must compute the contents of the stores
along edges in the flow graph. Permanently allocating the store on
each edge can use a large amount of memory. The usage is

{\ttfamily\mdseries
\ \ \ M = E (G + S + L) T / 8}

\noindent where 

{\ttfamily\mdseries
\ \ \ M = total memory, expressed in bytes, used by stores}

{\ttfamily\mdseries
\ \ \ E = the number of edges in the program flow graph}

{\ttfamily\mdseries
\ \ \ G = the number of global variables in the program}

{\ttfamily\mdseries
\ \ \ S = the number of static variables in the program}

{\ttfamily\mdseries
\ \ \ L = the maximum number of local variables in any procedure}

{\ttfamily\mdseries
\ \ \ T = the number of types in the type system}

Large programs with many structure creation points can produce
thousands of edges, dozens of variables, and hundreds of types,
requiring megabytes of memory to represent the stores.

The code generation phase of the compiler just needs the (possibly
dereferenced) types of operands, not the stores. If dereferenced types
are kept at the expressions where they are needed, it is not necessary
to keep a store with each edge of the flow graph.

Consider a section of straight-line code with no
backtracking. Abstract interpretation can be performed on the graph
starting with the initial store at the initial node and proceeding in
execution order. At each node, the store on the edge entering the node
is used to dereference variables and to compute the next store if
there are side effects. Once the computations at a node are done, the
store on the edge entering the node is no longer needed. If updates
are done carefully, they can be done in-place, so that the same memory
can be used for both the store entering a node and the one leaving it.

In the case of branching control paths (as in a case expression),
abstract interpretation must proceed along one path at a time. The
store at the start the branching of paths must be saved for use with
each path. However, it need only be saved until the last path is
interpreted. At that point, the memory for the store can be
reused. When paths join, the stores along each path must be
merged. The merging can be done as each path is completed; the store
from the path can then be reused in interpreting other paths. When all
paths have been interpreted, the merged store becomes the current
store for the node at the join point.

The start of a loop is a point where control paths join. Unlike
abstract interpretation for the joining of simple branching paths,
abstract interpretation for the joining of back edges is not just a
matter of interpreting all paths leading to the join point before
proceeding. The edge leaving the start of the loop is itself on a path
leading to the start of the loop. Circular dependencies among stores
are handled by repeatedly performing the abstract interpretation until
a fixed point is reached. In this type inferencing system, abstract
interpretation is performed in iterations, with each node in the flow
graph visited once per iteration. The nodes are visited in execution
order. For back edges, the store from the previous iteration is used
in the interpretation. The stores on these edges must be kept
throughout the interpretation. These stores are initialized to map all
variables to the empty type. This represents the fact that the
abstract interpretation has not yet proven that execution can reach
the corresponding edge.

The type inferencing system never explicitly represents the edges of
the flow graph in a data structure. Icon is a structured programming
language. Many edges are implicit in a tree walk performed in forward
execution order -- the order in which type inferencing is
performed. The location of back edges must be predetermined in order
to allocate stores for them, but the edges themselves are effectively
recomputed as part of the abstract interpretation.

There are two kinds of back edges. The back edges caused by looping
control structures can be trivially deduced from the syntax tree. A
store for such an edge is kept in the node for the control
structure. Other back edges are induced by goal-directed
evaluation. These edges are determined with the same techniques used
in liveness analysis. A store for such an edge is kept in the node of
the suspending operation that forms the start of the loop. Because the
node can be the start of several nested loops, this store is actually
the merged store for the stores that theoretically exist on each back
edge.

At any point in abstract interpretation, three stores are of
interest. The \textit{current store} is the store entering the node on
which abstract interpretation is currently being performed. It is
created by merging the stores on the incoming edges. The
\textit{success store} is the store representing the state of
computations when the operation succeeds. It is usually created by
modifying the current store. The \textit{failure store} is the store
representing the state of computations when the operation fails.

In the presence of a suspended operation, the failure store is the
store kept at the node for that operation. A new failure store is
established whenever a resumable operation is encountered. This works
because abstract interpretation is performed in forward execution
order and resumption is LIFO. Control structures, such as
if-then-else, with branching and joining paths of execution, cause
difficulties because there may be more than one possible suspended
operation when execution leaves the control structure. This results in
more than one failure store during abstract interpretation. Rather
than keeping multiple failure stores when such a control structure has
operations that suspend on multiple paths, type inferencing pretends
that the control structure ends with an operation that does nothing
other than suspend and then fail. It allocates a store for this
operation in the node for the control structure. When later operations
that fail are encountered, this store is updated. The failure of this
imaginary operation is the only failure seen by paths created by the
control structure and the information needed to update the failure
stores for these paths is that in the store for this imaginary
operation. This works because the imaginary operation just passes
along failure without modifying the store.


In the case where a control structure transforms failure into forward
execution, as in the first subexpression of a compound expression, the
failure store is allocated (with empty types) when the control
structure is encountered and deallocated when it is no longer
needed. If no failure can occur, no failure store need be
allocated. The lack of possible failure is noted while the location of
back edges is being computed during the initialization of type
inferencing. Because a failure store may be updated at several
operations that can fail, these are weak updates.  Typically, a
failure store is updated by merging the current store into it.

The interprocedural flow graph described earlier in this chapter has
edges between invocations and returns, suspends, and fails. Type
inferencing does not maintain separate stores for these theoretical
edges. Instead it maintains three stores per procedure that are
mergers of stores on several edges. One store is the merger of all
stores entering the procedure because of invocation; this store
contains parameter types in addition to the types of global and static
variables. Another store is the merger of all stores entering the
procedure because of resumption. The third store is the merger of all
stores leaving the procedure because of returns, suspends, and
fails. There is also a result type associated with the procedure. It
is updated when abstract interpretation encounters returns and
suspends.

Two stores are associated with each co-expression. One is the merger
of all stores entering the co-expression and the other is the merger
of all stores leaving the co-expression. Execution can not only leave
through an activation operator, it can also re-enter through the
activation. The store entering the activation is a merger of the
stores entering all co-expressions in which the activation can
occur. Because a procedure containing an activation may be called from
several co-expressions, it is necessary to keep track of those
co-expressions. A set of co-expressions is associated with each
procedure for this purpose. Each co-expression also contains a type
for the values transmitted to it. The result type of an activation
includes the result types for all co-expressions that might be
activated and the types of all values that can be transmitted to a
co-expression that the activation might be executed in.

When type inferencing encounters the invocation of a built-in
operation, it performs abstract interpretation on the representation
of the operation in the data base. It interprets the type-checking
code to see what paths might be taken through the operation. The
interpretation uses the abstract type computations and ignores the
detailed C code when determining the side effects and result type of
the operation. Because the code at this level of detail contains no
loops, it is not necessary to save stores internal to operations. An
operation is re-interpreted at each invocation.  This allows type
inferencing to produce good results for polymorphous operations. At
this level, the code for an operation is simple enough that the cost
of re-interpretation is not prohibitive. All side effects within these
operations are treated as weak updates; the only strong updates
recognized by type inferencing are the optimized assignments to named
variables (see Chapter 6).

The abstract semantics of control structures are hard-coded within the
type inferencing system. The system combines all the elements
described in this chapter to perform the abstract interpretation. A
global flag is set any time an update changes type information that is
used in the next iteration of abstract interpretation. The flag is
cleared between iterations. If the flag is not set during an
iteration, a fixed point has been reached and the interpretation
halts.

\bigskip
