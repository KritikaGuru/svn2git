\chapter{Performance of Compiled Code}

The performance of compiled code is affected by the various
optimizations performed by the compiler. This chapter demonstrates the
effects of these optimizations on the execution speed of Icon
expressions. It also presents speed improvements and memory usage for
compiled code versus interpreted code for a set of complete Icon
programs. All timing results used in this chapter were obtained on a
Sun 4/490 and are the average of the results from three runs.


\section[23.1 Expression Optimizations]{23.1 Expression Optimizations}

The effects of four categories of optimization are demonstrated. These
are assignment optimizations, invocation optimizations, control flow
optimizations, and optimizations using information from type
inferencing. Expression timings for the first three categories were
made using techniques described in the August 1990 issue of
\textit{The Icon Analyst} [.ianl1.]. The following program skeleton is
used to construct the programs to perform these timings.

\goodbreak
\begin{iconcode}
\>procedure main()\\
\>\>local x, start, overhead, iters\\
\>\>iters := 1000000\\
\>\>start := \&time\\
\>\>every 1 to iters do \{\\
\>\>\>\}\\
\>\>overhead := \&time - start\\
\>\>x := 0\\
\>\>start := \&time\\
\>\>every 1 to iters do \{\\
\>\>\>\textit{expression to be timed (may use x)}\\
\>\>\>\}\\
\>\>write(\&time - start - overhead)\\
\>end\\
\end{iconcode}

The timings are performed both with and without the desired
optimizations, and the results are compared by computing the ratio of
the time without optimization to the time with optimization.

The assignment optimizations are described in Chapter 22. The effect
of the assignment optimizations on the expression

\iconline{ \>x := \&null }

\noindent is measured using the program outlined above. The analysis
that produces the assignment optimization is disabled by enabling
debugging features in the generated code. The only other effect this
has on the assignment expression is to insert code to update the line
number of the expression being executed. In this test, the line number
code is removed before the C code is compiled, insuring that the
assignment optimization is the only thing measured. The timing results
for this test produce

\tablefirsthead{}
\tablehead{}
\tabletail{}
\tablelasttail{}
\begin{center}
\begin{tabular}{@{}r@{\hspace{0.6in}}r@{\hspace{0.2in}}r@{}}
\multicolumn{3}{c}{Assignment Test}\\
\multicolumn{3}{c}{Time in Milliseconds Averaged over Three Runs}\\
 Unoptimized & Optimized & Ratio\\
 1122  & 478  & 2.3 \\
\end{tabular}
\end{center}

The tests were performed with type inferencing enabled. Therefore,
even the {\textquotedbl}unoptimized{\textquotedbl} version of the
assignment has the standard operation optimizations applied to
it. This test demonstrates the importance of performing the
special-case assignment optimizations.

The next category of optimization measured is invocation
optimization. This results in the direct invocation of the C functions
implementing operations, or in some cases results in the operations
being generated in line. The execution time for the expression

\iconline{ \>\>\ tab(0) }

\noindent is measured with and without invocation optimizations. As
with the assignment optimizations, these optimizations are disabled by
enabling debugging features. Once again the line number code is
removed before the C code is compiled.  These optimizations interact
with the optimizations that use information from type inferencing. The
measurements are made with type inferencing disabled. Therefore, no
type checking simplifications are performed. Without the invocation
optimizations, the generated code consists of an indirect invocation
through the global variable \texttt{tab}. With the invocation optimizations,
the generated code consists of type checking/conversion code for the
argument to \texttt{tab} and a call to the function implementing the body
statement of \texttt{tab}. The timing results for \texttt{tab(0)} produce

\begin{center}
\begin{tabular}{@{}r@{\hspace{0.6in}}r@{\hspace{0.2in}}r@{}}
\multicolumn{3}{c}{Invocation Test}\\
\multicolumn{3}{c}{Time in Milliseconds Averaged over Three Runs}\\
 Unoptimized & Optimized & Ratio\\
 8394  & 4321  & 1.9\\
\end{tabular}
\end{center}

The third category of optimization is control flow optimization. As
explained in Chapter 21, these optimizations only perform improvements
that a C compiler will not perform when the code contains trivial call
chains. One situation that produces trivial call chains is nested
alternation. The execution time for the expression

\iconline{ \>\>every x := ixor(x, 1 | 2 | 3 | 4 | 5) }

\noindent is measured with and without control flow optimizations. The
timing results for this every loop produce

\begin{center}
\begin{tabular}{@{}r@{\hspace{0.6in}}r@{\hspace{0.2in}}r@{}}
\multicolumn{3}{c}{Control Flow Test}\\
\multicolumn{3}{c}{Time in Milliseconds Averaged over Three Runs}\\
Unoptimized & Optimized & Ratio\\
 6384  & 4184  & 1.5 \\
\end{tabular}
\end{center}

The final category of optimization results from type inferencing. The
speed improvements result from generating operations in line,
eliminating type checking, and generating success continuations in
line. Use of the to operation is a good example of where these
optimizations can be applied. This is demonstrated by measuring the
speed of an every loop using the to operation. The program that
performs the measurement is

\goodbreak
\begin{iconcode}
\>procedure main()\\
\>\>local x, start\\
\>\>start := \&time\\
\>\>every x := 1 to 5000000\\
\>\>write(\&time - start)\\
\>end\\
\end{iconcode}

The timing results for this program produce 

\begin{center}
\begin{tabular}{@{}r@{\hspace{0.6in}}r@{\hspace{0.2in}}r@{}}
\multicolumn{3}{c}{Type Inference Test}\\
\multicolumn{3}{c}{Time in Milliseconds Averaged over Three Runs}\\
Unoptimized & Optimized & Ratio\\
 9233  &
 2721  &
 3.3 \\
\end{tabular}
\end{center}

Another approach to determining the effectiveness of type inferencing
is to measure how small a set it deduces for the possible types of
operands to operations. This indicates whether future work should
concentrate on improving type inferencing itself or simply concentrate
on using type information more effectively in code generation. A
simple measure is used here: the percentage of operands for which type
inferencing deduces a unique Icon type. Measurements are made for
operands of all operators, except optimized assignment, and for
operands of all built-in functions appearing in optimized
invocations. For the most part, these are the operations where the
code generator can use type information. Measurements were made for a
set of 14 programs (described below). Unique operand types within each
program range from 63 percent to 100 percent of all operands, with an
overall figure for the tests suite of 80 percent (this is a straight
unweighted figure obtained by considering all operands in the test
suite without regard to what program they belong to); even a perfect
type inferencing system will not deduce unique types for 100 percent
of all operands, because not all operands have unique types. This
suggests that an improved type inferencing system may benefit some
programs, but will have only a small overall impact. Future work
should give priority to making better use of the type information
rather than to increasing the accuracy of type inferencing.


\section[23.2 Program Execution Speed]{23.2 Program Execution Speed}

It has been demonstrated that the compiler optimizations are effective
at improving the kinds of expressions they are directed toward. The
question remains: How fast is compiled code (with and without
optimizations) for complete programs as compared to interpreted code
for the same programs? For some expressions, optimizations may
interact to create significant cumulative speed improvements. For
example, the fully optimized code for the every loop in the previous
example is 30 times faster than the interpreted code; the improvement
of 3.3 from type inferencing contributes one factor in the total
improvement. Other expressions may spend so much time in the run-time
system (which is unaffected by compiler optimizations) that no
measurable improvements are produced.

A set of 14 programs was selected mostly from contributions to the
Icon program library [.tr90-7.] for testing the performance of the
compiler. These programs were selected to represent a variety of
applications and programming styles (an additional requirement is that
they run long enough to obtain good timing results).

The following table shows the speed improvements for the compiled code
as compared to interpreted code. The compiler and interpreter used for
the measurements both implement Version 8 of Icon. The execution time
used to compute the speed improvements is the cpu time measured using
the Bourne shell's time command. The first column in the table shows
the execution time under the interpreter. The second column is for
compiled code with debugging features enabled and optimizations
disabled. This code is still better than what would be obtained by
just removing the interpreter loop, because intelligent code
generation is performed, especially for bounded expressions, and
keywords are generated in line. The third column is for code with
debugging features disabled and full optimization enabled.

\eject
\begin{center}
\tablefirsthead{}
\tablehead{}
\tabletail{}
\tablelasttail{}
\begin{xtabular}{@{}r@{\hspace{0.4in}}r@{\hspace{0.4in}}r@{\hspace{0.4in}}r@{}}
\multicolumn{4}{c}{Execution Time in Seconds Averaged over Three Runs}\\
        &             & Compiler    & Compiler\\
Program & Interpreter & Unoptimised & Optimized\\
 cksol  &
 49.9  &
 33.5 (1.48)  &
 22.5 (2.21) \\
 concord  &
 31.1  &
 18.5 (1.68)  &
 9.8 (3.17) \\
 iidecode  &
 60.3  &
 34.0 (1.77)  &
 12.9 (4.67) \\
 iiencode  &
 50.4  &
 34.4 (1.46)  &
 10.5 (4.80) \\
 impress  &
 44.6  &
 24.8 (1.79)  &
 14.0 (3.18) \\
 list  &
 43.1  &
 24.5 (1.75)  &
 13.6 (3.16) \\
 memfiltr  &
 60.8  &
 34.3 (1.77)  &
 15.3 (3.97) \\
 mf  &
 30.1  &
 18.7 (1.60)  &
 14.7 (2.04) \\
 pssplit  &
 64.0  &
 39.0 (1.64)  &
 26.6 (2.40) \\
 roffcmds  &
 32.9  &
 18.1 (1.81)  &
 12.0 (2.74) \\
 sentence  &
 34.3  &
 23.9 (1.43)  &
 16.2 (2.11) \\
 spandex  &
 36.8  &
 23.3 (1.57)  &
 14.7 (2.50) \\
 textcnt  &
 36.2  &
 18.4 (1.96)  &
 9.9 (3.65) \\
 wrapper  &
 27.3  &
 15.9 (1.71)  &
 9.4 (2.90) \\
\end{xtabular}
\end{center}


The numbers in parentheses are speed-up factors obtained by dividing
the interpreter execution time by the execution time of compiled code.


\section[23.3 Code Size]{23.3 Code Size}

One advantage the compiler has over the interpreter is that, unless a
program is compiled with full string invocation enabled, the
executable code for a program need not include the full run-time
system. For systems with limited memory, this can be a significant
advantage.

The sizes of executable code presented here are obtained from file
sizes. All executable files have had debugging information stripped
from them. The size of the executable code in the interpreter system
is taken to be the size of the interpreter (278,528 bytes) plus the
size of the icode for the program being measured (under Unix systems,
the size of the executable header, 12,800 bytes for the Sun 4, is
subtracted from the size of the icode file, because it is not present
during interpretation). Measurements for the 14 test programs are:

\begin{center}
\tablefirsthead{}
\tablehead{}
\tabletail{}
\tablelasttail{}
\begin{xtabular}{@{}r@{\hspace{0.4in}}r@{\hspace{0.4in}}r@{\hspace{0.4in}}r@{}}
\multicolumn{4}{c}{Program Sizes in Bytes}\\
Program & Interpreter & Compiler & Ratio\\
 cksol  &
 282,153  &
 81,920  &
 0.29 \\
 concord  &
 284,416  &
 90,112  &
 0.31 \\
 iidecode  &
 285,525  &
 98,304  &
 0.34 \\
 iiencode  &
 283,567  &
 81,920  &
 0.28 \\
 impress  &
 295,656  &
 114,688  &
 0.38 \\
 list  &
 287,376  &
 98,304  &
 0.34 \\
 memfiltr  &
 296,082  &
 114,688  &
 0.38 \\
 mf  &
 282,739  &
 81,920  &
 0.28 \\
 pssplit  &
 279,709  &
 73,728  &
 0.26 \\
 roffcmds  &
 280,797  &
 81,920  &
 0.29 \\
 sentence  &
 283,249  &
 81,920  &
 0.28 \\
 spandex  &
 281,843  &
 81,920  &
 0.29 \\
 textcnt  &
 280,397  &
 73,728  &
 0.26 \\
 wrapper  &
 279,780  &
 73,728  &
 0.26 \\
\end{xtabular}
\end{center}

Other factors create differences in memory usage between the
interpreter and compiled code. For example, the interpreter allocates
a stack for expression evaluation. On the Sun 4, this stack is 40,000
bytes. The compiler, on the other hand, allocates work areas on a
per-procedure basis and only allocates the maximum needed at any
execution point within the procedure.
