\chapter{Optimizing Invocations}

Several optimizations apply to the invocation of procedures and
built-in operations. These include optimizations resulting from the
application of information from type inferencing, optimizations
resulting from the application of lifetime information to passing
parameters and returning results, and optimizations involving the
generation of in-line code. There are interactions between the
optimizations in these three categories.

A primary motivation in developing the Icon compiler was to explore
the optimizations that are possible using information from type
inferencing. These optimizations involve eliminating type checking and
type conversions where type inferencing indicates that they are not
needed. Dereferencing is not normally viewed as a type conversion,
because variable references are not first-class values in
Icon. However, variable references occur as intermediate values and do
appear in the type system used by the Icon compiler. Therefore, from
the perspective of the compiler, dereferencing is a type conversion.

When a procedure or built-in operation is implemented as a C function
conforming to the standard calling conventions of the compiler system,
that function is responsible for performing any type checking and type
conversions needed by the procedure or operation. For this reason, the
checking and conversions can only be eliminated from tailored
implementations.


\section[22.1 Invocation of Procedures]{22.1 Invocation of Procedures}

As explained earlier, a procedure has one implementation: either a
standard implementation or a tailored implementation.  If the compiler
decides to produce a tailored implementation, the caller of the
procedure is responsible for dereferencing. When type inferencing
determines that an operand is not a variable reference, no
dereferencing code is generated. Suppose \texttt{p} is a procedure
that takes one argument and always fails. If \texttt{P01\_p} is the
tailored C function implementing \texttt{p}, then it takes one
argument: a pointer to a descriptor containing the dereferenced Icon
argument.  Without using type information, the call \texttt{p(3)}
translates into

\goodbreak
\begin{iconcode}
\>frame.tend.d[0].dword = D\_Integer;\\
\>frame.tend.d[0].vword.integr = 3;\\
\>deref(\&frame.tend.d[0], \&frame.tend.d[0]);\\
\>P01\_p(\&frame.tend.d[0]);\\
\end{iconcode}

\noindent
With the use of type information, the call to \texttt{deref} is eliminated: 

\goodbreak
\begin{iconcode}
\>frame.tend.d[0].dword = D\_Integer;\\
\>frame.tend.d[0].vword.integr = 3;\\
\>P01\_p(\&frame.tend.d[0]);\\
\end{iconcode}

\section[22.2 Invocation and In-lining of Built-in Operations]{22.2 Invocation and In-lining of Built-in Operations}

Icon's built-in operations present more opportunities for these
optimizations than procedures, because they can contain type checking
and conversions beyond dereferencing. Built-in operations are treated
differently than procedures. Except for keywords, there is always a C
function in the run-time library that implements the operation using
the standard calling conventions. In addition, the compiler can create
several tailored in-line versions of an operation from the information
in the data base.

It is important to keep in mind that there are two levels of
in-lining. An in-line version of an operation always involves the type
checking and conversions of the operation (although they may be
optimized away during the tailoring process). However, detailed code
is placed in-line only if it is specified with an inline statement in
the run-time system. If the detailed code is specified with a body
statement, the ``in-line'' code is a function call to a routine in the
run-time library. The difference can be seen by comparing the code
produced by compiling the expression \texttt{\textasciitilde{}x} to
that produced by compiling the expression \texttt{/x}. The definition
in the run-time implementation language of cset complement is

\goodbreak
\begin{iconcode}
\>operator\{1\} \textasciitilde{} compl(x)\\
\>\>if !cnv:tmp\_cset(x) then\\
\>\>\>runerr(104, x)\\
\>\>abstract \{ return cset \}\\
\>\>body \{\\
\>\>\textit{...}\\
\>\>\}\\
\>\ end\\
\end{iconcode}

\noindent
The conversion to \texttt{tmp\_cset} is a conversion to a cset that
does not use space in the block region. Instead the cset is
constructed in a temporary local buffer. The data base entry for the
operation indicates that the argument must be dereferenced. The entry
has a C translation of the type conversion code with a call to the
support routine, \texttt{cnv\_tcset}, to do the actual
conversion. \texttt{cnv\_tcset} takes three arguments: a buffer, a
source descriptor, and a destination descriptor. The entry in the data
base has a call to the function \texttt{O160\_compl} in place of the
body statement. This function takes as arguments the argument and the
result location of the operation. The code generator ignores the
abstract clause. The in-line code for \texttt{\textasciitilde{}x} is

\goodbreak
\begin{iconcode}
\>\>frame.tend.d[3].dword = D\_Var;\\
\>\>frame.tend.d[3].vword.descptr = \&frame.tend.d[0] /* x */;\\
\>\>deref(\&frame.tend.d[3], \&frame.tend.d[3]);\\
\>\>if (cnv\_tcset(\&(frame.cbuf[0]), \&(frame.tend.d[3]), \&(frame.tend.d[3])))\\
\>\>\>goto L1 /* then: compl */;\\
\>\>err\_msg(104, \&(frame.tend.d[3]));\\
\>L1: /* then: compl */\\
\>\>O160\_compl(\&(frame.tend.d[3]) , \&frame.tend.d[2]);\\
\end{iconcode}


The following is the definition of the ``\texttt{/}'' operator. Note
that both undereferenced and dereferenced versions of the argument are
used.

\goodbreak
\begin{iconcode}
operator\{0,1\} / null(underef x -> dx)\\
\>abstract \{\\
\>\>return type(x)\\
\>\>\}\\
\>if is:null(dx) then\\
\>\>inline \{\\
\>\>\>return x;\\
\>\>\>\}\\
\>else inline \{\\
\>\>fail;\\
\>\>\}\\
end\\
\end{iconcode}

\noindent
In this operation, all detailed code is specified with inline
statements. The generated code for \texttt{/x} follows. Note that the
order of the \texttt{then} and \texttt{else} clauses is reversed to
simplify the test. \texttt{L3} is the failure label of the expression.
The \texttt{return} is implemented as an assignment to the result
location, \texttt{frame.tend.d[2]}, with execution falling off the end
of the in-line code.

\goodbreak
\begin{iconcode}
\>\>frame.tend.d[3].dword = D\_Var;\\
\>\>frame.tend.d[3].vword.descptr = \&frame.tend.d[0] /* x */;\\
\>\>deref(\&frame.tend.d[3], \&frame.tend.d[4]);\\
\>\>if (frame.tend.d[4].dword == D\_Null)\\
\>\>\>goto L2 /* then: null */;\\
\>\>goto L3 /* bound */;\\
\>L2: /* then: null */\\
\>\>frame.tend.d[2] = frame.tend.d[3];\\
\end{iconcode}


If type inferencing determines a unique type for \texttt{x} in each of
these expressions, the type checking is eliminated from the code.
Suppose type inferencing determines that \texttt{x} can only be of type
cset in the expression

\iconline{ \>a := \textasciitilde{}x }

\noindent
If parameter passing and assignment optimizations (these are explained
below) are combined with the elimination of type checking, the
resulting code is

\iconline{ \>O160\_compl(\&(frame.tend.d[0] /* x */), \&frame.tend.d[1] /* a */); }

\noindent
The form of this translated code meets the goals of the compiler
design for the invocation of a complicated operation: a simple call to
a type-specific C function with minimum parameter passing. The
implementation language for run-time operations requires that type
conversions be specified in the control clause of an \texttt{if}
statement. However, some conversions, such as converting a string to a
cset, are guaranteed to succeed. If the code generator recognizes one
of these conversions, it eliminates the \texttt{if} statement. The
only code generated is the conversion and the code to be executed when
the conversion succeeds. Suppose type inferencing determines that
\texttt{x} in the preceding example can only be a string. Then the
generated code for the example is

\goodbreak
\begin{iconcode}
\>frame.tend.d[2] = frame.tend.d[0] /* x */;\\
\>cnv\_tcset(\&(frame.cbuf[0]), \&(frame.tend.d[2]), \&(frame.tend.d[2]));\\
\>O160\_compl(\&(frame.tend.d[2]) , \&frame.tend.d[1] / a /);\\
\end{iconcode}


\section[22.3 Heuristic for Deciding to In-line]{22.3 Heuristic for Deciding to In-line}

The in-line code for the operators shown so far in the section is
relatively small. However, the untailored in-line code for operations
like the element generation operator, \texttt{!}, is large. If tailoring the
code does not produce a large reduction in size, it is better to
generate a call to the C function in the run-time library that uses
the standard calling conventions. A heuristic is needed for deciding
when to use in-line code and when to call the standard C function.


A simple heuristic is to use in-line code only when all type checking
and conversions can be eliminated. However, this precludes the
generation of in-lining code in some important situations. The
operator \texttt{/} is used to direct control flow.  It should always be used
with an operand whose type can vary at run time, and the generated
code should always be in-lined. Consider the Icon expression

\iconline{if /x then x := ""}

\noindent
The compiler applies parameter-passing optimizations to the
sub-expression \texttt{/x}. It also eliminates the return value of the
operator, because the value is discarded. An implementation convention
for operations allows the compiler to discard the expression that
computes the return value. The convention requires that a return
expression of an operation not contain user-visible side effects
(storage allocation is an exception to the rule; it is visible, but
the language makes no guarantees as to when it will occur). The code
for \texttt{/x} is reduced to a simple type check. The code generated for the
if expression is

\goodbreak
\begin{iconcode}
\>\>if ((frame.tend.d[0] /* x */).dword == D\_Null)\\
\>\>\>goto L2 /* bound */;\\
\>\>goto L3 /* bound */;\\
\>L2: /* bound */\\
\>\>frame.tend.d[0] /* x */.vword.sptr = ;\\
\>\>frame.tend.d[0] /* x */.dword = 0;\\
\>L3: /* bound */\\
\end{iconcode}


To accommodate expressions like those in the preceding example, the
heuristic used in the compiler is to produce tailored in-line code
when that code contains no more than one type check. Only conversions
retaining their \texttt{if} statements are counted as a type checks. This
simple heuristic produces reasonable code. Future work includes
examining more sophisticated heuristics.


\section[22.4 In-lining Success Continuations]{22.4 In-lining Success Continuations}

Suspension in in-line code provides further opportunity for
optimization. In general, suspension is implemented as a call to a
success continuation. However, if there is only one call to the
continuation, it is better not to put the code in a continuation. The
code should be generated at the site of the suspension. Consider the
expression

\iconline{ \>every p(1 to 10) }

\noindent
The implementation of the \texttt{to} operator is 

\goodbreak
\begin{iconcode}
\>operator\{*\} ... to(from, to)\\
\>\>/*\\
\>\>\ * arguments must be integers.\\
\>\>\ */\\
\>\>if !cnv:C\_integer(from) then\\
\>\>\>runerr(101, from)\\
\>\>if !cnv:C\_integer(to) then\\
\>\>\>runerr(101, to)\\
\>\>abstract \{\\
\>\>\>return integer\\
\>\>\>\}\\
\>\>inline \{\\
\>\>\>for ( ; from <= to; ++from) \{\\
\>\>\>\>suspend C\_integer from;\\
\>\>\>\>\}\\
\>\>\>fail;\\
\>\>\>\}\\
\>end\\
\end{iconcode}

\noindent
The arguments are known to be integers, so the tailored version
consists of just the code in the inline statement. The \texttt{for}
statement is converted to \texttt{goto}s and conditional
\texttt{goto}s, so the control flow optimizer can handle it (this
conversion is done by \texttt{rtt} before putting the code in the data
base). The \texttt{suspend} is translated into code to set the result
value and a failure label used for the code of the rest of the bounded
expression. This code is generated before the label and consists of a
call to the procedure \texttt{p} and the failure introduced by the
\texttt{every} expression. The generated code follows. The failure for
the \texttt{every} expression is translated into \texttt{goto L4},
where \texttt{L4} is the failure label introduced by the
\texttt{suspend}. The control flow optimizer removes both the
\texttt{goto} and the label. They are retained here to elucidate the
code generation process.

\goodbreak
\begin{iconcode}
\>\>frame.tend.d[1].dword = D\_Integer;\\
\>\>frame.tend.d[1].vword.integr = 1;\\
\>\>frame.tend.d[2].dword = D\_Integer;\\
\>\>frame.tend.d[2].vword.integr = 10;\\
\\
\>L1: /* within: to */\\
\>\>if (!(frame.tend.d[1].vword.integr <= frame.tend.d[2].vword.integr) )\\
\>\>\>goto L2 /* bound */;\\
\>\>frame.tend.d[0].vword.integr = frame.tend.d[1].vword.integr;\\
\>\>frame.tend.d[0].dword = D\_Integer;\\
\>\>P01\_p(\&frame.tend.d[0]);\\
\>\>goto L4 /* end suspend: to */;\\
\>L4: /* end suspend: to */\\
\>\>++frame.tend.d[1].vword.integr;\\
\>\>goto L1 /* within: to */;\\
\>L2: /* bound */\\
\end{iconcode}


This is an example of a generator within an \texttt{every} expression
being converted into an in-line loop. Except for the fact that
descriptors are being used instead of C integers, this is nearly as
good as the C code

\goodbreak
\begin{iconcode}
\>for (i = 1; i <= 10; ++i)\\
\>\>p(i);\\
\end{iconcode}


\section[22.5 Parameter Passing Optimizations]{22.5 Parameter Passing Optimizations}

As mentioned above, parameter-passing optimizations are used to
improved the generated code. These optimizations involve eliminating
unneeded argument computations and eliminating unnecessary
copying. These optimizations are applied to tailored in-line
code. They must take into account how a parameter is used and whether
the corresponding argument value has an extended lifetime.

In some situations, a parameter is not used in the tailored
code. There are two common circumstances in which this happens. One is
for the first operand of conjunction. The other occurs with a
polymorphous operation that has a type-specific optional parameter. If
a different type is being operated on, the optional parameter is not
referenced in the tailored code. If a tailored operation has an
unreferenced parameter and the invocation has a corresponding argument
expression, the compiler notes that the expression result is
discarded. Earlier in this chapter there are examples of optimizations
possible when expression results are discarded. If the corresponding
argument is missing, the compiler refrains from supplying a null value
for it. Consider the invocation

\iconline{ \>insert(x, 3) }

\noindent \texttt{insert} takes three arguments. If \texttt{x} is a
table, the third argument is used as the entry value and must be
supplied in the generated code. In the following generated code, the
default value for the third argument is computed into
\texttt{frame.tend.d[2].dword}:

\goodbreak
\begin{iconcode}
\>frame.tend.d[1].dword = D\_Integer;\\
\>frame.tend.d[1].vword.integr = 3;\\
\>frame.tend.d[2].dword = D\_Null;\\
\>frame.tend.d[3] = frame.tend.d[0] /* x */;\\
\>F1o0\_insert(\&(frame.tend.d[2]), \&(frame.tend.d[1]), \&(frame.tend.d[3]),\\
\>\>\&trashcan);\\
\end{iconcode}

\noindent
Because \texttt{F1o0\_insert} uses a tailored calling convention, its arguments
can be in a different order from those of the Icon function. It
appears that the argument expression \texttt{x} is computed in the wrong place
in the execution order.  However, this is not true; the expression is
not computed at all. If it were, the result would be a variable
reference.  Instead, the assignment of the value in \texttt{x} to the temporary
variable is a form of optimized dereferencing. Therefore, it must be
done as part of the operation, not as part of the argument
computations. This is explained below.

If the value of \texttt{x} in this expression is a set instead of a table, the
entry value is not used. This is illustrated by the following
code. Note that a different C function is called for a set than for a
table; this is because a different body statement is selected.

\goodbreak
\begin{iconcode}
\>frame.tend.d[1].dword = D\_Integer;\\
\>frame.tend.d[1].vword.integr = 3;\\
\>frame.tend.d[2] = frame.tend.d[0] /* x */;\\
\>F1o1\_insert(\&(frame.tend.d[1]) , \&(frame.tend.d[2]) , \&trashcan);\\
\end{iconcode}

In general, an operation must copy its argument to a new descriptor
before using it. This is because an operation is allowed to modify the
argument. Modification of the original argument location is not safe
in the presence of goal-directed evaluation. The operation could be
re-executed without recomputing the argument. Therefore, the original
value must be available. This is demonstrated with the following
expression.

\iconline{ \>every p(0 to (1 to 3)) }

\noindent
This is a double loop. The outer \texttt{to} expression is the inner
loop, while the inner \texttt{to} expression is the outer loop.
\texttt{to} modifies its first argument while counting. However,
the first argument to the outer \texttt{to} has an extended lifetime
due to the fact that the second argument is a generator. Therefore,
this \texttt{to} operator must make a copy of its first argument.  The
generated code for this \texttt{every} expression is

\goodbreak
\begin{iconcode}
\>\>frame.tend.d[2].dword = D\_Integer;\\
\>\>frame.tend.d[2].vword.integr = 0;\\
\>\>frame.tend.d[4].dword = D\_Integer;\\
\>\>frame.tend.d[4].vword.integr = 1;\\
\>\>frame.tend.d[5].dword = D\_Integer;\\
\>\>frame.tend.d[5].vword.integr = 3;\\
\>L1: /* within: to */\\
\>\>if (!(frame.tend.d[4].vword.integr <= frame.tend.d[5].vword.integr))\\
\>\>\>goto L2 /* bound */;\\
\>\>frame.tend.d[3].vword.integr = frame.tend.d[4].vword.integr;\\
\>\>frame.tend.d[3].dword = D\_Integer;\\
\>\>frame.tend.d[6] = frame.tend.d[2];\\
\>L3: /* within: to */\\
\>\>if (!(frame.tend.d[6].vword.integr <= frame.tend.d[3].vword.integr))\\
\>\>\>goto L4 /* end suspend: to */;\\
\>\>frame.tend.d[1].vword.integr = frame.tend.d[6].vword.integr;\\
\>\>frame.tend.d[1].dword = D\_Integer;\\
\>\>P01\_p(\&frame.tend.d[1]);\\
\>\>++frame.tend.d[6].vword.integr;\\
\>\>goto L3 /* within: to */;\\
\>L4: /* end suspend: to */\\
\>\>++frame.tend.d[4].vword.integr;\\
\>\>goto L1 /* within: to */;\\
\>L2: /* bound */\\
\end{iconcode}

\noindent
The first argument to the outer to is copied with the statement 

\iconline{ \>frame.tend.d[6] = frame.tend.d[2]; }

\noindent
The copying of the other arguments has been eliminated because of two
observations: the second argument of \texttt{to} is never modified and
the first argument of the inner \texttt{to} (outer loop) is never
reused without being recomputed. This second fact is determined while
the lifetime information is being calculated. There is no generator
occurring between the computation of the argument and the execution of
the operator. Even if there were, it would only necessitate copying if
the generator could be resumed after the operator started executing.

As noted above, another set of optimizations involves deferencing
named variables. If an operation needs only the dereferenced value of
an argument and type inferencing determines that the argument is a
specific named variable (recall that each named variable is given a
distinct variable reference type), the code generator does not need to
generate code to compute the variable reference, because it knows what
it is. That is, it does not need the value of the argument. If the
argument is a simple identifier, no code at all is generated for the
argument.

As shown in the code presented above for 

\iconline{ \>insert(x, 3) }

\noindent
dereferencing can be implemented as simple assignment rather than a
call to the \texttt{deref} function:

\iconline{ \>frame.tend.d[3] = frame.tend.d[0] /* x */; }

\noindent
In fact, unless certain conditions interfere, the variable can be used
directly as the argument descriptor and no copying is needed. This is
reflected in the code generated in a previous example:

\iconline{ \>if /x then \textit{...} }


\texttt{x} is used directly in the in-line code for \texttt{/}: 

\goodbreak
\begin{iconcode}
\>if ((frame.tend.d[0] /* x */).dword == D\_Null)\\
\>\>goto L2 /* bound */;\\
\end{iconcode}

\noindent
This optimization cannot be performed if the operation modifies the
argument, nor can it be performed if the variable's value might change
while the operation is executing. Performing the optimization in the
presence of the second condition would violate the semantics of
argument dereferencing. The compiler does two simple tests to
determine if the second condition might be true. If the operation has
a side effect, the compiler assumes that the side-effect might involve
the named variable. Side effects are explicitly coded in the abstract
type computations of the operation. The second test is to see if the
argument has an extended lifetime. The compiler assumes that the
variable might be changed by another operation during the extended
lifetime (that is, while the operation is suspended).


\section[22.6 Assignment Optimizations]{22.6 Assignment Optimizations}

The final set of invocation optimizations involves assignments to
named variables. These includes simple assignment and augmented
assignments. Optimizing these assignments is important and
optimizations are possible beyond those that can easily be done
working from the definition in the data base; assignments to named
variables are treated as special cases. The optimizations are divided
into the cases where the right-hand-side might produce a variable
reference and those where it produces a simple Icon value.

There are two cases when the right-hand-side of the assignment
evaluates to a variable reference. If the right-hand-side is a named
variable, a dereferencing optimization can be used. Consider

\iconline{ \>s := s1 }

\noindent This Icon expression is translated into 

\iconline{ \>frame.tend.d[0] /* s */ = frame.tend.d[1] /* s1 */; }

\noindent
This is the ideal translation of this expression. For other
situations, the \texttt{deref} function must be used. For example the
expression

\iconline{ \>s := ?x }

\noindent is translated into 

\goodbreak
\begin{iconcode}
\>if (O0f2\_random(\&(frame.tend.d[0] /* x */), \&frame.tend.d[2]) == A\_Resume)\\
\>\>goto L1 /* bound */;\\
\>deref(\&frame.tend.d[2], \&frame.tend.d[1] /* s */);\\
\end{iconcode}


When the right-hand-side computes to a simple Icon value, the named
variable on the left-hand-side can often be used directly as the
result location of the operation. This occurs in the earlier example

\iconline{ \>a := \textasciitilde{}x }

\noindent which translates into

\iconline{ \>O160\_compl(\&(frame.tend.d[0] /* x */), \&frame.tend.d[1] /* a */); }

This optimization is safe as long as setting the result location is
the last thing the operation does. If the operation uses the result
location as a work area and the variable were used as the result
location, the operation might see the premature change to the
variable. In this case, a separate result location must be allocated
and the Icon assignment implemented as a C assignment. String
concatenation is an example of an operation that uses its result
location as a work area. The expression

\iconline{ \>s := s1 || s }

\noindent is translated into 

\goodbreak
\begin{iconcode}
\>if (StrLoc(frame.tend.d[1] /* s1 */) + StrLen(frame.tend.d[1] /* s1 */)\\
\>\>== strfree )\\
\>\>goto L1 /* within: cater */;\\
\>StrLoc(frame.tend.d[2]) = alcstr(StrLoc(frame.tend.d[1] /* s1 */),\\
\>\>StrLen(frame.tend.d[1] /* s1 */));\\
\>StrLen(frame.tend.d[2]) = StrLen(frame.tend.d[1] /* s1 */);\\
\>goto L2 /* within: cater */;\\
L1: /* within: cater */\\
\>frame.tend.d[2] = frame.tend.d[1] /* s1 */;\\
L2: /* within: cater */\\
\>alcstr(StrLoc(frame.tend.d[0] /* s */), StrLen(frame.tend.d[0] /* s */));\\
\>StrLen(frame.tend.d[2]) = StrLen(frame.tend.d[1] /* s1 */) +\\
\>\>StrLen(frame.tend.d[0] /* s */);\\
\\
\>frame.tend.d[0] /* s */ = frame.tend.d[2];\\
\end{iconcode}


\noindent \texttt{frame.tend.d[2]} is the result location. If
\texttt{frame.tend.d[0]} (the variable \texttt{s}) were used instead,
the code would be wrong.


There are still some optimizations falling under the category covered
by this chapter to be explored as future work. For example, as shown
earlier,

\iconline{ \>\>a := \textasciitilde{}x }

\noindent is translated into

\goodbreak
\begin{iconcode}
\>\>frame.tend.d[2] = frame.tend.d[0] /* x */;\\
\>\>cnv\_tcset(\&(frame.cbuf[0]), \&(frame.tend.d[2]), \&(frame.tend.d[2]));\\
\>\>O160\_compl(\&(frame.tend.d[2]) , \&frame.tend.d[1] /* a */);\\
\end{iconcode}

\noindent when \texttt{x} is a string. The assignment to
\texttt{frame.tend.d[2]} can be combined with the conversion to
produce the code

\goodbreak
\begin{iconcode}
\>\>cnv\_tcset(\&(frame.cbuf[0]), \&(frame.tend.d[0] /* x */), \&(frame.tend.d[2]));\\
\>\>O160\_compl(\&(frame.tend.d[2]) , \&frame.tend.d[1] /* a */);\\
\end{iconcode}

There is, of course, always room for improvement in code generation
for specific cases. However, the optimizations in this chapter combine
to produce good code for most expressions. This is reflected in the
performance data presented in Chapter 23.

