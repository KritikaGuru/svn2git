\chapter{The Unicon Translator}

Unicon was originally a modest addition to Icon. As much as possible,
features are implemented by extending existing functions with new
semantics, and by the addition of new functions and keywords with no
new syntax. Originally, the object-oriented facilities were
implemented as a simple line-oriented preprocessor named Idol.  When
Idol was merged with the Posix and ODBC facilities to become Unicon,
the Idol preprocessor was substantially modified to a full to a full
parser that generates code by traversing a syntax tree. It is still
reasonable to call the Unicon translator a preprocessor, but it has
many of the traits of a compiler.


\section[26.1 Overview]{26.1 Overview}

The Unicon translator lives in uni/unicon/. In addition to many Unicon
source files, it uses the external tools iyacc and merr to generate
its parser and syntax error message tables, which depend on files
unigram.y and meta.err, respectively. Unicon is written in Unicon,
posing a bootstrapping problem. When building from sources, some of
the .icn files can be translated by icont (the Icon translator, a C
program). Those files that require Unicon itself in order to compile
are included in precompiled object format (in .u files) in order to
solve the bootstrapping problem.

\section[26.2 Lexical Analysis]{26.2 Lexical Analysis}

Unicon's lexical analyzer is written by hand using a lex-compatible
interface required by iyacc. Some of its design is borrowed from the
Icon lexical analyzer (which is handwritten C code). It would be
interesting to replace Unicon's lexical analyzer with a machine
generated lexical analyzer to reduce the amount of compiler source
code to maintain. The lexical analyzer consists of a function
yylex() located in unicon/uni/unicon/unilex.icn, about 500 lines of
code.

{\sffamily
Globals Comprising the Lex-compatible Public API }


The global declarations that exist in order to provide a
Lex-compatible API include:

{\ttfamily\mdseries
\$include {\textquotedbl}ytab\_h.icn{\textquotedbl} \ \ \ \ \ \ \ \ \ \ \ \ \ \ \ \ \ \ \# yacc's token categories}

{\ttfamily\mdseries
global yytext \ \ \ \ \ \ \ \ \ \ \ \ \ \ \ \ \ \ \ \ \ \ \ \ \ \ \# lexeme}

{\ttfamily\mdseries
global yyin \ \ \ \ \ \ \ \ \ \ \ \ \ \ \ \ \ \ \ \ \ \ \ \ \ \ \ \ \# source file we are reading}

{\ttfamily\mdseries
global yytoken \ \ \ \ \ \ \ \ \ \ \ \ \ \ \ \ \ \ \ \ \ \ \ \ \ \# token (a record)}

{\ttfamily\mdseries
global yylineno, yycolno, yyfilename \ \ \ \# source location}

{\sffamily
Character Categories }


The lexical analyzer uses several csets for different character
categories beyond the built-in ones:

{\ttfamily\mdseries
global O, D, L, H, R, FS, IS, W, idchars}


\bigskip

{\ttfamily\mdseries
procedure init\_csets()}

{\ttfamily\mdseries
\ \ \ O \ := '01234567'}

{\ttfamily\mdseries
\ \ \ D \ := \&digits}

{\ttfamily\mdseries
\ \ \ L \ := ${\leq}$tters ++ '\_'}

{\ttfamily\mdseries
\ \ \ H \ := \&digits ++ 'abcdefABCDEF'}

{\ttfamily\mdseries
\ \ \ R \ := \&digits ++ ${\leq}$tters}

{\ttfamily\mdseries
\ \ \ FS := 'fFlL'}

{\ttfamily\mdseries
\ \ \ IS := 'uUlL'}

{\ttfamily\mdseries
\ \ \ W \ := ' {\textbackslash}t{\textbackslash}v'}

{\ttfamily\mdseries
\ \ \ idchars := L ++ D}

{\ttfamily\mdseries
end}

{\sffamily
The Token Type }


The record type storing each token's information just bundles together
the syntactic category (an integer), lexeme (a string), and location
at which the token occurred. This is pretty minimalist.

{\ttfamily\mdseries
record token(tok, s, line, column, filename)}

{\sffamily
Global Variables for Error Handling and Debugging }


Several Remaining global variables are mainly used for error handling,
and for debugging the lexical analyzer itself.

{\sffamily
Reserved Words }


Global reswords() creates and becomes a table holding the Unicon
reserved words. For each word, a pair of integers [tokenflags,
category] is kept. Language design note: tables in this language need
a {\textquotedbl}literal{\textquotedbl} format.

{\ttfamily\mdseries
procedure reswords()}

{\ttfamily\mdseries
static t}

{\ttfamily\mdseries
initial \{}

{\ttfamily\mdseries
\ \ \ t := table([Beginner+Ender, IDENT])}


\bigskip

{\ttfamily\mdseries
\ \ \ t[{\textquotedbl}abstract{\textquotedbl}] := [0, ABSTRACT]}

{\ttfamily\mdseries
\ \ \ t[{\textquotedbl}break{\textquotedbl}] := [Beginner+Ender, BREAK]}

{\ttfamily\mdseries
\ \ \ t[{\textquotedbl}by{\textquotedbl}] := [0, BY]}

{\ttfamily\mdseries
\ \ \ t[{\textquotedbl}case{\textquotedbl}] := [Beginner, CASE]}

{\ttfamily\mdseries
\ \ \ t[{\textquotedbl}class{\textquotedbl}] := [0, CLASS]}

{\ttfamily\mdseries
\ \ \ t[{\textquotedbl}create{\textquotedbl}] := [Beginner, CREATE]}

{\ttfamily\mdseries
\ \ \ t[{\textquotedbl}default{\textquotedbl}] := [Beginner, DEFAULT]}

{\ttfamily\mdseries
\ \ \ t[{\textquotedbl}do{\textquotedbl}] := [0, DO]}

{\ttfamily\mdseries
\ \ \ t[{\textquotedbl}else{\textquotedbl}] := [0, ELSE]}

{\ttfamily\mdseries
\ \ \ t[{\textquotedbl}end{\textquotedbl}] := [Beginner, END]}

{\ttfamily\mdseries
\ \ \ t[{\textquotedbl}every{\textquotedbl}] := [Beginner, EVERY]}

{\ttfamily\mdseries
\ \ \ t[{\textquotedbl}fail{\textquotedbl}] := [Beginner+Ender, FAIL]}

{\ttfamily\mdseries
\ \ \ t[{\textquotedbl}global{\textquotedbl}] := [0, GLOBAL]}

{\ttfamily\mdseries
\ \ \ t[{\textquotedbl}if{\textquotedbl}] := [Beginner, IF]}

{\ttfamily\mdseries
\ \ \ t[{\textquotedbl}import{\textquotedbl}] := [0, IMPORT]}

{\ttfamily\mdseries
\ \ \ t[{\textquotedbl}initial{\textquotedbl}] := [Beginner, iconINITIAL]}

{\ttfamily\mdseries
\ \ \ t[{\textquotedbl}initially{\textquotedbl}] := [Ender, INITIALLY]}

{\ttfamily\mdseries
\ \ \ t[{\textquotedbl}invocable{\textquotedbl}] := [0, INVOCABLE]}

{\ttfamily\mdseries
\ \ \ t[{\textquotedbl}link{\textquotedbl}] := [0, LINK]}

{\ttfamily\mdseries
\ \ \ t[{\textquotedbl}local{\textquotedbl}] := [Beginner, LOCAL]}

{\ttfamily\mdseries
\ \ \ t[{\textquotedbl}method{\textquotedbl}] := [0, METHOD]}

{\ttfamily\mdseries
\ \ \ t[{\textquotedbl}next{\textquotedbl}] := [Beginner+Ender, NEXT]}

{\ttfamily\mdseries
\ \ \ t[{\textquotedbl}not{\textquotedbl}] := [Beginner, NOT]}

{\ttfamily\mdseries
\ \ \ t[{\textquotedbl}of{\textquotedbl}] := [0, OF]}

{\ttfamily\mdseries
\ \ \ t[{\textquotedbl}package{\textquotedbl}] := [0, PACKAGE]}

{\ttfamily\mdseries
\ \ \ t[{\textquotedbl}procedure{\textquotedbl}] := [0, PROCEDURE]}

{\ttfamily\mdseries
\ \ \ t[{\textquotedbl}record{\textquotedbl}] := [0, RECORD]}

{\ttfamily\mdseries
\ \ \ t[{\textquotedbl}repeat{\textquotedbl}] := [Beginner, REPEAT]}

{\ttfamily\mdseries
\ \ \ t[{\textquotedbl}return{\textquotedbl}] := [Beginner+Ender, RETURN]}

{\ttfamily\mdseries
\ \ \ t[{\textquotedbl}static{\textquotedbl}] := [Beginner, STATIC]}

{\ttfamily\mdseries
\ \ \ t[{\textquotedbl}suspend{\textquotedbl}] := [Beginner+Ender, SUSPEND]}

{\ttfamily\mdseries
\ \ \ t[{\textquotedbl}then{\textquotedbl}] := [0, THEN]}

{\ttfamily\mdseries
\ \ \ t[{\textquotedbl}to{\textquotedbl}] := [0, TO]}

{\ttfamily\mdseries
\ \ \ t[{\textquotedbl}until{\textquotedbl}] := [Beginner, UNTIL]}

{\ttfamily\mdseries
\ \ \ t[{\textquotedbl}while{\textquotedbl}] := [Beginner, WHILE]}

{\ttfamily\mdseries
\}}

{\ttfamily\mdseries
\ \ \ return t}

{\ttfamily\mdseries
end}

{\sffamily
Lexical Analyzer Initialization and the Big Inhale }


A function, yylex\_reinit() is called the first time yylex() is
called, along with each time the compiler moves to process a new file
named on the command line. Along with initializing the public API
variables, this function reads in the entire file, in a single global
string variable, named {\textquotedbl}buffer{\textquotedbl}. This
allows extremely fast subsequent processing, which does not file I/O
for each token, while avoiding complex buffering sometimes done to
reduce file I/O costs in compilers.

This {\textquotedbl}big-inhale{\textquotedbl} model did not work well
on original 128K PDP-11 UNIX computers, but works well in this
century. At present, the code assumes Unicon source files are less
than a megabyte -{}- a lazy programmer's error. Although Unicon
programs are much shorter than C programs, an upper limit of 1MB is
bound to be reached someday.

{\ttfamily\mdseries
procedure yylex\_reinit()}

{\ttfamily\mdseries
\ \ \ yytext := {\textquotedbl}{\textquotedbl}}

{\ttfamily\mdseries
\ \ \ yylineno := 0}

{\ttfamily\mdseries
\ \ \ yycolno := 1}

{\ttfamily\mdseries
\ \ \ lastchar := {\textquotedbl}{\textquotedbl}}

{\ttfamily\mdseries
\ \ \ if type(yyin) == {\textquotedbl}file{\textquotedbl} then}

{\ttfamily\mdseries
\ \ \ \ \ \ buffer := reads(yyin, 1000000)}

{\ttfamily\mdseries
\ \ \ else}

{\ttfamily\mdseries
\ \ \ \ \ \ buffer := yyin}

{\ttfamily\mdseries
\ \ \ tokflags := 0}

{\ttfamily\mdseries
end}

{\sffamily
Semicolon Insertion }


Icon and Unicon insert semicolons for you automatically. This is an
easy lexical analyzer trick. The lexical analyzer requires one token
of lookahead. Between each two tokens, it asks: was there a newline?
If yes, was the token before the newline one that could conceivably be
the end of an expression, and was the token at the start of the new
line one that could conceivably start a new expression? If it would be
legal to do so, it saves the new token and returns a semicolon
instead.

This little procedure is entirely hidden from the regular lexical
analyzer code by writing that regular code in a helper function
yylex2(), and writing the semicolon insertion logic in a yylex()
function that calls yylex2 when it needs a new token.

Initialization for the yylex() function shows the static variables
used to implement the one token of lookahead. If the global variable
buffer doesn't hold a string anymore, /buffer will succeed and it must
be that we are at end-of-file and should return 0.

{\ttfamily\mdseries
procedure yylex()}

{\ttfamily\mdseries
\ \ static saved\_tok, saved\_yytext}

{\ttfamily\mdseries
\ \ local rv, ender}

{\ttfamily\mdseries
\ \ initial \{}

{\ttfamily\mdseries
\ \ \ \ \ \ if /buffer then}

{\ttfamily\mdseries
\ \ \ \ \ \ \ \ \ \ yylex\_reinit()}

{\ttfamily\mdseries
\ \ \ \ \ \}}

{\ttfamily\mdseries
\ \ \ if /buffer then \{}

{\ttfamily\mdseries
\ \ \ \ \ \ if {\textbackslash}debuglex then}

{\ttfamily\mdseries
\ \ \ \ \ \ \ \ \ write({\textquotedbl}yylex() : 0{\textquotedbl})}

{\ttfamily\mdseries
\ \ \ \ \ \ return 0}

{\ttfamily\mdseries
\ \ \ \ \ \ \}}


If we inserted a semicolon last time we were called, the saved\_tok will be the first token of the next line; we should
return it. 

{\ttfamily\mdseries
\ \ if {\textbackslash}saved\_tok then \{}

{\ttfamily\mdseries
\ \ \ \ rv := saved\_tok}

{\ttfamily\mdseries
\ \ \ \ saved\_tok := $\nu $ll}

{\ttfamily\mdseries
\ \ \ \ yytext := saved\_yytext}

{\ttfamily\mdseries
\ \ \ \ yylval := yytoken := token(rv, yytext, yylineno, yycolno, yyfilename)}

{\ttfamily\mdseries
\ \ \ \ if {\textbackslash}debuglex then}

{\ttfamily\mdseries
\ \ \ \ \ \ write({\textquotedbl}yylex() : {\textquotedbl},tokenstr(rv),
{\textquotedbl}{\textbackslash}t{\textquotedbl}, image(yytext))}

{\ttfamily\mdseries
\ \ \ \ return rv}

{\ttfamily\mdseries
\ \ \}}

Otherwise, we should obtain the next token by calling yylex2(). We
have to check for end of file, remember if the last token could end an
expression, call yylex2(), and update buffer to be the smaller string
remaining after the token.

{\ttfamily\mdseries
\ \ ender := iand(tokflags, Ender)}

{\ttfamily\mdseries
\ \ tokflags := 0}

{\ttfamily\mdseries
\ \ if *buffer=0 then \{}

{\ttfamily\mdseries
\ \ \ \ \ \ buffer := $\nu $ll}

{\ttfamily\mdseries
\ \ \ \ \ \ if {\textbackslash}debuglex then}

{\ttfamily\mdseries
\ \ \ \ \ \ \ \ \ \ write({\textquotedbl}yylex() : EOFX{\textquotedbl})}

{\ttfamily\mdseries
\ \ \ \ \ \ return EOFX}

{\ttfamily\mdseries
\ \ \ \ \ \}}

{\ttfamily\mdseries
\ \ buffer ? \{}

{\ttfamily\mdseries
\ \ \ \ \ \ if rv := yylex2() then \{}

{\ttfamily\mdseries
\ \ \ \ \ \ \ \ \ \ buffer := tab(0)}

{\ttfamily\mdseries
\ \ \ \ \ \ \}}

{\ttfamily\mdseries
\ \ \ \ \ \ else \{}

{\ttfamily\mdseries
\ \ \ \ \ \ \ \ \ buffer := $\nu $ll}

{\ttfamily\mdseries
\ \ \ \ \ \ \ \ \ yytext := {\textquotedbl}{\textquotedbl}}

{\ttfamily\mdseries
\ \ \ \ \ \ \ \ \ if {\textbackslash}debuglex then}

{\ttfamily\mdseries
\ \ \ \ \ \ \ \ \ \ \ \ \ write({\textquotedbl}yylex() : EOFX{\textquotedbl})}

{\ttfamily\mdseries
\ \ \ \ \ \ \ \ \ return EOFX}

{\ttfamily\mdseries
\ \ \ \ \ \ \}}

{\ttfamily\mdseries
\ \ \}}

After fetching a new token, we have to decide whether to insert a
semicolon or not. This is based on global variable ender (whether the
previous token could end an expression) and global variable tokflags
(which holds both whether the current token could begin an expression,
and whether a newline occurred between the last token and the current
token).  iand() is a bitwise AND, equivalen to C language \& operator,
used to pick bits out of a set of boolean flags encoded as bits within
an integer.

{\ttfamily\mdseries
\ \ if ender\~{}=0 \& iand(tokflags, Beginner)\~{}=0 \& iand(tokflags, Newline)\~{}=0 then \{}

{\ttfamily\mdseries
\ \ \ \ saved\_tok := rv}

{\ttfamily\mdseries
\ \ \ \ saved\_yytext := yytext}

{\ttfamily\mdseries
\ \ \ \ yytext := {\textquotedbl};{\textquotedbl}}

{\ttfamily\mdseries
\ \ \ \ rv := SEMICOL}

{\ttfamily\mdseries
\ \ \ \ \}}

Returning a token requires allocation of a token() record instance,
which is stored in a global variable.

{\ttfamily\mdseries
\ \ \ yylval := yytoken := token(rv, yytext, yylineno, yycolno, yyfilename)}

{\ttfamily\mdseries
\ \ \ if {\textbackslash}debuglex then}

{\ttfamily\mdseries
\ \ \ \ \ \ write({\textquotedbl}yylex() : {\textquotedbl}, tokenstr(rv),
{\textquotedbl}{\textbackslash}t{\textquotedbl}, image(yytext))}

{\ttfamily\mdseries
\ \ \ return rv}

{\ttfamily\mdseries
end}

{\sffamily
The Real Lexical Analyzer Function, yylex2()}

This function maintains a table of functions, calling a helper
function depending on what the first character in the token is.

{\ttfamily\mdseries
procedure yylex2()}

{\ttfamily\mdseries
static punc\_table}

{\ttfamily\mdseries
initial \{}

{\ttfamily\mdseries
\ \ \ init\_csets()}

{\ttfamily\mdseries
\ \ \ reswords := reswords()}

{\ttfamily\mdseries
\ \ \ punc\_table := table(uni\_error)}

{\ttfamily\mdseries
\ \ \ punc\_table[{\textquotedbl}'{\textquotedbl}] := do\_literal}

{\ttfamily\mdseries
\ \ \ punc\_table[{\textquotedbl}{\textbackslash}{\textquotedbl}{\textquotedbl}] := do\_literal}

{\ttfamily\mdseries
\ \ \ punc\_table[{\textquotedbl}!{\textquotedbl}] := do\_bang}

{\ttfamily\mdseries
\ \ \ punc\_table[{\textquotedbl}\%{\textquotedbl}] := do\_mod}

{\ttfamily\mdseries
\ \ \ punc\_table[{\textquotedbl}\&{\textquotedbl}] := do\_and}

{\ttfamily\mdseries
\ \ \ punc\_table[{\textquotedbl}*{\textquotedbl}] := do\_star}

{\ttfamily\mdseries
\ \ \ punc\_table[{\textquotedbl}+{\textquotedbl}] := do\_plus}

{\ttfamily\mdseries
\ \ \ punc\_table[{\textquotedbl}-{\textquotedbl}] := do\_minus}

{\ttfamily\mdseries
\ \ \ punc\_table[{\textquotedbl}.{\textquotedbl}] := do\_dot}

{\ttfamily\mdseries
\ \ \ punc\_table[{\textquotedbl}/{\textquotedbl}] := do\_slash}

{\ttfamily\mdseries
\ \ \ punc\_table[{\textquotedbl}:{\textquotedbl}] := do\_colon}

{\ttfamily\mdseries
\ \ \ punc\_table[{\textquotedbl}{\textless}{\textquotedbl}] := do\_less}

{\ttfamily\mdseries
\ \ \ punc\_table[{\textquotedbl}={\textquotedbl}] := do\_equal}

{\ttfamily\mdseries
\ \ \ punc\_table[{\textquotedbl}{\textgreater}{\textquotedbl}] := do\_greater}

{\ttfamily\mdseries
\ \ \ punc\_table[{\textquotedbl}?{\textquotedbl}] := do\_qmark}

{\ttfamily\mdseries
\ \ \ punc\_table[{\textquotedbl}@{\textquotedbl}] := do\_at}

{\ttfamily\mdseries
\ \ \ punc\_table[{\textquotedbl}{\textbackslash}{\textbackslash}{\textquotedbl}] := do\_backslash}

{\ttfamily\mdseries
\ \ \ punc\_table[{\textquotedbl}\^{}{\textquotedbl}] := do\_caret}

{\ttfamily\mdseries
\ \ \ punc\_table[{\textquotedbl}{\textbar}{\textquotedbl}] := do\_or}

{\ttfamily\mdseries
\ \ \ punc\_table[{\textquotedbl}\~{}{\textquotedbl}] := do\_tilde}

{\ttfamily\mdseries
\ \ \ punc\_table[{\textquotedbl}({\textquotedbl}] := do\_lparen}

{\ttfamily\mdseries
\ \ \ punc\_table[{\textquotedbl}){\textquotedbl}] := do\_rparen}

{\ttfamily\mdseries
\ \ \ punc\_table[{\textquotedbl}[{\textquotedbl}] := do\_lbrack}

{\ttfamily\mdseries
\ \ \ punc\_table[{\textquotedbl}]{\textquotedbl}] := do\_rbrack}

{\ttfamily\mdseries
\ \ \ punc\_table[{\textquotedbl}\{{\textquotedbl}] := do\_lbrace}

{\ttfamily\mdseries
\ \ \ punc\_table[{\textquotedbl}\}{\textquotedbl}] := do\_rbrace}

{\ttfamily\mdseries
\ \ \ punc\_table[{\textquotedbl},{\textquotedbl}] := do\_comma}

{\ttfamily\mdseries
\ \ \ punc\_table[{\textquotedbl};{\textquotedbl}] := do\_semi}

{\ttfamily\mdseries
\ \ \ punc\_table[{\textquotedbl}\${\textquotedbl}] := do\_dollar}

{\ttfamily\mdseries
\ \ \ every punc\_table[!\&digits] := do\_digits}

{\ttfamily\mdseries
\ \ \ every punc\_table[{\textquotedbl}\_{\textquotedbl} {\textbar} !${\leq}$tters] := do\_letters}

{\ttfamily\mdseries
\ \ \ \}}


The main lexical analyzer code strips comments and whitespace, and calls the function table for the first non-whitespace
character it finds. Note support for \#line directives, and the use of string scanning. 

{\ttfamily\mdseries
\ \ \ yycolno +:= *yytext}


\bigskip

{\ttfamily\mdseries
\ \ \ repeat \{}

{\ttfamily\mdseries
\ \ \ \ \ \ \ if pos(0) then fail}

{\ttfamily\mdseries
\ \ \ \ \ \ \ if }

{\ttfamily\mdseries
\ \ \ \ \ \ \ \ \ \ \ ={\textquotedbl}\#{\textquotedbl} then \{}

{\ttfamily\mdseries
\ \ \ \ \ \ \ \ \ \ \ \ \ \ \ if ={\textquotedbl}line {\textquotedbl} then \{}

{\ttfamily\mdseries
\ \ \ \ \ \ \ \ \ \ \ \ \ \ \ \ \ \ \ if yylineno := integer(tab(many(\&digits))) then \{}

{\ttfamily\mdseries
\ \ \ \ \ \ \ \ \ \ \ \ \ \ \ \ \ \ \ \ \ \ \ ={\textquotedbl} {\textbackslash}{\textquotedbl}{\textquotedbl}}

{\ttfamily\mdseries
\ \ \ \ \ \ \ \ \ \ \ \ \ \ \ \ \ \ \ \ \ \ \ yyfilename :=
tab(find({\textquotedbl}{\textbackslash}{\textquotedbl}{\textquotedbl}){\textbar}0)}

{\ttfamily\mdseries
\ \ \ \ \ \ \ \ \ \ \ \ \ \ \ \ \ \ \ \}}

{\ttfamily\mdseries
\ \ \ \ \ \ \ \ \ \ \ \ \ \ \ \}}

{\ttfamily\mdseries
\ \ \ \ \ \ \ \ \ \ \ \ \ \ \ tab(find({\textquotedbl}{\textbackslash}n{\textquotedbl}) {\textbar} 0)}

{\ttfamily\mdseries
\ \ \ \ \ \ \ \ \ \ \ \ \ \ \ next}

{\ttfamily\mdseries
\ \ \ \ \ \ \ \ \ \ \ \}}

{\ttfamily\mdseries
\ \ \ \ \ \ \ if ={\textquotedbl}{\textbackslash}n{\textquotedbl} then \{}

{\ttfamily\mdseries
\ \ \ \ \ \ \ \ \ \ \ yylineno +:= 1}

{\ttfamily\mdseries
\ \ \ \ \ \ \ \ \ \ \ yycolno := 1}

{\ttfamily\mdseries
\ \ \ \ \ \ \ \ \ \ \ if tokflags {\textless} Newline then}

{\ttfamily\mdseries
\ \ \ \ \ \ \ \ \ \ \ \ \ \ \ tokflags +:= Newline}

{\ttfamily\mdseries
\ \ \ \ \ \ \ \ \ \ \ next}

{\ttfamily\mdseries
\ \ \ \ \ \ \ \}}

{\ttfamily\mdseries
\ \ \ \ \ \ \ if tab(any(' ')) then \{ yycolno +:= 1; next \}}

{\ttfamily\mdseries
\ \ \ \ \ \ \ if tab(any('{\textbackslash}v{\textbackslash}\^{}l')) then \{ next \}}

{\ttfamily\mdseries
\ \ \ \ \ \ \ if tab(any('{\textbackslash}t')) then \{}

{\ttfamily\mdseries
\ \ \ \ \ \ \ \ \ \ \ yycolno +:= 1}

{\ttfamily\mdseries
\ \ \ \ \ \ \ \ \ \ \ while (yycolno-1) \% 8 \~{}= 0 do yycolno +:= 1}

{\ttfamily\mdseries
\ \ \ \ \ \ \ \ \ \ \ next}

{\ttfamily\mdseries
\ \ \ \ \ \ \ \}}


\bigskip

{\ttfamily\mdseries
\ \ \ \ \ \ \ yytext := move(1)}

{\ttfamily\mdseries
\ \ \ \ \ \ \ return punc\_table[yytext]()}

{\ttfamily\mdseries
\ \ \ \}}

{\ttfamily\mdseries
end}


The functions in the punctuation table select integer codes and match
the rest of the lexeme. do\_comma() illustrates an unambiguous token
selection, while do\_plus() illustrates a more common case where the
{\textquotedbl}+{\textquotedbl} character could start any of 5
different tokens depending on the character(s) that follow it. Tokens
starting with {\textquotedbl}letters{\textquotedbl} are looked up in a
reserved words table, which tells whether they are special, or just a
variable name.

{\ttfamily\mdseries
procedure do\_comma()}

{\ttfamily\mdseries
\ \ \ return COMMA}

{\ttfamily\mdseries
end}


\bigskip

{\ttfamily\mdseries
procedure do\_plus()}

{\ttfamily\mdseries
\ \ \ if yytext {\textbar}{\textbar}:= ={\textquotedbl}:{\textquotedbl} then \{}

{\ttfamily\mdseries
\ \ \ \ \ \ if yytext {\textbar}{\textbar}:= ={\textquotedbl}={\textquotedbl} then \{ return AUGPLUS \}}

{\ttfamily\mdseries
\ \ \ \ \ \ \ \ \ return PCOLON}

{\ttfamily\mdseries
\ \ \ \ \ \ \}}

{\ttfamily\mdseries
\ \ \ if yytext {\textbar}{\textbar}:= ={\textquotedbl}+{\textquotedbl} then \{}

{\ttfamily\mdseries
\ \ \ \ \ \ if yytext {\textbar}{\textbar}:= ={\textquotedbl}:={\textquotedbl} then \{return AUGUNION\}}

{\ttfamily\mdseries
\ \ \ \ \ \ \ \ \ return UNION}

{\ttfamily\mdseries
\ \ \ \ \ \ \}}

{\ttfamily\mdseries
\ \ \ tokflags +:= Beginner}

{\ttfamily\mdseries
\ \ \ return PLUS}

{\ttfamily\mdseries
end}


\bigskip

{\ttfamily\mdseries
procedure do\_letters()}

{\ttfamily\mdseries
\ \ \ yytext {\textbar}{\textbar}:= tab(many(idchars))}

{\ttfamily\mdseries
\ \ \ x := reswords[yytext]}

{\ttfamily\mdseries
\ \ \ tokflags +:= x[1]}

{\ttfamily\mdseries
\ \ \ return x[2]}

{\ttfamily\mdseries
end}

\section[26.3 The Unicon Parser]{26.3 The Unicon Parser}

Unicon's parser is written using a YACC grammar; a graduate student
(Ray Pereda) modified Berkeley's public domain version of YACC (byacc)
to generate Unicon code, following in the footsteps of someone who had
earlier modified it to generate Java. The Unicon parser lives in
uni/unicon/unigram.y in the source distribution (22kB, 700 lines, 119
terminals, 71 nonterminals). Unicon's YACC grammar was obtained by
copying the Icon grammar, and adding Unicon syntax constructs. Prior
to this time the object-oriented dialect of Icon was called Idol and
really was a line-oriented preprocessor instead of a compiler.

The start symbol for the grammar is named
\textstyleSourceText{program}, and the semantic action code fragment
for this nonterminal calls the rest of the compiler (semantic analysis
and code generation) directly on the root of the syntax tree, rather
than storing it in a global variable for the main() procedure to
examine.

{\ttfamily\mdseries
program : decls EOFX \{ Progend(\$1);\} ;}


Many context free grammar rules are recursive, with an empty
production to terminate the recursion. The rule for declarations is
typical:

{\ttfamily\mdseries
decls \ \ : \{ \$\$ := EmptyNode \}}

{\ttfamily\mdseries
\ \ \ \ \ \ \ \ {\textbar} decls decl \{}

{\ttfamily\mdseries
\ \ \ \ \ \ \ \ \ \ \ if yynerrs = 0 then iwrites(\&errout,{\textquotedbl}.{\textquotedbl})}

{\ttfamily\mdseries
\ \ \ \ \ \ \ \ \ \ \ \$\$ := node({\textquotedbl}decls{\textquotedbl}, \$1, \$2)}

{\ttfamily\mdseries
\ \ \ \ \ \ \ \ \ \ \ \} ;}


The {\textquotedbl}semantic action{\textquotedbl} (code fragment) for
every production rule builds a syntax tree node and assigns it to \$\$
for the nonterminal left-hand side of the rule.

Another common grammar pattern is a production rule that has many
different alternatives, such as the one for individual declarations:

{\ttfamily\mdseries
decl \ \ \ : record}

{\ttfamily\mdseries
\ \ \ \ \ \ \ \ {\textbar} proc}

{\ttfamily\mdseries
\ \ \ \ \ \ \ \ {\textbar} global}

{\ttfamily\mdseries
\ \ \ \ \ \ \ \ {\textbar} link}

{\ttfamily\mdseries
\ \ \ \ \ \ \ \ {\textbar} package}

{\ttfamily\mdseries
\ \ \ \ \ \ \ \ {\textbar} import}

{\ttfamily\mdseries
\ \ \ \ \ \ \ \ {\textbar} invocable}

{\ttfamily\mdseries
\ \ \ \ \ \ \ \ {\textbar} cl}

{\ttfamily\mdseries
\ \ \ \ \ \ \ \ ;}


For such {\textquotedbl}unary{\textquotedbl} productions, child's
syntax tree node suffices for the parent, no new tree node is needed.

Some nonterminals mostly correspond to a specific sequence of
terminals, as is the case for package references:

{\ttfamily\mdseries
packageref : IDENT COLONCOLON IDENT \{ \$\$ := node({\textquotedbl}packageref{\textquotedbl}, \$1,\$2,\$3) \} }

{\ttfamily\mdseries
\ \ \ {\textbar} COLONCOLON IDENT \{ \$\$ := node({\textquotedbl}packageref{\textquotedbl}, \$1,\$2) \} \ }

{\ttfamily\mdseries
\ \ \ ;}


The lexical analyzer has already constructed a valid
{\textquotedbl}leaf{\textquotedbl} for each terminal symbol, so if a
production rule has only one terminal symbol in it, for a syntax tree
we can simply use the leaf for that nonterminal (for a parse tree, we
would need to allocate an extra unary internal node):

{\ttfamily\mdseries
lnkfile : IDENT ;}

{\ttfamily\mdseries
\ \ \ \ \ \ \ \ {\textbar} STRINGLIT ;}


The expressions (which comprise about half of the grammar) use a
separate nonterminal for each level of precedence intead of YACC's
declarations for handling precedence (\%left, \%right, etc). The Icon
and Unicon grammars approach 20 levels of nonterminals. A typical rule
looks like:

{\ttfamily\mdseries
expr6 \ \ : expr7 ;}

{\ttfamily\mdseries
\ \ \ \ \ \ \ \ {\textbar} expr6 PLUS expr7 \{ \$\$ := node({\textquotedbl}Bplus{\textquotedbl}, \$1,\$2,\$3);\} ;}

{\ttfamily\mdseries
\ \ \ \ \ \ \ \ {\textbar} expr6 DIFF expr7 \{ \$\$ := node({\textquotedbl}Bdiff{\textquotedbl}, \$1,\$2,\$3);\} ;}

{\ttfamily\mdseries
\ \ \ \ \ \ \ \ {\textbar} expr6 UNION expr7 \{ \$\$ := node({\textquotedbl}Bunion{\textquotedbl}, \$1,\$2,\$3);\} ;}

{\ttfamily\mdseries
\ \ \ \ \ \ \ \ {\textbar} expr6 MINUS expr7 \{ \$\$ := node({\textquotedbl}Bminus{\textquotedbl}, \$1,\$2,\$3);\} ;}


The {\textquotedbl}B{\textquotedbl} stands for
{\textquotedbl}binary{\textquotedbl}, to distinguish these operators
from their unary brethren. The 20 levels of nonterminals approach is
inherited from Icon and probably makes the parser larger than it has
to be, but taking these nonterminals out doesn't seem to help much.

\subsection[26.3.1 Syntax Error Handling]{26.3.1 Syntax Error Handling}

Icon employed a relatively clever approach to doing syntax error
messages with YACC --- the parse state at the time of error is enough
to do fairly good diagnoses. But, every time the grammar changed, the
parse state numbers could change wildly. For Unicon I developed the
Merr tool, which associates parse error example fragments with the
corresponding diagnostic error message, and detects/infers the parse
state for you, reducing the maintenance problem when changing the
grammar. Merr also considers the current input token in deciding what
error message to emit, making it fundamentally more precise than
Icon's approach.

\section[26.4 The Unicon Preprocessor]{26.4 The Unicon Preprocessor}

The Icon language originally did not include any preprocessor, but
eventually, a simple one was introduced, with ability to include
headers, define symbolic constants (macros without parameters), and
handle conditional compilation (ifdef).  The preprocessor
implementation in Unicon was written by Bob Alexander, and came to
Unicon by way of Jcon, an Icon-to-JVM translator. This preprocessor is
written in a single 600+ line file, uni/unicon/preproce.icn.

The external public interface of the preprocessor is line-oriented,
consisting of a generator preproc(filename, predefinedsyms) which
suspends each line of the output, one after another. Its invocation
from the main() procedure looks like:

{\ttfamily\mdseries
\ \ \ yyin := {\textquotedbl}{\textquotedbl}}

{\ttfamily\mdseries
\ \ \ every yyin {\textbar}{\textbar}:= preprocessor(fName, uni\_predefs) do}

{\ttfamily\mdseries
\ \ \ \ \ \ yyin {\textbar}{\textbar}:= {\textquotedbl}{\textbackslash}n{\textquotedbl}}


Since the preprocessor outputs line-by-line, there is a mismatch
between it and the lexical analyzer's big-inhale model.  The
preprocessor could be modified to fit better with the lexical analyzer
or vice versa.

The preprocessor function takes the filename to read from, along with
a table of predefined symbols which allows the preprocessor to respond
to lines like

{\ttfamily\mdseries
\$ifdef \_SQL}

\noindent based on what libraries are available and how Unicon was
built on a given platform.

The preprocessor() function itself starts each call off with initializations: 

{\ttfamily\mdseries
\ \ \ \ static nonpunctuation}

{\ttfamily\mdseries
\ \ \ \ initial \{}

{\ttfamily\mdseries
\ \ \ \ \ \ \ \ nonpunctuation := ${\leq}$\&letters ++ \&digits ++ '
{\textbackslash}t{\textbackslash}f{\textbackslash}r'}

{\ttfamily\mdseries
\ \ \ \ \}}


\bigskip

{\ttfamily\mdseries
\ \ \ \ preproc\_new(fname,predefined\_syms)}

The initialization code opens fname, creates empty stacks to keep
track of nested \$ifdef's and \$include's, initializes counters to 0
and so forth.

The preprocessor is line-oriented. For each line, it looks for a
preprocessor directive, and if it does not find one, it just scans for
symbols to replace and returns the line. The main loop looks like

{\ttfamily\mdseries
\ \ \ while line := preproc\_read() do line ? \{}

{\ttfamily\mdseries
\ \ \ \ \ \ preproc\_space() \ \ \ \ \ \ \# eat whitespace}

{\ttfamily\mdseries
\ \ \ \ \ \ if (={\textquotedbl}\#{\textquotedbl} \& match({\textquotedbl}line{\textquotedbl})) {\textbar}
(={\textquotedbl}\${\textquotedbl} \& any(nonpunctuation)) then \{}

{\ttfamily\mdseries
\ \ \ \ \ \ \ \ \ suspend preproc\_scan\_directive()}

{\ttfamily\mdseries
\ \ \ \ \ \ \ \ \ \}}

{\ttfamily\mdseries
\ \ \ \ \ \ else \{}

{\ttfamily\mdseries
\ \ \ \ \ \ \ \ \ \&pos := 1}

{\ttfamily\mdseries
\ \ \ \ \ \ \ \ \ suspend preproc\_scan\_text()}

{\ttfamily\mdseries
\ \ \ \ \ \ \ \ \ \}}

{\ttfamily\mdseries
\ \ \ \ \ \ \}}


The procedures preproc\_scan\_directive() and preproc\_scan\_text()
work on special and ordinary lines, respectively.  The line is not a
parameter because it is held in the current string scanning
environment. The preproc\_scan\_directive() starts by discardign
whitespace and identifying the first word on the line (which must be a
valid preprocessor directive). A case expression handles the various
directives (define, undef, ifdef, etc.). Defined symbols are stored in
a table. \$ifdef and \$ifndef are handled using a global variable
preproc\_if\_state to track the boolean conditions. A count of
\$ifdef's is maintained, in order to handle matching endif's.

Include files are handled using a stack, but an additional set of
filenames is kept to prevent infinite recursion when files include
each other. When a new include directive is encountered it is checked
against the preproc\_include\_set and if OK, it is opened. The
including file (and its associated name, line, etc) are pushed onto a
list named preproc\_file\_stack. It is possible to run out of open
files under this model, although this is not easy under modern
operating systems.

Include files are searched on an include file path, consisting of a
list of directories given on an optional environment variable (LPATH)
followed by a list of standard directories. The standard directories
are expected to be found relative to the location of the virtual
machine binaries.

The procedure preproc\_scan\_text has the relatively simple job of
replacing any symbols by their definitions within an ordinary source
line. Since macros do not have parameters, it is vastly simpler than
in a C preprocessor. The main challenges are to avoid macro
substitutions when a symbol is in a comment or within quotes (string
or cset literals).  An additional issue is to handle multiline string
literals, which occur in Icon when a string literal is not closed on a
line, and instead the line ends with an underscore indicating that it
is continued on the next line. Skipping over quoted text sounds
simple, but is trickier than it looks. Escape characters mean you
can't just look for the closing quote without considering what comes
before it, and you can't just look at the preceding character since it
might have been escaped, as in
{\textquotedbl}{\textbackslash}{\textbackslash}{\textquotedbl}. The
code looks similar to:

{\ttfamily\mdseries
repeat \{}

{\ttfamily\mdseries
\ \ \ while tab(upto('{\textquotedbl}{\textbackslash}{\textbackslash}')) do \{}

{\ttfamily\mdseries
\ \ \ \ \ \ case move(1) of \{}

{\ttfamily\mdseries
\ \ \ \ \ \ \ \ \ {\textquotedbl}{\textbackslash}{\textbackslash}{\textquotedbl}: move(1)}

{\ttfamily\mdseries
\ \ \ \ \ \ \ \ \ default: \{}

{\ttfamily\mdseries
\ \ \ \ \ \ \ \ \ \ \ \ break break}

{\ttfamily\mdseries
\ \ \ \ \ \ \ \ \ \ \ \ \}}

{\ttfamily\mdseries
\ \ \ \ \ \ \ \ \ \}}

{\ttfamily\mdseries
\ \ \ \ \ \ \}}

{\ttfamily\mdseries
\ \ \ \# ...}

{\ttfamily\mdseries
\ \ \ if not match({\textquotedbl}\_{\textquotedbl},,-1) then}

{\ttfamily\mdseries
\ \ \ \ \ \ break}

{\ttfamily\mdseries
\ \ \ ${\subset}$\&subject := preproc\_read() {\textbar} fail}

{\ttfamily\mdseries
\ \ \ \# ...}

{\ttfamily\mdseries
\ \ \ \}}

The code in preproc\_read() for reading a line does a regular Icon
read(); end of file causes the preprocessor file\_stack to be popped
for the previous file's information. Performance has not been
perceived as a significant problem, it it would be interesting to
convert preproc\_read() to use a big-inhale model to see if any
statistical difference could be observed. When an include is
encountered under a big-inhale, the saved state would contain the
string of remaining file contents, instead of the open file value.


\section[26.5 Semantic Analysis]{26.5 Semantic Analysis}

The Unicon translator's semantic analysis is minimal, and revolves
mainly around object-oriented features such as inheritance and package
imports. Before we can look at those things, we need to look at the
syntax tree structure.

In conventional YACC, a \%union declaration is necessary to handle the
varying types of objects on the value stack including the type used
for syntax tree nodes, but iyacc has no need of this awkward
mechanism: the value stack like all structure types can hold any type
of value in each slot. Similarly, tree nodes can hold children of any
type, potentially eliminating any awkwardness of mixing tokens and
internal nodes. Of course, you do still have to check what kind of
value you are working with.

{\sffamily
Parse Tree Nodes }

uni/unicon/tree.icn contains procedures to handle the syntax tree node
data type, including both the following declaration and the yyprint()
traversal function we'll be discussing in today's lecture.

{\ttfamily\mdseries
record treenode(label, children)}

\noindent holds one node worth of information. For convenience, a
procedure node(label, kids[]) takes an arbitrary number of parameters
and constructs the list of children for you. Leaves have a null
children field.

{\sffamily
{\textquotedbl}Code Generation{\textquotedbl} in the Unicon Translator }

In a regular preprocessor, there is no code generation, there is a
text-filter model in which the preprocessor writes out (modified)
versions of the lines it reads in. In the Unicon translator, the code
that is written out is produced by a traversal of the syntax tree. The
same technique might be used by a {\textquotedbl}pretty
printer{\textquotedbl}. We will explore this aspect of the Unicon
translator as the best available demonstration of working with Unicon
syntax trees. Later on we will consider more
{\textquotedbl}real{\textquotedbl} code generation in the virtual
machine and the optimizing compiler.

Earlier we saw that the start symbol of the Unicon grammar had a
semantic action that called a procedure Progend(). We will cover most
of that procedure next week since it is all about object-orientation,
but at the end Progend(), a call to yyprint() performs the tree
traversal for code generation. A classic tree traversal pattern would
look like:

{\ttfamily\mdseries
procedure traverse(node)}

{\ttfamily\mdseries
\ \ \ if node is an internal node \{}

{\ttfamily\mdseries
\ \ \ \ \ \ every child := ! node.children do traverse(child)}

{\ttfamily\mdseries
\ \ \ \ \ \ generate code for this internal node (postfix)}

{\ttfamily\mdseries
\ \ \ \ \ \ \}}

{\ttfamily\mdseries
\ \ \ else}

{\ttfamily\mdseries
\ \ \ \ \ \ generate code for this leaf}

{\ttfamily\mdseries
end}

The code generator traversal yyprint() is a lot more complicated than
that, but fits the general pattern. The main work done at various
nodes is to write some text to the output file, yyout. Most ordinary
internal nodes are of type treenode as described above. But because
there are several kinds of internal nodes and several kinds of leaves,
the {\textquotedbl}if node is an internal node{\textquotedbl} is
implemented as a case expression. Besides a regular treenode, the
other kinds of internal nodes are objects of type declaration, class,
and argument list. For regular treenodes, another case expression on
the node's label field is used to determine what kind of code to
generate, if any, besides visiting children and generating their code.

The default behavior for an internal node is to just visit the
children, generating their code. For ordinary syntax constructs (if,
while, etc.) this works great and a copy of the code is written out,
token by token. But several exceptions occur, mainly for the pieces of
Unicon syntax that extend Icon's repertoire. For example, packages and
imports are not in Icon and require special treatment.

{\ttfamily\mdseries
procedure yyprint(node)}

{\ttfamily\mdseries
\ \ \ static lasttok}

{\ttfamily\mdseries
\ \ \ case type(node) of \{}

{\ttfamily\mdseries
\ \ \ \ \ \ {\textquotedbl}treenode{\textquotedbl} : \{}

{\ttfamily\mdseries
\ \ \ \ \ \ \ \ \ case node.label of \{}

{\ttfamily\mdseries
\ \ \ \ \ \ \ \ \ {\textquotedbl}package{\textquotedbl}: \{ \} \# handled by semantic analysis}

{\ttfamily\mdseries
\ \ \ \ \ \ \ \ \ {\textquotedbl}import{\textquotedbl}: \{ print\_imports(node.children[2]) \}}

{\ttfamily\mdseries
\ \ \ \ \ \ \ \ \ \# implement packages via name mangling}

{\ttfamily\mdseries
\ \ \ \ \ \ \ \ \ {\textquotedbl}packageref{\textquotedbl}: \{}

{\ttfamily\mdseries
\ \ \ \ \ \ \ \ \ \ \ \ \ if *node.children = 2 then}

{\ttfamily\mdseries
\ \ \ \ \ \ \ \ \ \ \ \ \ \ \ \ \ yyprint(node.children[2]) \# ::ident}

{\ttfamily\mdseries
\ \ \ \ \ \ \ \ \ \ \ \ \ else \{ \# ident :: ident}

{\ttfamily\mdseries
\ \ \ \ \ \ \ \ \ \ \ \ \ \ \ \ yyprint(node.children[1])}

{\ttfamily\mdseries
\ \ \ \ \ \ \ \ \ \ \ \ \ \ \ \ writes(yyout, {\textquotedbl}\_\_{\textquotedbl})}

{\ttfamily\mdseries
\ \ \ \ \ \ \ \ \ \ \ \ \ \ \ \ outcol +:= ((* writes(yyout, node.children[3].s)) + 2)}

{\ttfamily\mdseries
\ \ \ \ \ \ \ \ \ \ \ \ \ \ \ \ \}}

{\ttfamily\mdseries
\ \ \ \ \ \ \ \ \ \ \ \ \}}


New syntax constructs such as procedure parameter defaults and type
restrictions, and variable initializers, are other examples where the
default traversal would output things illegal in Icon. They are
implemented by skipping some of the children (assignment and value) in
the regular pass, and adding extra code elsewhere, discussed below.

{\ttfamily\mdseries
\ \ \ \ \ \ \ \ \ {\textquotedbl}varlist2{\textquotedbl}{\textbar}{\textquotedbl}stalist2{\textquotedbl}: \{
yyprint(node.children[1]) \}}

{\ttfamily\mdseries
\ \ \ \ \ \ \ \ \ {\textquotedbl}varlist4{\textquotedbl}{\textbar}{\textquotedbl}stalist4{\textquotedbl}: \{}

{\ttfamily\mdseries
\ \ \ \ \ \ \ \ \ \ \ \ yyprint(node.children[1])}

{\ttfamily\mdseries
\ \ \ \ \ \ \ \ \ \ \ \ yyprint(node.children[2])}

{\ttfamily\mdseries
\ \ \ \ \ \ \ \ \ \ \ \ yyprint(node.children[3])}

{\ttfamily\mdseries
\ \ \ \ \ \ \ \ \ \ \ \ \}}


Much of this special logic is orchestrated by the code for traversing
a procedure; it can visit its arguments and variable declarations and
apply special rules to them.

{\ttfamily\mdseries
\ \ \ \ \ \ \ \ \ {\textquotedbl}proc{\textquotedbl}: \{}

{\ttfamily\mdseries
\ \ \ \ \ \ \ \ \ \ \ \ yyprint(node.children[1])}

{\ttfamily\mdseries
\ \ \ \ \ \ \ \ \ \ \ \ every yyprint(node.children[2 to 3])}

{\ttfamily\mdseries
\ \ \ \ \ \ \ \ \ \ \ \ if exists\_statlists(node.children[3]) then \{}

{\ttfamily\mdseries
\ \ \ \ \ \ \ \ \ \ \ \ \ \ \ ini := node.children[4]}

{\ttfamily\mdseries
\ \ \ \ \ \ \ \ \ \ \ \ \ \ \ yyprint({\textquotedbl}{\textbackslash}ninitial \{{\textquotedbl})}

{\ttfamily\mdseries
\ \ \ \ \ \ \ \ \ \ \ \ \ \ \ if ini \~{}=== EmptyNode then \{ \# append into existing initial}

{\ttfamily\mdseries
\ \ \ \ \ \ \ \ \ \ \ \ \ \ \ \ \ \ yyprint(ini.children[2])}

{\ttfamily\mdseries
\ \ \ \ \ \ \ \ \ \ \ \ \ \ \ \ \ \ yyprint({\textquotedbl};{\textbackslash}n{\textquotedbl})}

{\ttfamily\mdseries
\ \ \ \ \ \ \ \ \ \ \ \ \ \ \ \ \ \ \}}

{\ttfamily\mdseries
\ \ \ \ \ \ \ \ \ \ \ \ \ \ \ yystalists(node.children[3])}

{\ttfamily\mdseries
\ \ \ \ \ \ \ \ \ \ \ \ \ \ \ yyprint({\textquotedbl}{\textbackslash}n\}{\textbackslash}n{\textquotedbl})}

{\ttfamily\mdseries
\ \ \ \ \ \ \ \ \ \ \ \ \ \ \ \}}

{\ttfamily\mdseries
\ \ \ \ \ \ \ \ \ \ \ \ else}

{\ttfamily\mdseries
\ \ \ \ \ \ \ \ \ \ \ \ \ \ \ every yyprint(node.children[4])}

{\ttfamily\mdseries
\ \ \ \ \ \ \ \ \ \ \ \ (node.children[1].fields).coercions()}

{\ttfamily\mdseries
\ \ \ \ \ \ \ \ \ \ \ \ yyvarlists(node.children[3])}

{\ttfamily\mdseries
\ \ \ \ \ \ \ \ \ \ \ \ yyprint(node.children[5])}

{\ttfamily\mdseries
\ \ \ \ \ \ \ \ \ \ \ \ yyprint(node.children[6])}

{\ttfamily\mdseries
\ \ \ \ \ \ \ \ \ \ \ \ \}}

The default behavior of visiting one's children is very simple, as is
the handling of other kinds of internal nodes, which are objects. For
the objects, a method Write() is invoked.

{\ttfamily\mdseries
\ \ \ \ \ \ \ \ \ {\textquotedbl}error{\textquotedbl}: fail}

{\ttfamily\mdseries
\ \ \ \ \ \ \ \ \ default:}

{\ttfamily\mdseries
\ \ \ \ \ \ \ \ \ \ \ \ every yyprint(!node.children)}

{\ttfamily\mdseries
\ \ \ \ \ \ \ \ \ \}}

{\ttfamily\mdseries
\ \ \ \ \ \ {\textquotedbl}declaration\_\_state{\textquotedbl} {\textbar} {\textquotedbl}Class\_\_state{\textquotedbl}
{\textbar} {\textquotedbl}argList\_\_state{\textquotedbl}:}

{\ttfamily\mdseries
\ \ \ \ \ \ \ \ \ node.Write(yyout)}


The outer case expression of yyprint() continues with various kinds of
leaf (token) nodes. These mainly know how to write their lexemes
out. But, a lot of effort is made to try to keep line and column
number information consistent.  Variables outline and outcol are
maintained as each token is written out. Integers and string literals
found in the syntax tree are written out as themselves. Since they
have no attached lexical attributes, they are a bit suspect in terms
of maintaining debugging consistency. It turns out the reason they
occur at all, and the reason they have no source lexical attributes,
is that artificial syntax subtrees are generated to handle certain
object-oriented constructs, and within those subtrees strings and
integers may be placed, which do not correspond to anywhere in the
source code.

{\ttfamily\mdseries
\ \ \ \ \ \ {\textquotedbl}integer{\textquotedbl}: \{}

{\ttfamily\mdseries
\ \ \ \ \ \ \ \ \ writes(yyout, node); outcol +:= *string(node)}

{\ttfamily\mdseries
\ \ \ \ \ \ \ \ \ \}}

{\ttfamily\mdseries
\ \ \ \ \ \ {\textquotedbl}string{\textquotedbl}: \{}

{\ttfamily\mdseries
\ \ \ \ \ \ \ \ \ node ? \{}

{\ttfamily\mdseries
\ \ \ \ \ \ \ \ \ \ \ \ while writes(yyout, tab(find({\textquotedbl}{\textbackslash}n{\textquotedbl})+1)) do \{}

{\ttfamily\mdseries
\ \ \ \ \ \ \ \ \ \ \ \ \ \ \ outline+:=1; outcol:=1;}

{\ttfamily\mdseries
\ \ \ \ \ \ \ \ \ \ \ \ \ \ \ \}}

{\ttfamily\mdseries
\ \ \ \ \ \ \ \ \ \ \ \ node := tab(0)}

{\ttfamily\mdseries
\ \ \ \ \ \ \ \ \ \ \ \ \}}

{\ttfamily\mdseries
\ \ \ \ \ \ \ \ \ writes(yyout, node); outcol +:= *node}

{\ttfamily\mdseries
\ \ \ \ \ \ \ \ \ \}}


{\textquotedbl}Normally{\textquotedbl}, tokens are written out at
exactly the line and column they appear at in the source code. But a
myriad of constructs may bump them around. If the output falls behind
(in lines, or columns) extra whitespace can be inserted to stay in
sync. If output gets ahead by lines, a \#line directive can back it
up, but if output gets ahead by columns, there is nothing much one can
do, except make sure subsequent tokens don't accidentally get
attached/concatenated onto earlier tokens. This occurs, for example,
when the output code for an object-oriented construct in an expression
is longer than the source expression, perhaps due to name
mangling. Specific token combinations are checked, but the list here
may be incomplete (possible BUG!). For source tokens, not only might
the line and column change, the filename could be different as well.

{\ttfamily\mdseries
\ \ \ \ \ \ {\textquotedbl}token{\textquotedbl}: \{}

{\ttfamily\mdseries
\ \ \ \ \ \ \ \ \ if outfilename \~{}== node.filename {\textbar} outline {\textgreater} node.line then \{}

{\ttfamily\mdseries
\ \ \ \ \ \ \ \ \ \ \ \ write(yyout,{\textquotedbl}{\textbackslash}n\#line {\textquotedbl}, node.line-1,{\textquotedbl}
{\textbackslash}{\textquotedbl}{\textquotedbl},
node.filename,{\textquotedbl}{\textbackslash}{\textquotedbl}{\textquotedbl})}

{\ttfamily\mdseries
\ \ \ \ \ \ \ \ \ \ \ \ outline := node.line}

{\ttfamily\mdseries
\ \ \ \ \ \ \ \ \ \ \ \ outcol := 1}

{\ttfamily\mdseries
\ \ \ \ \ \ \ \ \ \ \ \ outfilename := node.filename}

{\ttfamily\mdseries
\ \ \ \ \ \ \ \ \ \ \ \ \}}

{\ttfamily\mdseries
\ \ \ \ \ \ \ \ \ while outline {\textless} node.line do \{}

{\ttfamily\mdseries
\ \ \ \ \ \ \ \ \ \ \ \ write(yyout); outline +:= 1; outcol := 1}

{\ttfamily\mdseries
\ \ \ \ \ \ \ \ \ \ \ \ \}}

{\ttfamily\mdseries
\ \ \ \ \ \ \ \ \ if outcol {\textgreater}= node.column then \{}

{\ttfamily\mdseries
\ \ \ \ \ \ \ \ \ \ \ \ \# force space between idents and reserved words, and other}

{\ttfamily\mdseries
\ \ \ \ \ \ \ \ \ \ \ \ \# deadly combinations (need to add some more)}

{\ttfamily\mdseries
\ \ \ \ \ \ \ \ \ \ \ \ if (({\textbackslash}lasttok).tok = (IDENT{\textbar}INTLIT{\textbar}REALLIT) \&
reswords[node.s][2]\~{}=IDENT){\textbar}}

{\ttfamily\mdseries
\ \ \ \ \ \ \ \ \ \ \ \ \ \ \ \ ((({\textbackslash}lasttok).tok = NMLT) \& (node.tok = MINUS)) {\textbar}}

{\ttfamily\mdseries
\ \ \ \ \ \ \ \ \ \ \ \ \ \ \ \ (({\textbackslash}lasttok).tok = node.tok = PLUS) {\textbar}}

{\ttfamily\mdseries
\ \ \ \ \ \ \ \ \ \ \ \ \ \ \ \ (({\textbackslash}lasttok).tok = node.tok = MINUS) {\textbar}}

{\ttfamily\mdseries
\ \ \ \ \ \ \ \ \ \ \ \ \ \ \ \ ((reswords[({\textbackslash}lasttok).s][2]\~{}=IDENT) \&
(node.tok=(IDENT{\textbar}INTLIT{\textbar}REALLIT))){\textbar}}

{\ttfamily\mdseries
\ \ \ \ \ \ \ \ \ \ \ \ \ \ \ \ ((reswords[({\textbackslash}lasttok).s][2]\~{}=IDENT) \&
(reswords[node.s][2]\~{}=IDENT))}

{\ttfamily\mdseries
\ \ \ \ \ \ \ \ \ \ \ \ \ \ \ \ \ \ \ then}

{\ttfamily\mdseries
\ \ \ \ \ \ \ \ \ \ \ \ \ \ \ writes(yyout, {\textquotedbl} {\textquotedbl})}

{\ttfamily\mdseries
\ \ \ \ \ \ \ \ \ \ \ \ \}}

{\ttfamily\mdseries
\ \ \ \ \ \ \ \ \ else}

{\ttfamily\mdseries
\ \ \ \ \ \ \ \ \ \ \ \ while outcol {\textless} node.column do \{ writes(yyout, {\textquotedbl} {\textquotedbl});
outcol +:= 1 \}}


Most tokens' lexemes are finally written out by writing node.s: 

{\ttfamily\mdseries
\ \ \ \ \ \ \ \ \ writes(yyout, node.s)}

{\ttfamily\mdseries
\ \ \ \ \ \ \ \ \ outcol +:= *node.s}

{\ttfamily\mdseries
\ \ \ \ \ \ \ \ \ lasttok := node}

{\ttfamily\mdseries
\ \ \ \ \ \ \ \ \ \}}

{\ttfamily\mdseries
\ \ \ \ \ \ {\textquotedbl}null{\textquotedbl}: \{ \}}

{\ttfamily\mdseries
\ \ \ \ \ \ default: write({\textquotedbl}its a {\textquotedbl}, type(node))}

{\ttfamily\mdseries
\ \ \ \ \ \ \}}

{\ttfamily\mdseries
end}

{\sffamily
Keywords }


Besides the large set of interesting reserved words, Icon and Unicon
have another set of predefined special words called
\textstyleEmphasis{keywords}. These words are prefixed by an
ampersand, for example, \&subject holds the current
{\textquotedbl}subject{\textquotedbl} string being examined by string
scanning. A procedure Keyword(x1,x2) semantically checks that an
identifier following a unary ampersand is one of the valid keyword
names. The valid names are kept in a set data structure.


\section[26.6 Object Oriented Facilities]{26.6 Object Oriented Facilities}

Unicon features classes, packages, and a novel multiple inheritance
mechanism. These items are implemented entirely within the Unicon
translator. The Icon virtual machine thusfar has only the slightest of
extensions for object-orientation, specifically, the dot operator has
been extended to handle objects and method invocation.

The Unicon OOP facilities were originally prototyped as a semester
class project in a {\textquotedbl}special topics{\textquotedbl}
graduate course. Writing the prototype in a very high-level language
like Icon, and developing it as a preprocessor with name mangling,
allowed the initial class mechanism to be developed in a single
evening, and a fairly full, usable system with working inheritance to
be developed in the first weekend. By the end of the semester, the
system was robust enough to write it in itself, and it was released to
the public shortly afterwards as a package for Icon called
{\textquotedbl}Idol{\textquotedbl}. Many many improvements were made
after this point, often at the suggestion of users.

An initial design goal was to make the absolute smallest additions to
the language that were necessary to support
object-orientation. Classes were viewed as a version of Icon's record
data type, retaining its syntax for fields (member variables), but
appending a set of associated procedures. Because records have no
concept of public and private, neither did classes. Another graduate
student criticized this lack of privacy, and for several versions,
everything was made private unless an explicit public keyword was
used. But eventually support for privacy was dropped on the grounds
that it added no positive capabilities and was un-Iconish. The
existence of classes with hundreds of
{\textquotedbl}getter{\textquotedbl} and
{\textquotedbl}setter{\textquotedbl} methods was considered a direct
proof that {\textquotedbl}private{\textquotedbl} was idiotic in a
rapid prototyping language.

{\sffamily
The Code Generation Model for Classes }


{\textquotedbl}unicon -E foo{\textquotedbl} will show you what code is
generated for Unicon file foo.icn. If foo.icn contains classes, you
can enjoy the code generation model and experiment to see what it does
under various circumstances. As a first example, consider

{\ttfamily\mdseries
class A(x,y)}

{\ttfamily\mdseries
\ \ \ method m()}

{\ttfamily\mdseries
\ \ \ \ \ \ write({\textquotedbl}hello{\textquotedbl})}

{\ttfamily\mdseries
\ \ \ end}

{\ttfamily\mdseries
end}


These five lines generate 25 lines for Icont to translate into virtual
machine code. The first two lines are line directives showing from
whence this source code originated:

{\ttfamily\mdseries
\#line 0 {\textquotedbl}/tmp/uni13804206{\textquotedbl}}

{\ttfamily\mdseries
\#line 0 {\textquotedbl}a.icn{\textquotedbl}}


Global declarations (including procedures) would be passed through the
preprocessor pretty nearly intact, but for the class, we get a bunch
of very different code. Methods are written out, with names mangled to
a classname\_methodname format.

{\ttfamily\mdseries
procedure A\_m(self)}


\bigskip


\bigskip

{\ttfamily\mdseries
\#line 2 {\textquotedbl}a.icn{\textquotedbl}}

{\ttfamily\mdseries
\ \ \ \ \ write({\textquotedbl}hello{\textquotedbl});}

{\ttfamily\mdseries
end}


Two record types are defined, one for the class instances and one for
the {\textquotedbl}methods vector{\textquotedbl}, or
{\textquotedbl}operation record{\textquotedbl}. The methods vector is
instantiated exactly once in a global variable in classname\_\_oprec
format.

{\ttfamily\mdseries
record A\_\_state(\_\_s,\_\_m,x,y)}

{\ttfamily\mdseries
record A\_\_methods(m)}

{\ttfamily\mdseries
global A\_\_oprec}


The default constructor for a class takes fields as parameters and
uses them directly for initialization purposes. The first time it is
called, a methods vector is created. Instances are given a pointer to
themselves in an \_\_s field (mainly for historical reasons) and to
the methods vector in an \_\_m field. Current NMSU grad student Sumant
Tambe did an independent study project to get rid of \_\_s and \_\_m
with partial success, but his work is not finished or robust enough to
be enabled by default.

{\ttfamily\mdseries
procedure A(x,y)}

{\ttfamily\mdseries
local self,clone}

{\ttfamily\mdseries
initial \{}

{\ttfamily\mdseries
\ \ if /A\_\_oprec then Ainitialize()}

{\ttfamily\mdseries
\ \ \}}

{\ttfamily\mdseries
\ \ self := A\_\_state($\nu $ll,A\_\_oprec,x,y)}

{\ttfamily\mdseries
\ \ self.\_\_s := self}

{\ttfamily\mdseries
\ \ return self}

{\ttfamily\mdseries
end}


\bigskip

{\ttfamily\mdseries
procedure Ainitialize()}

{\ttfamily\mdseries
\ \ initial A\_\_oprec := A\_\_methods(A\_m)}

{\ttfamily\mdseries
end}

{\sffamily
Symbols and Scope Resolution }


One of the basic aspects of semantic analysis is: for each variable,
where was it declared, so we can identify its address, etc. Unicon
inherits from Icon the curious convenience that variables do not have
to be declared: they are local by default. This feature is implemented
by deferring the local vs. global decision until link time, so the
Unicon translator has no local vs. global issues. Class variables,
however, have to be identified, and looked up relative to the implicit
{\textquotedbl}self{\textquotedbl} variable. A family of procedures in
uni/unicon/tree.icn with names starting
{\textquotedbl}scopecheck{\textquotedbl} go through the syntax tree
looking for such class variables. Like most tree traversals, this is a
recursive process, and since local and parameter declarations override
class variables, there are helper functions to walk through subtrees
building mini-symbol tables such as local\_vars in
scopecheck\_proc(node):

{\ttfamily\mdseries
\ \ \ \# Build local\_vars from the params and local var expressions.}

{\ttfamily\mdseries
\ \ \ local\_vars := set()}

{\ttfamily\mdseries
\ \ \ extract\_identifiers(node.children[1].fields, local\_vars)}

{\ttfamily\mdseries
\ \ \ extract\_identifiers(node.children[3], local\_vars)}

Eventually, every identifier in every expression is checked against
local\_vars, and if not found there, against the class variables
stored in a variable self\_vars:

{\ttfamily\mdseries
\ \ \ self\_vars := set()}

{\ttfamily\mdseries
\ \ \ every insert(self\_vars, c.foreachmethod().name)}

{\ttfamily\mdseries
\ \ \ every insert(self\_vars, c.foreachfield())}

{\ttfamily\mdseries
\ \ \ every insert(self\_vars, (!c.ifields).ident)}

{\ttfamily\mdseries
\ \ \ every insert(self\_vars, (!c.imethods).ident)}

For an IDENT node, the tests boil down to: 

{\ttfamily\mdseries
\ \ \ if node.tok = IDENT then \{}

{\ttfamily\mdseries
\ \ \ \ \ \ if not member({\textbackslash}local\_vars, node.s) then \{}

{\ttfamily\mdseries
\ \ \ \ \ \ \ \ \ if member({\textbackslash}self\_vars, node.s) then}

{\ttfamily\mdseries
\ \ \ \ \ \ \ \ \ \ \ \ node.s := {\textquotedbl}self.{\textquotedbl} {\textbar}{\textbar} node.s}

{\ttfamily\mdseries
\ \ \ \ \ \ \ \ \ else }

{\ttfamily\mdseries
\ \ \ \ \ \ \ \ \ \ \ \ node.s := mangle\_sym(node.s)}

{\ttfamily\mdseries
\ \ \ \ \ \ \}}

{\ttfamily\mdseries
\ \ \ \}}


Undeclared locals and globals are mangled to include the current
package name if there is one.

{\sffamily
Inheritance }


Inheritance means: creating a class that is similar to an existing
class. In object-oriented literature there is {\textquotedbl}abstract
inheritance{\textquotedbl} in which a class supports all the same
operations with the same signatures, and there is concrete inheritance
in which actual code is shared. Early object-oriented languages
supported only concrete inheritance, while more recent languages tend
to discourage it. Unicon is not typed at compile time, so abstract
inheritance is not a big deal. There are abstract methods, and classes
whose every method is abstract, but the use of abstract is mainly for
documentation: subclass authors must provide certain methods. Anyhow,
the syntax of inheritance in Unicon is

{\ttfamily\mdseries
class subclass : super1 : super2 : ... ( ...fields... )}


The semantics of inheritance, and particularly of multiple
inheritance, are interesting in Unicon; the implementation is
relatively simple. An example of inheritance is given by class Class,
from uni/unicon/idol.icn

{\ttfamily\mdseries
class declaration(name,fields,tag,lptoken,rptoken)}

{\ttfamily\mdseries
\ \ \ ...}

{\ttfamily\mdseries
end}

{\ttfamily\mdseries
...}

{\ttfamily\mdseries
class Class : declaration (supers,}

{\ttfamily\mdseries
\ \ \ \ \ \ \ \ \ \ \ \ \ \ \ \ \ \ \ \ \ \ \ \ \ \ \ methods,}

{\ttfamily\mdseries
\ \ \ \ \ \ \ \ \ \ \ \ \ \ \ \ \ \ \ \ \ \ \ \ \ \ \ text,}

{\ttfamily\mdseries
\ \ \ \ \ \ \ \ \ \ \ \ \ \ \ \ \ \ \ \ \ \ \ \ \ \ \ imethods,}

{\ttfamily\mdseries
\ \ \ \ \ \ \ \ \ \ \ \ \ \ \ \ \ \ \ \ \ \ \ \ \ \ \ ifields,}

{\ttfamily\mdseries
\ \ \ \ \ \ \ \ \ \ \ \ \ \ \ \ \ \ \ \ \ \ \ \ \ \ \ glob,}

{\ttfamily\mdseries
\ \ \ \ \ \ \ \ \ \ \ \ \ \ \ \ \ \ \ \ \ \ \ \ \ \ \ linkfile,}

{\ttfamily\mdseries
\ \ \ \ \ \ \ \ \ \ \ \ \ \ \ \ \ \ \ \ \ \ \ \ \ \ \ dir,}

{\ttfamily\mdseries
\ \ \ \ \ \ \ \ \ \ \ \ \ \ \ \ \ \ \ \ \ \ \ \ \ \ \ unmangled\_name,}

{\ttfamily\mdseries
\ \ \ \ \ \ \ \ \ \ \ \ \ \ \ \ \ \ \ \ \ \ \ \ \ \ \ supers\_node)}


Unique perspective on inheritance in Unicon comes from the actual
acquisition of inherited data fields and methods by the subclass. Some
object-oriented languages do this inheritance {\textquotedbl}by
aggregation{\textquotedbl}, creating a copy of the superclass in the
subclass. This is fine, but it makes
{\textquotedbl}overriding{\textquotedbl} an anomaly, when overriding
the parent with new/different behavior is entirely routine. Unicon
instead inherits by the child looking for things in the parent (and
the parent's parent, etc.) that they don't already have. In the above
example, class declaration effectively appends 5 fields from class
declaration onto the end of its field list. The generated code for
instances looks like

{\ttfamily\mdseries
record Class\_\_state(\_\_s,\_\_m,}

{\ttfamily\mdseries
\ \ \ \ \ \ \ \ \ \ \ \ \ \ \ \ \ \ \ \ supers,methods,text,imethods,ifields,}

{\ttfamily\mdseries
\ \ \ \ \ \ \ \ \ \ \ \ \ \ \ \ \ \ \ \ glob,linkfile,dir,unmangled\_name,supers\_node,}

{\ttfamily\mdseries
\ \ \ \ \ \ \ \ \ \ \ \ \ \ \ \ \ \ \ \ name,fields,tag,lptoken,rptoken)}

The inheritance semantics is called {\textquotedbl}closure
based{\textquotedbl} because the process of looking for things to add
from parent superclasses iterates until no new information can be
added, after which the subclass is said to be closed on its
parents. Other forms of closure appear frequently in CS.

\subsection[26.6.1 Implementing Multiple Inheritance in Unicon]{26.6.1 Implementing Multiple Inheritance in Unicon}

The actual code in the Unicon translator is, by analogy to transitive
closure, looking for things to inherit via a depthfirst traversal of
the inheritance graph. Multiple inheritance can be separated out into
two portions:

\liststyleLxliii
\begin{enumerate}

\item a method transitive\_closure() that finds all superclasses and
provides a linearization of them, flattening the graph into a single
ordered list of all superclasses

\item a method resolve() that walks the list and looks for classes and
fields to add.

\end{enumerate}

Method transitive\_closure() is one of the cleaner demonstrations of
why Unicon is a fun language in which to write complex algorithms. It
is walking through a class graph, but by the way it is not recursive.

{\ttfamily\mdseries
\ \ method transitive\_closure()}

{\ttfamily\mdseries
\ \ \ \ count := supers.size()}

{\ttfamily\mdseries
\ \ \ \ while count {\textgreater} 0 do \{}

{\ttfamily\mdseries
\ \ \ \ \ \ \ \ added := taque()}

{\ttfamily\mdseries
\ \ \ \ \ \ \ \ every sc := supers.foreach() do \{}

{\ttfamily\mdseries
\ \ \ \ \ \ \ \ \ \ if /(super := classes.lookup(sc)) then}

{\ttfamily\mdseries
\ \ \ \ \ \ \ \ \ \ \ \ halt({\textquotedbl}class/transitive\_closure: \_}

{\ttfamily\mdseries
\ \ \ \ \ \ \ \ \ \ \ \ \ \ \ \ \ \ couldn't find superclass {\textquotedbl},sc)}

{\ttfamily\mdseries
\ \ \ \ \ \ \ \ \ \ every supersuper := super.foreachsuper() do \{}

{\ttfamily\mdseries
\ \ \ \ \ \ \ \ \ \ \ \ if / self.supers.lookup(supersuper) \&}

{\ttfamily\mdseries
\ \ \ \ \ \ \ \ \ \ \ \ \ \ \ \ \ /added.lookup(supersuper) then \{}

{\ttfamily\mdseries
\ \ \ \ \ \ \ \ \ \ \ \ \ \ added.insert(supersuper)}

{\ttfamily\mdseries
\ \ \ \ \ \ \ \ \ \ \ \ \}}

{\ttfamily\mdseries
\ \ \ \ \ \ \ \ \ \ \}}

{\ttfamily\mdseries
\ \ \ \ \ \ \ \ \}}

{\ttfamily\mdseries
\ \ \ \ \ \ \ \ count := added.size()}

{\ttfamily\mdseries
\ \ \ \ \ \ \ \ every self.supers.insert(added.foreach())}

{\ttfamily\mdseries
\ \ \ \ \}}

{\ttfamily\mdseries
\ \ end}


Now, given that Unicon provides a depthfirst inheritance hierarchy
semantics, what is wrong with this picture? The code is stable and
hasn't needed changes in several years, so this question is not
fishing for syntax bugs, or claiming that there is a bug. But there is
something odd.

The method resolve() within class Class finds the inherited fields and
methods from the linearized list of superclasses.


{\ttfamily\mdseries
\ \ \#}

{\ttfamily\mdseries
\ \ \# resolve -{}- primary inheritance resolution utility}

{\ttfamily\mdseries
\ \ \#}

{\ttfamily\mdseries
\ \ method resolve()}

{\ttfamily\mdseries
\ \ \ \ \#}

{\ttfamily\mdseries
\ \ \ \ \# these are lists of [class , ident] records}

{\ttfamily\mdseries
\ \ \ \ \#}

{\ttfamily\mdseries
\ \ \ \ self.imethods := []}

{\ttfamily\mdseries
\ \ \ \ self.ifields := []}

{\ttfamily\mdseries
\ \ \ \ ipublics := []}

{\ttfamily\mdseries
\ \ \ \ addedfields := table()}

{\ttfamily\mdseries
\ \ \ \ addedmethods := table()}

{\ttfamily\mdseries
\ \ \ \ every sc := supers.foreach() do \{}

{\ttfamily\mdseries
\ \ \ \ \ \ \ \ if /(superclass := classes.lookup(sc)) then}

{\ttfamily\mdseries
\ \ \ \ \ \ \ \ \ \ \ \ halt({\textquotedbl}class/resolve: couldn't find superclass {\textquotedbl},sc)}

{\ttfamily\mdseries
\ \ \ \ \ \ \ \ every superclassfield := superclass.foreachfield() do \{}

{\ttfamily\mdseries
\ \ \ \ \ \ \ \ \ \ \ \ if /self.fields.lookup(superclassfield) \&}

{\ttfamily\mdseries
\ \ \ \ \ \ \ \ \ \ \ \ \ \ \ /addedfields[superclassfield] then \{}

{\ttfamily\mdseries
\ \ \ \ \ \ \ \ \ \ \ \ \ \ \ \ addedfields[superclassfield] := superclassfield}

{\ttfamily\mdseries
\ \ \ \ \ \ \ \ \ \ \ \ \ \ \ \ put ( self.ifields , classident(sc,superclassfield) )}

{\ttfamily\mdseries
\ \ \ \ \ \ \ \ \ \ \ \ \ \ \ \ if superclass.ispublic(superclassfield) then}

{\ttfamily\mdseries
\ \ \ \ \ \ \ \ \ \ \ \ \ \ \ \ \ \ \ \ put( ipublics, classident(sc,superclassfield) )}

{\ttfamily\mdseries
\ \ \ \ \ \ \ \ \ \ \ \ \} else if {\textbackslash}strict then \{}

{\ttfamily\mdseries
\ \ \ \ \ \ \ \ \ \ \ \ \ \ \ \ warn({\textquotedbl}class/resolve: '{\textquotedbl},sc,{\textquotedbl}' field
'{\textquotedbl},superclassfield,}

{\ttfamily\mdseries
\ \ \ \ \ \ \ \ \ \ \ \ \ \ \ \ \ \ \ \ \ {\textquotedbl}' is redeclared in subclass {\textquotedbl},self.name)}

{\ttfamily\mdseries
\ \ \ \ \ \ \ \ \ \ \ \ \}}

{\ttfamily\mdseries
\ \ \ \ \ \ \ \ \}}

{\ttfamily\mdseries
\ \ \ \ \ \ \ \ every superclassmethod := (superclass.foreachmethod()).name() do \{}

{\ttfamily\mdseries
\ \ \ \ \ \ \ \ \ \ \ \ if /self.methods.lookup(superclassmethod) \&}

{\ttfamily\mdseries
\ \ \ \ \ \ \ \ \ \ \ \ \ \ \ /addedmethods[superclassmethod] then \{}

{\ttfamily\mdseries
\ \ \ \ \ \ \ \ \ \ \ \ \ \ \ \ addedmethods[superclassmethod] := superclassmethod}

{\ttfamily\mdseries
\ \ \ \ \ \ \ \ \ \ \ \ \ \ \ \ put ( self.imethods, classident(sc,superclassmethod) )}

{\ttfamily\mdseries
\ \ \ \ \ \ \ \ \ \ \ \ \}}

{\ttfamily\mdseries
\ \ \ \ \ \ \ \ \}}

{\ttfamily\mdseries
\ \ \ \ \ \ \ \ every public := (!ipublics) do \{}

{\ttfamily\mdseries
\ \ \ \ \ \ \ \ \ \ \ \ if public.Class == sc then}

{\ttfamily\mdseries
\ \ \ \ \ \ \ \ \ \ \ \ \ \ \ \ put (self.imethods, classident(sc,public.ident))}

{\ttfamily\mdseries
\ \ \ \ \ \ \ \ \}}

{\ttfamily\mdseries
\ \ \ \ \}}

{\ttfamily\mdseries
\ \ end}

{\sffamily
Class and Package Specifications }


In the {\textquotedbl}old days{\textquotedbl} of Unicon's ancestor
Idol, you could only inherit from a class that appeared in the same
source file. Anything else poses a librarian's problem of identifying
from what file to inherit.  Java, for instances, takes a brute-force
approach of one class per file.

Unicon generates in each source directory an NDBM database (named
uniclass.dir and uniclass.pag) that includes a mapping from class name
to: what file the class lives in, plus, what superclasses, fields, and
methods appear in that class.  From these specifications,
{\textquotedbl}link{\textquotedbl} declarations are generated for
superclasses within subclass modules, plus the subclass can perform
inheritance resolution. The code to find a class specification is
given in idol.icn's fetchspec(). A key fragment looks like

{\ttfamily\mdseries
\ \ \ if f := open(dir {\textbar}{\textbar} {\textquotedbl}/{\textquotedbl} {\textbar}{\textbar} env,
{\textquotedbl}dr{\textquotedbl}) then \{}

{\ttfamily\mdseries
\ \ \ \ \ \ if s := fetch(f, name) then \{}

{\ttfamily\mdseries
\ \ \ \ \ \ \ \ \ close(f)}

{\ttfamily\mdseries
\ \ \ \ \ \ \ \ \ return db\_entry(dir, s)}

{\ttfamily\mdseries
\ \ \ \ \ \ \ \ \ \}}

{\ttfamily\mdseries
\ \ \ \ \ \ close(f)}

{\ttfamily\mdseries
\ \ \ \ \ \ \}}


Unicon searches for {\textquotedbl}link{\textquotedbl} declarations in
a particular order, given by the current directory followed by
directories in an IPATH (Icode path, or perhaps Icon path) environment
variable, followed by system library directories such as ipl/lib and
uni/lib. This same list of directories is searched for inherited
classes.


The string stored in uniclass.dir and returned from fetch() for class Class is: 

{\ttfamily\mdseries
idol.icn}

{\ttfamily\mdseries
class Class : declaration(supers,methods,text,imethods,ifields,glob,linkfile,dir,unmangled\_name,supers\_node)}

{\ttfamily\mdseries
ismethod}

{\ttfamily\mdseries
isfield}

{\ttfamily\mdseries
Read}

{\ttfamily\mdseries
ReadBody}

{\ttfamily\mdseries
has\_initially}

{\ttfamily\mdseries
ispublic}

{\ttfamily\mdseries
foreachmethod}

{\ttfamily\mdseries
foreachsuper}

{\ttfamily\mdseries
foreachfield}

{\ttfamily\mdseries
isvarg}

{\ttfamily\mdseries
transitive\_closure}

{\ttfamily\mdseries
writedecl}

{\ttfamily\mdseries
WriteSpec}

{\ttfamily\mdseries
writemethods}

{\ttfamily\mdseries
Write}

{\ttfamily\mdseries
resolve}

{\ttfamily\mdseries
end}

\subsection[26.6.2 Unicon's Progend() revisited]{26.6.2 Unicon's Progend() revisited}

Having presented scope resolution, inheritance, and importing packages
and inheriting classes from other files via the uniclass.dir NDBM
files, we can finally show the complete semantic analysis in the
Unicon compiler, prior to writing out the syntax tree as Icon code:

{\ttfamily\mdseries
procedure Progend(x1)}


\bigskip

{\ttfamily\mdseries
\ \ \ package\_level\_syms := set()}

{\ttfamily\mdseries
\ \ \ package\_level\_class\_syms := set()}

{\ttfamily\mdseries
\ \ \ set\_package\_level\_syms(x1)}

{\ttfamily\mdseries
\ \ \ scopecheck\_superclass\_decs(x1)}


\bigskip

{\ttfamily\mdseries
\ \ \ outline := 1}

{\ttfamily\mdseries
\ \ \ outcol := 1}

{\ttfamily\mdseries
\ \ \ \#}

{\ttfamily\mdseries
\ \ \ \# export specifications for each class}

{\ttfamily\mdseries
\ \ \ \#}

{\ttfamily\mdseries
\ \ \ native := set()}

{\ttfamily\mdseries
\ \ \ every cl := classes.foreach\_t() do \{}

{\ttfamily\mdseries
\ \ \ \ \ \ cl.WriteSpec()}

{\ttfamily\mdseries
\ \ \ \ \ \ insert(native, cl)}

{\ttfamily\mdseries
\ \ \ \ \ \ \}}

{\ttfamily\mdseries
\ \ \ \#}

{\ttfamily\mdseries
\ \ \ \# import class specifications, transitively}

{\ttfamily\mdseries
\ \ \ \#}

{\ttfamily\mdseries
\ \ \ repeat \{}

{\ttfamily\mdseries
\ \ \ \ \ \ added := 0}

{\ttfamily\mdseries
\ \ \ \ \ \ every super := ((classes.foreach\_t()).foreachsuper() {\textbar} !imports) do \{}

{\ttfamily\mdseries
\ \ \ \ \ \ \ \ \ if /classes.lookup(super) then \{}

{\ttfamily\mdseries
\ \ \ \ \ \ \ \ \ \ \ \ added := 1}

{\ttfamily\mdseries
\ \ \ \ \ \ \ \ \ \ \ \ readspec(super)}

{\ttfamily\mdseries
\ \ \ \ \ \ \ \ \ \ \ \ cl := classes.lookup(super)}

{\ttfamily\mdseries
\ \ \ \ \ \ \ \ \ \ \ \ if /cl then halt({\textquotedbl}can't inherit class
'{\textquotedbl},super,{\textquotedbl}'{\textquotedbl})}

{\ttfamily\mdseries
\ \ \ \ \ \ \ \ \ \ \ \ iwrite({\textquotedbl} \ inherits {\textquotedbl}, super, {\textquotedbl} from {\textquotedbl},
cl.linkfile)}

{\ttfamily\mdseries
\ \ \ \ \ \ \ \ \ \ \ \ writelink(cl.dir, cl.linkfile)}

{\ttfamily\mdseries
\ \ \ \ \ \ \ \ \ \ \ \ outline +:= 1}

{\ttfamily\mdseries
\ \ \ \ \ \ \ \ \ \ \ \ \}}

{\ttfamily\mdseries
\ \ \ \ \ \ \ \}}

{\ttfamily\mdseries
\ \ \ \ if added = 0 then break}

{\ttfamily\mdseries
\ \ \}}

{\ttfamily\mdseries
\ \ \#}

{\ttfamily\mdseries
\ \ \# Compute the transitive closure of the superclass graph. Then}

{\ttfamily\mdseries
\ \ \# resolve inheritance for each class, and use it to apply scoping rules.}

{\ttfamily\mdseries
\ \ \#}

{\ttfamily\mdseries
\ \ every (classes.foreach\_t()).transitive\_closure()}

{\ttfamily\mdseries
\ \ every (classes.foreach\_t()).resolve()}


\bigskip

{\ttfamily\mdseries
\ \ scopecheck\_bodies(x1)}


\bigskip

{\ttfamily\mdseries
\ \ \ if {\textbackslash}thePackage then \{}

{\ttfamily\mdseries
\ \ \ \ \ \ every thePackage.insertsym(!package\_level\_syms)}

{\ttfamily\mdseries
\ \ \ \ \ \ \}}


\bigskip

{\ttfamily\mdseries
\ \ \ \#}

{\ttfamily\mdseries
\ \ \ \# generate output}

{\ttfamily\mdseries
\ \ \ \#}

{\ttfamily\mdseries
\ \ \ yyprint(x1)}

{\ttfamily\mdseries
\ \ \ write(yyout)}

\subsection[26.6.3 Other OOP Issues]{26.6.3 Other OOP Issues}

The primary mechanisms for object-oriented programming that we have
discussed so far include: classes, method invocation,
inheritance. There were certainly a few parts we glossed over (like
how a\$super.m() is implemented.) The main way to look for additional
issues we skipped is to read uni/unicon/idol.icn, which handles all
the object-oriented features and comes from the original Idol
preprocessor. Here are some thoughts from a scan of idol.icn:

\liststyleLxliv
\begin{itemize}
\item 
the preprocessor semi-parsed class and method headers in order to do inheritance. After the real (YACC-based) parser was
added, I hoped to remove the parsing code, but it is retained in order to handle class specifications in the
uniclass.dir NDBM files 
\item 
The classes in idol.icn correspond fairly directly to major syntax constructs; the compiler itself is object-oriented. 
\item 
Packages are a {\textquotedbl}virtual syntax construct{\textquotedbl}: no explicit representation in the source, but
stored in the uniclass.dir database 
\item 
There is a curious data structure, a tabular queue, or taque, that combines (hash) table lookup and preserves (lexical)
ordering. 
\item 
Aggregation and delegation patterns are used a lot. A class is an aggregate of methods, fields, etc. and delegates a lot
of its work to objects created for subparts of its overall syntax. 
\end{itemize}

\subsection[26.6.4 On Public Interfaces and Runtime Type Checking]{26.6.4 On Public Interfaces and Runtime Type Checking}

Object-oriented facilities are usually discussed in the context of
large complex applications where software engineering is an issue. We
don't usually need OOP for 100 line programs, but for 10,000+ line
programs it is often a big help.

Besides classes and packages, Unicon adds to Icon one additional
syntax construct in support of this kind of program: type checking and
coercion of parameters. Parameters and return values are the points at
which type errors usually occur, during an integration phase in a
large project where one person's code calls another. The type checking
and coercion syntax was inspired by the type checks done by the Icon
runtime system at the boundary where Icon program code calls the C
code for a given function or operator.


One additional comment about types is that the lack of types in
declarations for ordinary variables such as {\textquotedbl}local
x{\textquotedbl} does not prevent the Icon compiler iconc from
determining the exact types of well over 90\% of uses at compile time
using type inference. Type checking can generally be done at compile
time even if variable declarations do not refer to types... as long as
the type information is available across file and module boundaries.
