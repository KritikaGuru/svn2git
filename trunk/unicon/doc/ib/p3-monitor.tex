\chapter{Execution Monitoring}

This chapter describes the implementation of program execution
monitoring facilities in Unicon. Prior to these facilities, some
simple monitors of Icon programs had been written in C, using
instrumentation that was added to generate log files from the C
implementation of Icon's virtual machine, iconx. Development of these
tools in C was too labor intensive for experimental research work. The
primary goal of the monitoring facilities is to allow monitors for
Icon and Unicon to be written in Unicon instead of C.

The monitoring facilities were added by writing a dynamic loader for
Icon icode and extending the co-expression data type to allow the
loading and execution of programs as co-expressions within the same
virtual machine. This was considered viable to achieve high
performance, since the co-expression switch was frequently and
publically extolled by Ralph Griswold as being very high performance
(comparable to a procedure call in Icon), while transmitting
monitoring information to a separate process via pipes, sockets or
any other means that would require operating system synchronization
was considered too expensive.

Since that time, the co-expression type has gotten much slower in
Icon, whose current implementation has eliminated native assembler
co-expressions in favor of a portable pthreads implementation and
removed the monitoring facilities entirely. In Unicon, the monitoring
facilities remain a compelling reason to retain native co-expressions
indefinitely, since they constitute the core of the Unicon debugger
udb, and the profiler uprof.

\section{Dynamic Loading and the Program State}

Several major changes were required in order to support multiple
programs in the virtual machine. Loading the icode into a separate
malloced area was the easy part.  Numerous global and/or static C
variables contained program-specific state -- on a abstract level
they could be considered ``registers'' of the Icon virtual machine.
In practice, they must be saved/restored whenever switching between
programs, and it was quickly found that they were too numerous to
easily save and restore: switching time would be too slow. Instead,
these variables were all moved into a new structure, and a pointer
to this structure became the one variable that needed to be
saved/restored whenever switching between programs. The variable
was named {\tt curpstate} and the structure in src/h/rstructs.h
looks like:

\begin{iconcode}
/* \\
 * Program state encapsulation.  This consists \\
 * of the VARIABLE parts of many global structures. \\
 */ \\
struct progstate \{ \\
   long hsize;				/* size of icode */ \\
	/* hsize is a constant defined at load time, MT safe */ \\
   struct progstate *parent; \\
	/* parent is a constant defined at load time, MT safe */ \\
   struct progstate *next; \\
	/* next is a link list, seldom used, needs mutex */ \\
   struct descrip parentdesc;		/* implicit \&parent */ \\
	/* parentdesc is a constant defined at load time, MT safe */ \\
   struct descrip eventmask;		/* implicit \&eventmask */ \\
	/* eventmask is read-only (to me), MT safe */ \\
   struct descrip eventcount;		/* implicit \&eventcount */ \\
   struct descrip valuemask; \\
   struct descrip eventcode;		/* \&eventcode */ \\
   struct descrip eventval;		/* \&eventval */ \\
   struct descrip eventsource;		/* \&eventsource */ \\
\ \\
   /* Systems don't have more than, oh, about 50 signals, eh? \\
    * Currently in the system there is 40 of them            */ \\
   struct descrip Handlers[41]; \\
   /*int Inited; not used anymore */ \\
   int signal; \\
   /* \\
    * trapped variable keywords' values \\
    */ \\
   struct descrip Kywd\_err;          /* Probably mutex. not important now */ \\
   struct descrip Kywd\_prog; \\
\ \\
   struct descrip Kywd\_trc;         /* leave global for now   */ \\
   struct b\_coexpr *Mainhead; \\
   char *Code; \\
   char *Ecode; \\
   word *Records; \\
   int *Ftabp; \\
   \#ifdef FieldTableCompression \\
      short Ftabwidth, Foffwidth; \\
      unsigned char *Ftabcp, *Focp; \\
      short *Ftabsp, *Fosp; \\
      int *Fo; \\
      char *Bm; \\
   \#endif				/* FieldTableCompression */ \\
   dptr Fnames, Efnames; \\
   dptr Globals, Eglobals; \\
   dptr Gnames, Egnames; \\
   dptr Statics, Estatics; \\
   int NGlobals, NStatics; \\
   char *Strcons; \\
   struct ipc\_fname *Filenms, *Efilenms; \\
   struct ipc\_line *Ilines, *Elines; \\
  struct ipc\_line * Current\_line\_ptr;  /* not used or what ?*/ \\
\ \\
   \#ifdef Graphics \\
      struct descrip AmperX, AmperY, AmperRow, AmperCol; \\
      struct descrip AmperInterval;			/* \&interval */ \\
      struct descrip LastEventWin;			/* last Event() win */\\
      int LastEvFWidth; \\
      int LastEvLeading; \\
      int LastEvAscent; \\
      uword PrevTimeStamp;				/* previous timestamp */ \\
      uword Xmod\_Control, Xmod\_Shift, Xmod\_Meta;	/* control,shift,meta */ \\
      struct descrip Kywd\_xwin[2];			/* \&window + ... */\\
\ \\
   \#ifdef Graphics3D \\
      struct descrip AmperPick;				/* \&pick */ \\
   \#endif				/* Graphics3D */ \\
   \#endif				/* Graphics */ \\
\ \\   
   word Coexp\_ser;			/* this program's serial numbers */\\
   word List\_ser; \\
   word Intern\_list\_ser; \\
\#ifdef PatternType \\
   word Pat\_ser; \\
\#endif					/* PatternType */\\
   word Set\_ser; \\
   word Table\_ser; \\
\ \\
   word Kywd\_time\_elsewhere;		/* ???? TLS vs global \&time spent in other programs */ \\
   word Kywd\_time\_out;			/* ????  TLS vs global \&time at last program switch out */ \\
\ \\
\#ifdef Concurrent \\
   word mutexid\_stringtotal; \\
   word mutexid\_blocktotal; \\
   word mutexid\_coll;\\
\ \\
   struct region *Public\_stringregion;         /*  separate regions vs shared */ \\
   struct region *Public\_blockregion;          /*     same above   */ \\
\#endif					/* Concurrent */
\ \\
   struct region *stringregion;         /*  separate regions vs shared      */\\
   struct region *blockregion;          /*     same above     */ \\
\ \\
  /* in case we have separate heaps, i.e ThreadHeap is defined\\
   * total here will be only for "dead" threads */\\
   uword stringtotal;			/* cumulative total allocation */\\
   uword blocktotal;			/* cumulative total allocation */\\
\ \\
   word colltot;			/*  m      total number of collections */\\
   word collstat;			/*  u      number of static collect requests */\\
   word collstr;			/*  t      number of string collect requests */\\
   word collblk;			/*  ex     number of block collect requests */\\
\ \\
   struct descrip K\_main;\\
   struct b\_file K\_errout;\\
   struct b\_file K\_input;\\
   struct b\_file K\_output;\\
\ \\
   /* dynamic record types */\\
  int yyy;\\
  int Longest\_dr;\\
  struct b\_proc\_list **Dr\_arrays;\\
\ \\
   /*\\
    * Function Instrumentation Fields.\\
    */\\
\#ifdef Arrays \\
   int (*Cprealarray)(dptr, dptr, word, word);\\
   int (*Cpintarray)(dptr, dptr, word, word);\\
\#endif					/* Arrays */\\
   int (*Cplist)(dptr, dptr, word, word);\\
   int (*Cpset)(dptr, dptr, word);\\
   int (*Cptable)(dptr, dptr, word);\\
   void (*EVstralc)(word);\\
\#ifdef TSTATARG\\
   int (*Interp)(int,dptr, struct threadstate*);\\
\#else 		 	   	  	 /* TSTATARG */\\
   int (*Interp)(int,dptr);\\
\#endif 		 	   	  	 /* TSTATARG */\\
   int (*Cnvcset)(dptr,dptr);\\
   int (*Cnvint)(dptr,dptr);\\
   int (*Cnvreal)(dptr,dptr);\\
   int (*Cnvstr)(dptr,dptr);\\
   int (*Cnvtcset)(struct b\_cset *,dptr,dptr);\\
   int (*Cnvtstr)(char *,dptr,dptr);\\
   void (*Deref)(dptr,dptr);\\
   struct b\_bignum * (*Alcbignum)(word);\\
   struct b\_cset * (*Alccset)();\\
   struct b\_file * (*Alcfile)(FILE*,int,dptr);\\
   union block * (*Alchash)(int);\\
   struct b\_slots * (*Alcsegment)(word);\\
\#ifdef PatternType\\
   struct b\_pattern * (*Alcpattern)(word);\\
   struct b\_pelem * (*Alcpelem)(word, word *);\\
   int (*Cnvpattern)(dptr,dptr);\\
   int (*Internalmatch)(char*,int,int,struct b\_pelem*,int*,int*,int,int);\\
\#endif					/* PatternType */\\
   struct b\_list *(*Alclist\_raw)(uword,uword);\\
   struct b\_list *(*Alclist)(uword,uword);\\
   struct b\_lelem *(*Alclstb)(uword,uword,uword);\\
\#ifndef DescriptorDouble\\
   struct b\_real *(*Alcreal)(double);\\
\#endif					/* DescriptorDouble */\\
   struct b\_record *(*Alcrecd)(int, union block *);\\
   struct b\_refresh *(*Alcrefresh)(word *, int, int);\\
   struct b\_selem *(*Alcselem)(dptr, uword);\\
   char *(*Alcstr)(char *, word);\\
   struct b\_tvsubs *(*Alcsubs)(word, word, dptr);\\
   struct b\_telem *(*Alctelem)(void);\\
   struct b\_tvtbl *(*Alctvtbl)(dptr, dptr, uword);\\
   struct b\_tvmonitored *(*Alctvmonitored) (dptr);\\
   void (*Deallocate)(union block *);\\
   char * (*Reserve)(int, word);\\
\ \\
  struct threadstate *tstate, maintstate;\\
   \};
\end{iconcode}


\section{Event Instrumentation, Filtering and Reporting}

Interesting behavior within the Unicon virtual machine is instrumented
using macros. Originally the macros were defined as no-ops or as
output statements that would write to logfiles. In the monitoring
facilities, the macro can instead be defined to check whether that
event has been requested by a monitor, and if so, to report that
event.

The monitoring macros began with clean and simple form. They have been
elaborated many times and are now more subtle. Originally the macros
consisted of

\begin{iconcode}
\#define EVVal(value,event) ... \\
\#define EVValD(dp,event) ...
\end{iconcode}

\noindent
where EVVal() reported an integer event value while EVValD() took a
descriptor pointer and reported an arbitrary value. The second
argument in both cases was the event code, an integer indicating what
was happening.  These macros live in src/h/grttin.h where they are
automatically included in every compile by the rtt program.

\subsection{Filtering: Event Masks and Value Masks}

If all events were reported to the monitor, it would have complete
information but be overwhelmed. Filtering of desired events was
essential from the beginning, and was refined over time from a single
level to a two-level process.

Event codes of interest are specified by monitors using an event
set. Since the event codes are all in the range 0-255, the most
efficient representation of these sets is by means of a bit vector,
and the most convenient bit vector implementation already available
was the cset data type.  Initially, then, the definitions of EVVal()
and EVValD() looked something like

\begin{iconcode}
\#define EVVal(value,event) {\textbackslash} \\
\>   if (Testb((word)ToAscii(event)), curpstate-$>$eventmask) \{ {\textbackslash} \\
\>\>      MakeInt(value, *(curpstate-$>$parent-$>$eventval)); actparent(event); \}
\end{iconcode}

Event masks are csets, with individual characters ASCII values
indicating which event they denote. \texttt{Testb()} is the internal
cset membership test operation.  If the event is of interest (i.e. in
the program's event mask, set by a monitor), it is reported by placing
its value in the parent (monitor) \texttt{\&eventvalue} and calling
\texttt{actparent()} to activate and switch control to the parent,
thus accomplishing an event report.

Many of the event codes started with mnemonics, such as P for
``procedure call''.  Over time, mnemonics proved unhelpful and all
events were in fact re-ordered at some point for efficiency purposes,
such that events must be referred to by symbolic names associated with
their unsigned char integer codes.  These names are in src/h/monitor.h
and all begin with the prefex E\_, as in \texttt{E\_Pcall} for
``procedure call event''.

Event masks were essential in creating run-time monitoring facilities,
but in certain cases, even more filtering is needed.  For example, the
event for virtual machine instructions is extremely frequent.  An
additional layer of filtering was introduced in the form of value
masks, such that for each event code in the event mask, a (Unicon) set
of values of interest could be specified. The macro became so long
that it was preferable to use rtt's multi-line macro syntax
(\#begdef...\#enddef).    With value mask tests in
place the EVVal() macros look more like:

\begin{iconcode}
\#begdef RealEVVal(value,event,exint,entint) \\
\>   do \{ \\
\>\>    if (is:null(mycurpstate-$>$eventmask)) break; \\
\>\>    else if (!Testb((word)ToAscii(event), mycurpstate-$>$eventmask)) break; \\
\>\>    MakeInt(value, \&(mycurpstate-$>$parent-$>$eventval)); \\
\>\>      if (!is:null(mycurpstate-$>$valuemask) \&\& \\
\>\>	  (invaluemask(mycurpstate, event, \&(mycurpstate-$>$parent-$>$eventval)) \\
\>\>\>\>\>\>\>\>\>\> != Succeeded)) \\
\>\>\>	 break; \\
\>\>      exint; \\
\>      actparent(event); \\
\>      entint; \\
\>   \} while (0) \\
\#enddef					/* RealEVVal */
\end{iconcode}

The reason that these macro definitions are prefixed ``Real'' is that
they can in turn enabled or disabled on a per-eventcode basis via

\begin{iconcode}
\#begdef EVVal(value,event) \\
\#if event \\
\>   RealEVVal(value,event,/*noop*/,/*noop*/) \\
\#endif \\
\#enddef					/* EVVal */
\end{iconcode}

This is probably overkill.

\section{Instrumented vs. Uninstrumented Run-time Functions}

In Icon, the pervasive instrumentation of run-time events was seen to
cost about 50\% performance, even when not in use. For
this reason a separate copy of the iconx virtual machine was built
specifically for monitoring, an miconx.  It was workable but
inconvenient to be shipping an extra copy of the VM executable for
monitoring purposes; it took up extra space, and complicated the build
makefile.

For Unicon, a scheme was devised to compile both the instrumented and
full-speed uninstrumented versions of the 30 runtime functions that
had events in them.  For each such function, say cnv\_str() for
example, the body of the function is placed in a macro that is
instantiated twice, once with events and once without.  The macro is
parameterized by the event codes that that function has instrumented
within it and is capable of producing:

\begin{iconcode}
\#begdef cnv\_str\_macro(f, e\_aconv, e\_tconv, e\_nconv, e\_sconf, e\_fconv) \\
/* \\
 * cnv\_str - cnv:string(*s, *d), convert to a string \\
 */ \\
int f(dptr s, dptr d) \\
\>   \{ \\
\> ... \\
\> \#enddef
\ \\
\#ifdef MultiProgram \\
\#passthru \#undef cnv\_str \\
cnv\_str\_macro(cnv\_str,0,0,0,0,0) \\
cnv\_str\_macro(cnv\_str\_1,E\_Aconv,E\_Tconv,E\_Nconv,E\_Sconv,E\_Fconv) \\
\#else					/* MultiProgram */ \\
cnv\_str\_macro(cnv\_str,0,0,0,0,0) \\
\#endif					/* MultiProgram */
\end{iconcode}

So the definition of \texttt{cnv\_str} accomplishes the definition of
an instrumented \texttt{cnv\_str\_1} as a by-product.  In the current
implementation, instrumentation of a given function is an
all-or-nothing decision. The macros are capable of supporting
multiple, finer-grained instantiations with partial instrumentation,
if desired.

Whether to use the instrumented or uninstrumented version of each of
these functions is decided at runtime on a per-program basis and can
change over time. Which functions are in use are specified by a set of
function pointers in the current program state structure.  These
function pointers are calculated based on the program's event mask.
Whenever the event mask changes, the set of instrumented functions is
recalculated; this is normally during EvGet() and lives in a function
named \texttt{assign\_event\_functions()}.

\begin{iconcode}
void assign\_event\_functions(struct progstate *p, struct descrip cs) \\
\{ \\
\>   p-$>$eventmask = cs; \\
\ \\
\>   /* \\
\>    * Most instrumentation functions depend on a single event. \\
\>    */ \\
\>   p-$>$Cplist = \\
\>\>      ((Testb((word)ToAscii(E\_Lcreate), cs)) ? cplist\_1 : cplist\_0);
\> ... \\
\end{iconcode}

Using function pointers in the program state instead of direct
function calls for these 30 common functions in the Unicon virtual
machine imposes a cost on runtime performance due to added indirect
memory references performed during those calls. The cost was measured
and believed to be under 5\% at the time it was introduced, which was
considered acceptable for the added capabilities and convenience of
having monitoring available and standard within the regular Unicon
virtual machine.
