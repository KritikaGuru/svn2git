\chapter{Execution Monitoring}

This chapter describes the implementation of program execution
monitoring facilities in Unicon. Prior to these facilities, some
simple monitors of Icon programs had been written in C, using
instrumentation that was added to generate log files from the C
implementation of Icon's virtual machine, iconx. Development of these
tools in C was too labor intensive for experimental research work. The
primary goal of the monitoring facilities is to allow monitors for
Icon and Unicon to be written in Icon and Unicon instead of C.

The monitoring facilities were added by writing a dynamic loader for
Icon icode and extending the co-expression data type to allow the
loading and execution of programs as co-expressions within the same
virtual machine. This was considered necessary to achieve high
performance, since the co-expression switch was frequently and
publically lauded by Ralph Griswold as being very high performance
(comparable to a procedure call in Icon), while transmitting
monitoring information to a separate process via pipes, sockets or
any other means that would require operating system synchronization
was considered too expensive.

Since that time, the co-expression type has gotten much slower in
Icon, the current implementation having eliminated native assembler
co-expressions in favor of a portable pthreads implementation. The
monitoring facilities remain a compelling reason for Unicon to retain
native co-expressions indefinitely, since they constitute the core
of the Unicon debugger udb, and the profiler uprof.

\section{Dynamic Loading and the Program State}

\section{Event Instrumentation, Filtering and Reporting}

Interesting behavior within the Unicon virtual machine is instrumented
using macros. Originally the macros were defined as no-ops or as
output statements that would write to logfiles. In the monitoring
facilities, the macro can instead be defined to check whether that
event has been requested by a monitor, and if so, to report that
event.

The monitoring macros began with clean and simple form. They have been
elaborated many times and are now more subtle. Originally the macros
consisted of

\begin{iconcode}
\#define EVVal(value,event) ... \\
\#define EVValD(dp,event) ...
\end{iconcode}

\noindent
where EVVal() reported an integer event value while EVValD() took a
descriptor pointer and reported an arbitrary value. The second
argument in both cases was the event code, an integer indicating what
was happening.  These macros live in src/h/grttin.h where they are
automatically included in every compile by the rtt program.

\subsection{Filtering: Event Masks and Value Masks}

If all events were reported to the monitor, it would have complete
information but be overwhelmed. Filtering of desired events was
essential from the beginning, and was refined over time from a single
level to a two-level process.

Event codes of interest are specified by monitors using an event
set. Since the event codes are all in the range 0-255, the most
efficient representation of these sets is by means of a bit vector,
and the most convenient bit vector implementation already available
was the cset data type.  Initially, then, the definitions of EVVal()
and EVValD() looked something like

\begin{iconcode}
\#define EVVal(value,event) {\textbackslash} \\
\>   if (Testb((word)ToAscii(event)), curpstate-$>$eventmask) \{ {\textbackslash} \\
\>\>      MakeInt(value, *(curpstate-$>$parent-$>$eventval)); actparent(event); \}
\end{iconcode}

Event masks are csets, with individual characters ASCII values
indicating which event they denote. \texttt{Testb()} is the internal
cset membership test operation.  If the event is of interest (i.e. in
the program's event mask, set by a monitor), it is reported by placing
its value in the parent (monitor) \texttt{\&eventvalue} and calling
\texttt{actparent()} to activate and switch control to the parent,
thus accomplishing an event report.

Many of the event codes started with mnemonics, such as P for
``procedure call''.  Over time, mnemonics proved unhelpful and all
events were in fact re-ordered at some point for efficiency purposes,
such that events must be referred to by symbolic names associated with
their unsigned char integer codes.  These names are in src/h/monitor.h
and all begin with the prefex E\_, as in \texttt{E\_Pcall} for
``procedure call event''.

Event masks were essential in creating run-time monitoring facilities,
but in certain cases, even more filtering is needed.  For example, the
event for virtual machine instructions is extremely frequent.  An
additional layer of filtering was introduced in the form of value
masks, such that for each event code in the event mask, a (Unicon) set
of values of interest could be specified. The macro became so long
that it was preferable to use rtt's multi-line macro syntax
(\#begdef...\#enddef).    With value mask tests in
place the EVVal() macros look more like:

\begin{iconcode}
\#begdef RealEVVal(value,event,exint,entint) \\
\>   do \{ \\
\>\>    if (is:null(mycurpstate-$>$eventmask)) break; \\
\>\>    else if (!Testb((word)ToAscii(event), mycurpstate-$>$eventmask)) break; \\
\>\>    MakeInt(value, \&(mycurpstate-$>$parent-$>$eventval)); \\
\>\>      if (!is:null(mycurpstate-$>$valuemask) \&\& \\
\>\>	  (invaluemask(mycurpstate, event, \&(mycurpstate-$>$parent-$>$eventval)) \\
\>\>\>\>\>\>\>\>\>\> != Succeeded)) \\
\>\>\>	 break; \\
\>\>      exint; \\
\>      actparent(event); \\
\>      entint; \\
\>   \} while (0) \\
\#enddef					/* RealEVVal */
\end{iconcode}

The reason that these macro definitions are prefixed ``Real'' is that
they can in turn enabled or disabled on a per-eventcode basis via

\begin{iconcode}
\#begdef EVVal(value,event) \\
\#if event \\
\>   RealEVVal(value,event,/*noop*/,/*noop*/) \\
\#endif \\
\#enddef					/* EVVal */
\end{iconcode}

This is probably overkill.

\section{Instrumented vs. Uninstrumented Run-time Functions}

In Icon, the pervasive instrumentation of run-time events was seen to
cost about 50\% performance, even when not in use. For
this reason a separate copy of the iconx virtual machine was built
specifically for monitoring, an miconx.  It was workable but
inconvenient to be shipping an extra copy of the VM executable for
monitoring purposes; it took up extra space, and complicated the build
makefile.

For Unicon, a scheme was devised to compile both the instrumented and
full-speed uninstrumented versions of the 30 runtime functions that
had events in them.  For each such function, say cnv\_str() for
example, the body of the function is placed in a macro that is
instantiated twice, once with events and once without.  The macro is
parameterized by the event codes that that function has instrumented
within it and is capable of producing:

\begin{iconcode}
\#begdef cnv\_str\_macro(f, e\_aconv, e\_tconv, e\_nconv, e\_sconf, e\_fconv) \\
/* \\
 * cnv\_str - cnv:string(*s, *d), convert to a string \\
 */ \\
int f(dptr s, dptr d) \\
\>   \{ \\
\> ... \\
\> \#enddef
\ \\
\#ifdef MultiThread \\
\#passthru \#undef cnv\_str \\
cnv\_str\_macro(cnv\_str,0,0,0,0,0) \\
cnv\_str\_macro(cnv\_str\_1,E\_Aconv,E\_Tconv,E\_Nconv,E\_Sconv,E\_Fconv) \\
\#else					/* MultiThread */ \\
cnv\_str\_macro(cnv\_str,0,0,0,0,0) \\
\#endif					/* MultiThread */
\end{iconcode}

So the definition of \texttt{cnv\_str} accomplishes the definition of
an instrumented \texttt{cnv\_str\_1} as a by-product.  In the current
implementation, instrumentation of a given function is an
all-or-nothing decision. The macros are capable of supporting
multiple, finer-grained instantiations with partial instrumentation,
if desired.

Whether to use the instrumented or uninstrumented version of each of
these functions is decided at runtime on a per-program basis and can
change over time. Which functions are in use are specified by a set of
function pointers in the current program state structure.  These
function pointers are calculated based on the program's event mask.
Whenever the event mask changes, the set of instrumented functions is
recalculated; this is normally during EvGet() and lives in a function
named \texttt{assign\_event\_functions()}.

\begin{iconcode}
void assign\_event\_functions(struct progstate *p, struct descrip cs) \\
\{ \\
\>   p-$>$eventmask = cs; \\
\ \\
\>   /* \\
\>    * Most instrumentation functions depend on a single event. \\
\>    */ \\
\>   p-$>$Cplist = \\
\>\>      ((Testb((word)ToAscii(E\_Lcreate), cs)) ? cplist\_1 : cplist\_0);
\> ... \\
\end{iconcode}

Using function pointers in the program state instead of direct
function calls for these 30 common functions in the Unicon virtual
machine imposes a cost on runtime performance due to added indirect
memory references performed during those calls. The cost was measured
and believed to be under 5\% at the time it was introduced, which was
considered acceptable for the added capabilities and convenience of
having monitoring available and standard within the regular Unicon
virtual machine.
