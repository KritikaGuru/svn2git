\chapter{Execution Monitoring}

This chapter describes the implementation of program execution
monitoring facilities in Unicon. Prior to these facilities, some
simple monitors of Icon programs had been written in C, using
instrumentation that was added to generate log files from the C
implementation of Icon's virtual machine, iconx. Development of these
tools in C was too labor intensive for experimental research work. The
primary goal of the monitoring facilities is to allow monitors for
Icon and Unicon to be written in Icon and Unicon instead of C.

The monitoring facilities were added by writing a dynamic loader for
Icon icode and extending the co-expression data type to allow the
loading and execution of programs as co-expressions within the same
virtual machine. This was considered necessary to achieve high
performance, since the co-expression switch was frequently and
publically lauded by Ralph Griswold as being very high performance
(comparable to a procedure call in Icon), while transmitting
monitoring information to a separate process via pipes, sockets or
any other means that would require operating system synchronization
was considered too expensive.

Since that time, the co-expression type has gotten much slower in
Icon, the current implementation having eliminated native assembler
co-expressions in favor of a portable pthreads implementation. The
monitoring facilities remain a compelling reason for Unicon to retain
native co-expressions indefinitely, since they constitute the core
of the Unicon debugger udb, and the profiler uprof.

\section{Dynamic Loading and the Program State}

\section{Instrumentation}

\section{Event Requests, Filtering, and Reporting}
