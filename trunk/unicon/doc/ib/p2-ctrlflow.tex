\chapter{Control Flow Optimizations}

\section{Naive Code Generation}

Naive code generation does not consider the effects and needs of the
immediately surrounding program. The result is often a poor use of the
target language. Even sophisticated code generation schemes that
consider the effects of relatively large pieces of the program still
produce poor code at the boundaries between the pieces. This problem
is typically solved by adding a \textit{peephole optimizer} to the
compiler to improve the generated code [.peep1,Wulf,Tanenbaum
peephole,dragon.]. A peephole optimizer looks at several instructions
that are adjacent (in terms of execution) and tries to replace the
instructions by better, usually fewer, instructions. It typically
analyzes a variety of properties of the instructions such as
addressing modes and control flow.

The Icon compiler has a peephole optimizer that works on the internal
form of the generated C code and deals only with control flow. The
previous examples of generated code contain a number of instances of
code where control flow can be handled much better. For example, it is
possible to entirely eliminate the following code fragment generated
for the example explaining procedure suspension.

\goodbreak
\begin{iconcode}
\>switch (signal) \{\\
\>\>case A\_Resume:\\
\>\>\>goto L2 /* bound */;\\
\>\}\\
L2: /* bound */\\
\end{iconcode}

\noindent
This code is produced because the code generator does not take into
account the fact that the bounding label happens to immediately follow
the test.


\section{Success Continuations}

For the C code in the preceding example, it is quite possible that a C
compiler would produce machine code that its own peephole optimizer
could eliminate. However, it is unlikely that a C compiler would
optimize naively generated success continuations. An earlier example
of code generation produced the continuation:

\goodbreak
\begin{iconcode}
static int P02\_main()\\
\{\\
\>register struct PF00\_main *rpfp;\\
\\
\>rpfp = (struct PF00\_main *)pfp;\\
\>switch (O0o\_numeq(2, \&(rpfp->tend.d[1]), \&trashcan, (continuation)NULL)) \{\\
\>\>case A\_Continue:\\
\>\>\>break;\\
\>\>case A\_Resume:\\
\>\>\>return A\_Resume;\\
\>\>\}\\
\>return 4; /* bound */\\
\}\\
\end{iconcode}

\noindent
If the statement 

\iconline{return 4; /* bound */ }

\noindent
is brought into the \texttt{switch} statement, replacing the
\texttt{break}, then \texttt{P02\_main} consists of a simple operation
call (a C call with associated associated signal handling code). This
operation call is

\goodbreak
\begin{iconcode}
switch (O0o\_numeq(2, \&(rpfp->tend.d[1]), \&trashcan, (continuation)NULL)) \{\\
\>case A\_Continue:\\
\>\>return 4; /* bound */\\
\>case A\_Resume:\\
\>\>return A\_Resume;\\
\}\\
\end{iconcode}

\noindent
\texttt{P02\_main} is called directly in two places in the following code. 

\goodbreak
\begin{iconcode}
frame.tend.d[2].dword = D\_Integer;\\
frame.tend.d[2].vword.integr = 1;\\
switch (P02\_main()) \{\\
\>case A\_Resume:\\
\>\>goto L2 /* alt */;\\
\>case 4 /* bound */:\\
\>\>goto L4 /* bound */;\\
\>\}\\
L2: /* alt */\\
\>frame.tend.d[2].dword = D\_Integer;\\
\>frame.tend.d[2].vword.integr = 2;\\
\>switch (P02\_main()) \{\\
\>\>case A\_Resume:\\
\>\>\>goto L5 /* else */;\\
\>\>case 4 /* bound */:\\
\>\>\>goto L4 /* bound */;\\
\>\>\}\\
L4: /* bound */\\
\end{iconcode}


A direct call to a trivial function can reasonably be replaced by the
body of that function. When this is done for a continuation, it is
necessary to \textit{compose} the signal handling code of the body of
a continuation with that of the call. This is accomplished by
replacing each return statement in the body with the action in the
call corresponding to the signal returned. The following table
illustrates the signal handling composition for the first call in the
code.  The resulting code checks the signal from \texttt{O0o\_numeq} and
performs the final action.

\begin{center}
\tablefirsthead{}
\tablehead{}
\tabletail{}
\tablelasttail{}
%\begin{xtabular}{|m{1.8in}|m{1.8in}|m{1.1in}|}
\begin{xtabular}{|c|c|c|}
\hline
~~signal from \texttt{O0o\_numeq}~~  &
~~signal from \texttt{P02\_main}~~  &
~~final action~~ \\\hline
\texttt{A\_Continue} & 4                  & \texttt{goto L4;}\\
\texttt{A\_Resume}   & \texttt{A\_Resume} & \texttt{goto L2;}\\\hline
\end{xtabular}
\end{center}

\noindent
The result of in-lining \texttt{P02\_main} is 

\goodbreak
\begin{iconcode}
frame.tend.d[2].dword = D\_Integer;\\
frame.tend.d[2].vword.integr = 1;\\
switch (O0o\_numeq(2, \&frame.tend.d[1], \&trashcan, (continuation)NULL)) \{\\
\>case A\_Continue:\\
\>\>goto L4 /* bound */;\\
\>case A\_Resume:\\
\>\>goto L2 /* alt */;\\
\>\}\\
L2: /* alt */\\
\>frame.tend.d[2].dword = D\_Integer;\\
\>frame.tend.d[2].vword.integr = 2;\\
\>switch (O0o\_numeq(2, \&frame.tend.d[1], \&trashcan,%
                    (continuation)NULL)) \{\\
\>\>case A\_Continue:\\
\>\>\>goto L4 /* bound */;\\
\>\>case A\_Resume:\\
\>\>\>goto L5 /* else */;\\
\>\}\\
L4: /* bound */\\
\end{iconcode}

With a little more manipulation, the \texttt{switch} statements can be
converted into \texttt{if} statements and the label \texttt{L2} can be
eliminated:

\goodbreak
\begin{iconcode}
frame.tend.d[2].dword = D\_Integer;\\
frame.tend.d[2].vword.integr = 1;\\
if (O0o\_numeq(2, \&frame.tend.d[1], \&trashcan,%
              (continuation)NULL) == A\_Continue)\\
\>goto L4 /* bound */;\\
frame.tend.d[2].dword = D\_Integer;\\
frame.tend.d[2].vword.integr = 2;\\
if (O0o\_numeq(2, \&frame.tend.d[1], \&trashcan,%
              (continuation)NULL) == A\_Resume)\\
\>goto L5 /* else */;\\
L4: /* bound */\\
\end{iconcode}


The Icon compiler's peephole optimizer recognizes two kinds of trivial
continuations. The kind illustrated in the previous example consists
of a single call with associated signal handling. The other kind
simply consists of a single return-signal statement. As in the above
example, continuations do not usually meet this definition of
triviality until control flow optimizations are performed on them. For
this reason, the Icon compiler's peephole optimizer must perform some
optimizations that could otherwise be left to the C compiler.


\section{Iconc's Peephole Optimizer}

The peephole optimizer performs the following optimizations: 

\liststyleLxxxiv
\begin{itemize}
\item 
elimination of unreachable code 
\item 
elimination of gotos immediately preceding their destinations 
\item 
collapse of branch chains 
\item 
deletion of unused labels 
\item 
collapse of trivial call chains (that is, in-lining trivial continuations) 
\item 
deletion of unused continuations 
\item 
simplification of signal checking 
\end{itemize}

Unreachable code follows a \texttt{goto} or a \texttt{return}, and it
continues to the first referenced label or to the end of the function.
This optimization may eliminate code that returns signals, thereby
reducing the number of signals that must be handled by a continuation
call. This provides another reason for performing this traditional
optimization in the Icon compiler rather than letting the C compiler
do it. This code is eliminated when the fix-up pass for signal
handling is being performed. \texttt{goto}s immediately preceding their labels
also are eliminated at this time.

Unused labels usually are eliminated when the code is written out, but
they may be deleted as part of a segment of unreachable code. Unused
continuations are simply not written out.

A branch chain is formed when the destination of a \texttt{goto} is
another \texttt{goto} or a \texttt{return}. A \texttt{break} in a
\texttt{switch} statement is treated as a \texttt{goto}. There may be
several \texttt{goto}s in a chain. Each \texttt{goto} is replaced by
the \texttt{goto} or \texttt{return} at the end of the chain. This may
leave some labels unreferenced and may leave some of the intermediate
\texttt{goto}s unreachable. Branch chains are collapsed during the
fix-up pass.

Inter-function optimization is not traditionally considered a peephole
optimization. This is because human beings seldom write trivial
functions and most code generators do not produce continuations. The
Icon compiler, however, uses calls to success continuations as freely
as it uses \texttt{goto}s. Therefore collapsing trivial call chains is
as important as collapsing branch chains.

There are two kinds of calls to trivial continuations: direct calls
and indirect calls through an operation. A direct call always can be
replaced by the body of the continuation using signal handling code
that is a composition of that in the continuation and that of the
call. If the continuation consists of just a \texttt{return}
statement, this means that the call is replaced by the action
associated with the returned signal: either another \texttt{return}
statement or a \texttt{goto} statement. For continuations consisting
of a call, the composition is more complicated, as demonstrated by the
example given earlier in this chapter.

In the case of an indirect call through an operation, the continuation
cannot be placed in line. However, there is an optimization that can
be applied. Under some circumstances, the compiler produces a
continuation that simply calls another continuation. For example, this
occurs when compiling the Icon expression

\iconline{every write(!x | end) }

The compiler allocates a continuation for the alternation, then
compiles the expression \texttt{!x}. The element generation operator
suspends, so the compiler allocates a continuation for it and code
generation proceeds in this continuation.  However, the end of the
first alternative has been reached so the only code for this
continuation is a call to the continuation for the alternation. The
continuation for the alternation contains the code for the invocation
of \texttt{write} and for the end of the \texttt{every} control
structure. The code for the first alternative is

\goodbreak
\begin{iconcode}
frame.tend.d[2].dword = D\_Var;\\
frame.tend.d[2].vword.descptr = \&frame.tend.d[0] /* x */;\\
\\
switch (O0e\_bang(1, \&frame.tend.d[2], \&frame.tend.d[1], P02\_main)) \{\\
\>case A\_Resume:\\
\>\>goto L1 /* alt */;\\
\>\}\\
\ L1: /* alt */\\
\end{iconcode}

\noindent
The code for the two continuations are 

\goodbreak
\begin{iconcode}
static int P02\_main()\\
\{\\
\>switch (P03\_main()) \{\\
\>\>case A\_Resume:\\
\>\>\>return A\_Resume;\\
\>\>\}\\
\}\\
\end{iconcode}

\begin{iconcode}
static int P03\_main()\\
\{\\
\>register struct PF00\_main *rpfp;\\
\\
\>rpfp = (struct PF00\_main *)pfp;\\
\>F0c\_write(1, \&rpfp->tend.d[1], \&trashcan, (continuation)NULL);\\
\>return A\_Resume;\\
\}\\
\end{iconcode}

\noindent
The call to \texttt{O0e\_bang} can be optimized by passing the
continuation \texttt{P03\_main} in place of \texttt{P02\_main}.

The optimizations that collapse trivial call chains are performed
during the fix-up pass for signal handling.

The final peephole optimization involves simplifying the signal
handling code associated with a call. In general, signals are handled
with a \texttt{switch} statement containing a \texttt{case} clause for
each signal. The C compiler does not know that these signals are the
only values that are ever tested by the \texttt{switch} statement, nor
is the C compiler likely to notice that some cases simply pass along
to the next function down in the call chain the signal that was
received. The Icon compiler can use this information to optimize the
signal handling beyond the level to which the C compiler is able to
optimize it. The optimizer may replace the general form of the
\texttt{switch} statement with a \texttt{switch} statement utilizing a
\texttt{default} clause or with an \texttt{if} statement. In some
cases, the optimizer completely eliminates signal checking. This
optimization is done when the code is written.

