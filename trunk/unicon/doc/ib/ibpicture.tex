%-------------------------------------------------------------------------------
% Macros for the Icon Implementation Book LaTeX pictures
%
%    Don Ward
%    December 2016
%
%  The macros are designed so that if the same starting coord is used,
%  the components of the box come out in the right place. (x,y) is
%  the location of the bottom left-hand corner of the box.
%  eg.
%
% \put(x,y){\dvbox{title}{flags}{}}% Note that #3 (value) is null because we put a rdptr there
% \put(x,y){\leftboxlabels{top-left}{bottom-left}}
% \put(x,y){\trboxlabel{top-right-label}}
% \put(x,y){\rdptr{30}{20}}
% \put(x,y){\upetc}
%
% produces a picture (roughly) like this
%
%                                 .
%                                 .
%                  |              .               |
%                  |------------------------------|
%       top-left   | flags                  title |   top-right-label
%                  |..............................|
%    bottom-left   |                      --------+------------
%                  |------------------------------|           |
%              (x,y)                                          |
%                                                             V
%
%   The boxes are mostly 100pt wide and 32pt high, although some are 48pt high
%   and there are three  half-height (16pt) boxes, for use when there is an odd
%   number of 'slots' in the picture).
%-------------------------------------------------------------------------------
% Draw a descriptor/value box  \dvbox{title}{flags}{value}
%                  |------------------------------|
%                  | flags                  title |
%                  |..............................|
%                  |                        value |
%                  |------------------------------|
\newcommand{\dvbox}[3]{
\begin{picture}(120,32)
\multiput(0,16)(2,0){50}{\line(1,0){1}}% The dotted line in the middle
\put(0,0){\framebox(100,32){}}%          The box
\put(-2,-4){\makebox(100,32)[tr]{#1}}%   title
\put(2,-4){\makebox(100,32)[tl]{#2}}%    flags
\put(-2,4){\makebox(100,32)[br]{#3}}%    value
\end{picture}
}
%-------------------------------------------------------------------------------
% Draw a descriptor/value box with an arrow \dvptrbox{title}{flags}{len}{value}
% (len is the length of the arrow; value is placed after it, to the right)
%                  |------------------------------|
%                  | flags                  title |
%                  |..............................|
%                  |                       -------+---------------->  value
%                  |------------------------------|
\newcommand{\dvptrbox}[4]{
\begin{picture}(120,32)
\multiput(0,16)(2,0){50}{\line(1,0){1}}% The dotted line in the middle
\put(0,0){\framebox(100,32){}}%          The box
\put(-2,-4){\makebox(100,32)[tr]{#1}}%    title
\put(2,-4){\makebox(100,32)[tl]{#2}}%     flags
\put(80,8){\vector(1,0){#3}}%             arrow
\put(90,6){\hspace{#3pt}#4}%              value
\end{picture}
}
%-------------------------------------------------------------------------------
% Draw a block box with a single value \blkbox{title}{value_1}
% A block box is like a dvbox, but no flags and a solid separator line
\newcommand{\blkbox}[2]{
\begin{picture}(120,32)
\put(0,16){\line(1,0){100}}%             The line in the middle
\put(0,0){\framebox(100,32){}}%          The box
\put(-2,-4){\makebox(100,32)[tr]{#1}}%    title
\put(-2,4){\makebox(100,32)[br]{#2}}%     value
\end{picture}
}
%-------------------------------------------------------------------------------
% Draw a block box with a single value and an arrow \bklptrkbox{title}{len}{value}
\newcommand{\blkptrbox}[3]{
\begin{picture}(120,32)
\put(0,16){\line(1,0){100}}%             The line in the middle
\put(0,0){\framebox(100,32){}}%          The box
\put(-2,-4){\makebox(100,32)[tr]{#1}}%   title
\put(80,8){\vector(1,0){#2}}%            arrow
\put(90,6){\hspace{#2pt}#3}%             value
\end{picture}
}
%-------------------------------------------------------------------------------
% Draw a block box with two arrows and values \dblptrkbox{len}{value1}{value2}
\newcommand{\dblkptrbox}[3]{
\begin{picture}(120,32)
\put(0,16){\line(1,0){100}}%             The line in the middle
\put(0,0){\framebox(100,32){}}%          The box
\put(80,24){\vector(1,0){#1}}%           arrow
\put(90,22){\hspace{#1pt}#2}%            value1
\put(80,8){\vector(1,0){#1}}%            arrow
\put(90,6){\hspace{#1pt}#3}%             value2
\end{picture}
}
%-------------------------------------------------------------------------------
% Draw a block box with two values \blkboxtwo{title}{value_1}{value_2}
% This box is 48pt high, rather than 32
\newcommand{\blkboxtwo}[3]{
\begin{picture}(120,48)
\put(0,16){\line(1,0){100}}%             Lower separator
\put(0,32){\line(1,0){100}}%             Upper separator
\put(0,0){\framebox(100,48){}}%          The box
\put(-2,-4){\makebox(100,48)[tr]{#1}}%   title
\put(-2,24){\makebox(100,32)[br]{#2}}%   value_1
\put(-2,4){\makebox(100,48)[br]{#3}}%    value_2
\end{picture}
}
%-------------------------------------------------------------------------------
% Draw a block box with a large value \blklrgbox{title}{value}
\newcommand{\blklrgbox}[2]{
\begin{picture}(120,48)
\put(0,32){\line(1,0){100}}%             Separator
\multiput(0,16)(95,0){2}{\line(1,0){5}}% Decorative flim-flam
\put(0,0){\framebox(100,48){}}%          The box
\put(-2,-4){\makebox(100,48)[tr]{#1}}%   title
\put(0,0){\makebox(100,32){#2}}%         value
\end{picture}
}
%-------------------------------------------------------------------------------
% Half height boxes
%
% Draw a word box (half height) \wordbox{title}{flags}
\newcommand{\wordbox}[2]{
\begin{picture}(120,16)
\put(1.5,0){\framebox(100,16){}}%          The box
\put(-2,-4){\makebox(100,16)[tr]{#1}}%   title
\put(2,-4){\makebox(100,16)[tl]{#2}}%    flags
\end{picture}
}
%-------------------------------------------------------------------------------
% Draw a word box (half height) with an arrow \wordptr{len}{value}
\newcommand{\wordptr}[2]{
\begin{picture}(120,16)
\put(0,0){\framebox(100,16){}}%          The box
\put(80,8){\vector(1,0){#1}}%             arrow
\put(90,6){\hspace{#1pt}#2}%              value
\end{picture}
}
%-------------------------------------------------------------------------------
% Draw a word box (half height) with a null pointer symbol and a label 
% \nullptrbox{label}
\newcommand{\nullptrbox}[1]{
\begin{picture}(120,16)
\put(0,16){\line(1,0){100}}%             The line in the middle
\put(0,0){\framebox(100,16){}}%          The box
\put(-2,4){\makebox(100,32)[br]{\o}}%    Null ptr symbol
\put(10,8){\makebox(100,32)[br]{\makebox(0,0)[l]{#1}}}% label
\end{picture}
}
%-------------------------------------------------------------------------------
% Box labels
% Draw two right-aliged left hand labels  \leftboxlabels{top-left}{bottom-left}
\newcommand{\leftboxlabels}[2]{
\begin{picture}(120,32)
\put(-10,-8){\makebox(100,32)[tl]{\makebox(0,0)[r]{#1}}}
\put(-10,8){\makebox(100,32)[bl]{\makebox(0,0)[r]{#2}}}
\end{picture}
}
% Draw two left-aliged right hand labels  \rightboxlabels{top-right}{bottom-right}
\newcommand{\rightboxlabels}[2]{
\begin{picture}(120,32)
\put(10,-8){\makebox(100,32)[tr]{\makebox(0,0)[l]{#1}}}
\put(10,8){\makebox(100,32)[br]{\makebox(0,0)[l]{#2}}}
\end{picture}
}
%-------------------------------------------------------------------------------
% Draw individual labels (tl = top left, br = bottom right etc.)
\newcommand{\tlboxlabel}[1]{
\begin{picture}(120,32)
\put(-10,-8){\makebox(100,32)[tl]{\makebox(0,0)[r]{#1}}}
\end{picture}
}
\newcommand{\blboxlabel}[1]{
\begin{picture}(120,32)
\put(-10,8){\makebox(100,32)[bl]{\makebox(0,0)[r]{#1}}}
\end{picture}
}
\newcommand{\trboxlabel}[1]{
\begin{picture}(120,32)
\put(10,-8){\makebox(100,32)[tr]{\makebox(0,0)[l]{#1}}}
\end{picture}
}
\newcommand{\brboxlabel}[1]{
\begin{picture}(120,32)
\put(10,8){\makebox(100,32)[br]{\makebox(0,0)[l]{#1}}}
\end{picture}
}
%-------------------------------------------------------------------------------
% Arrows (from the v-word)
\newcommand{\rptr}[1]{%  An arrow (of length #1) to the right
\begin{picture}(120,32)
\put(80,8){\vector(1,0){#1}}
\end{picture}
}
\newcommand{\lptr}[1]{%  An arrow (of length #1) to the left
\begin{picture}(120,32)
\put(20,8){\vector(-1,0){#1}}
\end{picture}
}
% Arrows with a right-angle bend (ru = right then up, ld = left then down etc.)
\newcommand{\rdptr}[2]{
\begin{picture}(120,32)
\put(80,8){\line(1,0){#1}}
\put(80,8){\hspace{#1pt}\vector(0,-1){#2}}
\end{picture}
}
\newcommand{\ruptr}[2]{
\begin{picture}(120,32)
\put(80,8){\line(1,0){#1}}
\put(80,8){\hspace{#1pt}\vector(0,1){#2}}
\end{picture}
}
\newcommand{\ldptr}[2]{
\begin{picture}(120,32)
\put(20,8){\line(-1,0){#1}}
\put(20,8){\hspace{-#1pt}\vector(0,-1){#2}}
\end{picture}
}
\newcommand{\luptr}[2]{
\begin{picture}(120,32)
\put(20,8){\line(-1,0){#1}}
\put(20,8){\hspace{-#1pt}\vector(0,1){#2}}
\end{picture}
}
%-------------------------------------------------------------------------------
% Register pointers (to the vword)
\newcommand{\lregptr}[2]{
\begin{picture}(32,32)
\put(-2,12){\makebox(0,0)[r]{\makebox(0,-2)[r]{#1}\hspace{6pt}\vector(1,0){#2}}}
\end{picture}
}
\newcommand{\rregptr}[2]{
\begin{picture}(32,32)
\put(102,8){\makebox(0,0)[l]{\hspace{#2pt}\vector(-1,0){#2}}}
\put(98,12){\makebox(0,0)[l]{\hspace{#2pt}\hspace{10pt}{\makebox(0,-8)[l]{#1}}}}
\end{picture}
}
% Register pointers (to the dword)
\newcommand{\luregptr}[2]{
\begin{picture}(32,32)
\put(-2,28){\makebox(0,0)[r]{\makebox(0,-2){#1}\hspace{10pt}\vector(1,0){#2}}}
\end{picture}
}
\newcommand{\ruregptr}[2]{
\begin{picture}(32,32)
\put(102,24){\makebox(0,0)[l]{\hspace{#2pt}\vector(-1,0){#2}}}
\put(98,28){\makebox(0,0)[l]{\hspace{#2pt}\hspace{10pt}{\makebox(0,-8)[l]{#1}}}}
\end{picture}
}
%-------------------------------------------------------------------------------
% vertical dots and bars
\newcommand{\downetc}{% NB. This picture goes BELOW it's origin
\begin{picture}(120,16)(-4,0)
\put(0,0){\line(1,0){100}}
\put(0,0){\line(0,-1){16}}
\put(100,0){\line(0,-1){16}}
\put(50,-8){.}
\put(50,-16){.}
\put(50,-24){.}
\end{picture}
}
\newcommand{\upetc}{% NB this picture is offset up by 32
\begin{picture}(120,16)(-4,-32)
\put(0,0){\line(1,0){100}}
\put(0,0){\line(0,1){16}}
\put(100,0){\line(0,1){16}}
\put(50,8){.}
\put(50,16){.}
\put(50,24){.}
\end{picture}
}
\newcommand{\downbars}{% NB. This picture goes BELOW it's origin
\begin{picture}(120,16)(-4,0)
\put(0,0){\line(1,0){100}}
\put(0,0){\line(0,-1){16}}
\put(100,0){\line(0,-1){16}}
\end{picture}
}
\newcommand{\upbars}{% NB this picture is offset up by 32
\begin{picture}(120,16)(-4,-32)
\put(0,0){\line(1,0){100}}
\put(0,0){\line(0,1){16}}
\put(100,0){\line(0,1){16}}
\end{picture}
}
\newcommand{\updownbars}{% NB. This picture goes BELOW it's origin
\begin{picture}(120,16)(-4,0)
\put(0,0){\line(1,0){100}}
\put(0,0){\line(0,-1){16}}
\put(100,0){\line(0,-1){16}}
\put(0,0){\line(0,1){16}}
\put(100,0){\line(0,1){16}}
\end{picture}
}
