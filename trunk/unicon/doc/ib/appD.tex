% Make sure we don't inherit an xtabular definition from a previous chapter
\tablehead{}
\tablefirsthead{}
\tabletail{}
\tablelasttail{}

\chapter{Adding Functions and Data Types}

{\color{red} This appendix is still being updated from the Icon V6
documentation and is not consistent or correct.}

Icon is designed so that new functions and data types can be added
with comparative ease.  This appendix provides some guidelines for
modifying the Icon run-time system and lists useful macro definitions
and support routines. It is designed to be read in conjunction with
the source code for the implementation. The material included here
only touches on the possibilities. There is no substitute for actually
implementing new features and spending time studying the more
intricate parts of the Icon system.

This appendix provides an action summary of how to extend the language
implementation. For non-trivial additions, the reader should consult
Appendix G which contains a detailed description of the RTL language
in which the run-time system is written.


\section[D.1 File Organization]{D.1 File Organization}

The Icon system is organized in a hierarchy. Under UNIX, the root of
the Icon hierarchy is called icon.git/trunk {\color{blue} (the Unicon
root directory is called unicon)} and may be located anywhere.
Neither Icon nor Unicon depends on the name of the root directory and,
for other operating systems, it may be named differently.

The {\it root} directory has several subdirectories that contain
source code, test programs, documents, and so forth. The source code
is in {\it root}/src.  The subdirectories of src are shown below.

The subdirectory h holds header files that are included by files in
the other subdirectories. The file \textfn{h/rt.h} contains most of
the definitions and declarations used in the run-time system.

The rest of the code related to the run-time system is in the
subdirectory \textfn{runtime}. Source file extensions in this
directory are \texttt{.r} and \texttt{.ri} for ``run-time'' and
``run-time include'' files which are processed by rtt.
First letters of files in this
subdirectory indicate the nature of their contents. Files that begin
with the letter f contain code for functions, while files that begin
with o contain code for operators. Code related directly to the
interpretive process is in files that begin with the letter
i. ``Library'' routines for operations such as list construction that
correspond to virtual machine instructions are in files that begin
with the letter l. Finally, files that begin with the letter r hold
run-time support routines.

Within each category, routines are grouped by functionality. For
example, string construction functions such as map are in \textfn{fstr.r},
while storage allocation and garbage collection routines are in
\textfn{rmemmgt.r}.

A simplified file structure is shown below. The emphasis is on
directories that contain source or documentation. Directories that are
common to both Icon and Unicon are in black; Icon specific directories are
shown in grey; Unicon specific directories are shown in blue.
In some cases there is additional sub-structure not shown here.

\noindent\index{Icon Directory structure}\index{Unicon Directory structure}
\begin{picture}(450,370)(0,40)
%\put(0,40){\graphpaper{45}{37}}
  {\thicklines
  \put(0,205){\makebox(80,20){Icon.git/trunk}}
  \put(0,185){\color{blue}\makebox(80,20){unicon}}
  \put(0,205){\line(1,0){80}}
  \put(80,100){\line(0,1){280}}
  \put(110,370){\makebox(0,20)[l]{bin}}
  \put(150,370){\makebox(0,20)[l]
    {\parbox{100pt}{\em Executable binaries and support files}}}
  \put(110,330){\makebox(0,20)[l]{config}}
  \put(150,330){\makebox(0,20)[l]{\parbox{70pt}{\em Configuration directories}}}
  \put(110,290){\makebox(0,20)[l]{doc}}
  \put(150,290){\makebox(0,20)[l]{\parbox{100pt}{\em Documents}}}
  \put(110,250){\makebox(0,20)[l]{ipl}}
  \put(150,250){\makebox(0,20)[l]{\parbox{70pt}{\em Icon program library}}}
  {\color[rgb]{0.5,0.5,0.5}
    \put(110,210){\makebox(0,20)[l]{man}}
    \put(150,210){\makebox(0,20)[l]{\parbox{100pt}{\em Man pages}}}
  }%grey
  \put(110,170){\makebox(0,20)[l]{src}}
  \put(130,180){\line(1,0){130}}
  \put(110,130){\makebox(0,20)[l]{tests}}
  \put(150,130){\parbox[l]{80pt}
    {\em benchmarks, C~interface~tests, general~tests etc.}}
  {\color{blue}
    \put(110,90){\makebox(0,20)[l]{uni}}
    \put(150,90){\makebox(0,20)[l]{\parbox{80pt}{Unicon library}}}
  }%blue
  \multiput(80,100)(0,40){8}{\line(1,0){20}}
  %src tree
  \begin{picture}(0,0)(0,30)
  \put(260,80){\line(0,1){300}}
  \multiput(260,80)(0,30){11}{\line(1,0){20}}
  \put(290,370){\makebox(0,20)[l]{common}}
  \put(350,370){\makebox(0,20)[l]{\em common source}}
  \put(290,340){\makebox(0,20)[l]{h}}
  \put(350,340){\makebox(0,20)[l]{\em header files}}
  \put(290,310){\makebox(0,20)[l]{\color{blue}iconc}}
  \put(350,310){\makebox(0,20)[l]{\color{blue}\em Icon compiler source}}
  \put(290,280){\makebox(0,20)[l]{icont}}
  \put(350,280){\makebox(0,20)[l]{\em Icon translator source}}
  {\color{blue}
    \put(290,250){\makebox(0,20)[l]{lib}}
    \put(350,250){\makebox(0,20)[l]{
        \parbox{100pt}{\em Additional libraries for some platforms}}}
    \put(290,220){\makebox(0,20)[l]{libtp}}
    \put(350,220){\makebox(0,20)[l]{
        \parbox{120pt}{\em transfer protocol library}}}
  }%blue
  \put(290,190){\makebox(0,20)[l]{preproc}}
  \put(350,190){\makebox(0,20)[l]{\em preprocessor source}}
  \put(290,160){\makebox(0,20)[l]{rtt}}
  \put(350,160){\makebox(0,20)[l]{\em run-time translator source}}
  \put(290,130){\makebox(0,20)[l]{runtime}}
  \put(350,130){\makebox(0,20)[l]{\em run-time source}}
  {\color[rgb]{0.5,0.5,0.5}
    \put(290,100){\makebox(0,20)[l]{wincap}}
    \put(350,100){\makebox(0,20)[l]{\em Wincap library source}}
  }%grey
  \put(290,70){\makebox(0,20)[l]{xpm}}
  \put(350,70){\makebox(0,20)[l]{\em xpm library source}}
  \end{picture}% src tree
  }% thicklines
\end{picture}


\section[D.2 Adding Functions]{D.2 Adding Functions}

There are several conventions and rules of protocol that must be followed
in writing a new function. The situations that arise most frequently are
covered in the following sections. The existing functions in f files in
\textfn{runtime} provide many examples to supplement the information given
here.

\subsection[D.2.1 Function Declarations]{D.2.1 Function Declarations}

A function begins with the rtt header \texttt{function\{...\} name(args)},
where
name is the name of the function as it is called in a source-language
program. For example,
\iconline{ function\{1\} map(s1,s2,s3) }
\noindent
appears at the beginning of the function map. This header declares the
procedure block for the function and includes information about its
number of results (in curly brackets) and the number and names of its
parameters. See Appendix G.

A \texttt{Z} is prepended to the name given to avoid a collision with the
names of other C routines in the run-time system.  Thus, the C function
that implements map is named \texttt{Zmap}. Although the Icon function map
has three arguments, the corresponding C function has only one:
\texttt{r\_args}, which is a pointer to an array of descriptors on the
interpreter stack.  For example, \texttt{function {1} map(s1,s2,s3)} generates
\goodbreak
\begin{iconcode}
\ \ int Zmap(r\_args)\\
\ \ dptr r\_args;
\end{iconcode}
\noindent

A new function must also be listed via a one-line macro in
src/h/fdefs.h. For a fixed number of arguments the macro is
FncDef(name,nargs) and for a variable number of arguments
the macro is FncDefV(name).

\subsection[D.2.2 Returning from a Function]{D.2.2 Returning from a Function}

A function returns control to the interpreter by use of one of three
reserved words
\texttt{return}, \texttt{suspend}, or \texttt{fail}, depending on whether the
function returns, suspends, or fails, respectively. \texttt{return} and
\texttt{fail} return codes that the interpreter uses to differentiate between
the two situations. \texttt{suspend} returns control to the interpreter by
calling it recursively, as described in Sec. 9.3.

The use of \texttt{return} is illustrated by the following trivial function
that simply returns its argument:
\goodbreak
\begin{iconcode}
function\{1\} idem(x)\\
abstract \{ return type(x) \} \\
body \{\\
\>return x;\\
\> \} \\
end
\end{iconcode}
\noindent
For example,
\iconline{ \ \ write(idem("hello")) }
\noindent writes \texttt{hello}.

The use of \texttt{suspend} and \texttt{fail} is illustrated by the
following function, which generates its first and second arguments in
succession:
\goodbreak
\begin{iconcode}
function\{2\} gen2(x,y)\\
abstract \{ return type(x) ++ type(y) \} \\
body \{\\
\>suspend x;\\
\>suspend y;\\
\>fail;\\
\}
\end{iconcode}
\noindent
For example,
\iconline{every write(gen2("hello", "there"))}
\noindent
writes
\goodbreak
\begin{iconcode}
\ \ hello\\
\ \ there
\end{iconcode}

\noindent
As illustrated previously, \texttt{fail} is used when there is not another
result to produce. It is safe to assume that arguments
are intact when the function is resumed to produce another result.

Most functions have a fixed number of arguments. A number of
functions such as \texttt{write}, \texttt{writes},
and \texttt{stop} in the standard Icon repertoire can be called with an
arbitrary number of arguments. For a function that can be called with
an arbitrary number of arguments, the final rtt argument is given a
square brackets enclosing a number of arguments, as in
\texttt{function{1} name(x[nargs])}. When this format is used, the function is
called with two arguments: the number of arguments in the call and a
pointer to the corresponding array of descriptors. For example,
\texttt{function{1} write(x[nargs])} generates

\goodbreak
\begin{iconcode}
Zwrite(r\_nargs, r\_args)\\
int r\_nargs;\\
dptr r\_args;
\end{iconcode}

Within such a function,
the arguments are referenced using array syntax. For example, a
function that takes an arbitrary number of arguments and suspends with
them as values in succession is

\goodbreak
\begin{iconcode}
function\{*\} gen(x[n])\\
{\color{red} abstract \{ return type(x[]) \} /* probably doesn't work */ } \\
body \{\\
\>register int i;\\
\>for (i = 0; i < n; i++) \{\\
\>\>suspend x[i];\\
\>\}\\
\>fail;\\
\}
\end{iconcode}

\noindent
For example,
\iconline{
\ \ every write(gen("hello","there","!"))
}

\noindent
writes

\goodbreak
\begin{iconcode}
\ \ hello\\
\ \ there\\
\ \ !
\end{iconcode}

\noindent
Note the use of \texttt{fail} at the end of the function; the omission of
\texttt{fail} would be an error, since returning by flowing off the end of the
function would not provide the return code that the interpreter expects.

\subsection[D.2.3 Type Checking and Conversion]{D.2.3 Type Checking and Conversion}

Some functions need to perform different operations, depending on the
types of their arguments. An example is \texttt{type(x)}:

\goodbreak
\begin{iconcode}
function{1} type(x)\\
abstract \{ return string \} \\
body \{\\
\>type\_case x of  \{\\
\>\>      string:   inline \{ return C\_string "string";    \} \\
\>\>      null:     inline \{ return C\_string "null";      \} \\
\>\>      integer:  inline \{ return C\_string "integer";   \} \\
\>\>      real:     inline \{ return C\_string "real";      \} \\
\>\>      cset:     inline \{ return C\_string "cset";      \} \\
\>\>      \vdots \\
\>\} \\
\end{iconcode}

\noindent
Icon values are stored in descriptors, wherein the d-word serves to
differentiate between types. Strings require a separate test prior
to examining the d-word's type code, but this is abstracted by the
RTL \texttt{type\_case} construct.

For most functions, arguments must be of a specific type. As described in
Sec. 12.1, type conversion routines are used for this purpose. For example,
the function \texttt{tab(i)} requires that \texttt{i} be an integer. It
begins as follows:

\goodbreak
\begin{iconcode}
function\{0,1+\} tab(i)\\
\>   if !def:C\_integer(i, 0) then runerr(101, i); \\
\>   abstract \{ return string \} \\
body \{\\
\> \vdots
\end{iconcode}

\noindent
Note that \texttt{cvint} is called with the addresses of \texttt{Arg1} and
\texttt{I1}. If the conversion is successful, the resulting integer is assigned
to \texttt{I1}. As indicated by this example, it is the responsibility of a
function to terminate execution by calling \texttt{runerr} if a required
conversion cannot be made.

The routine \texttt{cvstr}, which converts values to strings, requires a buffer,
which is supplied by the routine that calls it.  See Sec. 4.4.4. This buffer
must be large enough to hold the longest string that can be produced by the
conversion of any value. This size is given by the defined constant
\texttt{MaxCvtLen}. For example, the function to reverse a string begins as
follows:

\goodbreak
\begin{iconcode}
\color{red}FncDcl(reverse,1)\\
\{\\
\>register char c. *floc. *lloc;\\
\>register word slen;\\
\>char sbuf[MaxCvtLen];\\
\>extern char *alcstr();\\
\>/*\\
\>\ * Make sure that Arg1 is a string.\\
\>\ */\\
\>if (cvstr(\&Arg1. sbuf) == CvtFail)\\
\>\>runerr(103. \&Arg1);
\end{iconcode}

\noindent
The buffer is used only if a nonstring value is converted to a string. In this
case, \texttt{Arg1} is changed to a qualifier whose v-word points to the
converted string in \texttt{sbuf}. This string does not necessarily begin at the
beginning of \texttt{sbuf}. In any event, after a successful call to
\texttt{cvstr}, the argument is an appropriate qualifier, regardless of whether
a conversion actually was performed.

\subsection[D.2.4 Constructing New Descriptors]{D.2.4 Constructing New Descriptors}

Some functions need to construct new descriptors to return in \texttt{Arg0}.
Sometimes it is convenient to construct a descriptor by assignment to its d- and
v-words. Various macros are provided to simplify these assignments. As given in
the function \texttt{type} previously, \texttt{StrLen} and \texttt{StrLoc} can
be used to construct a qualifier. For example, to return a qualifier for the
string \texttt{"integer"}, the following code suffices:
\goodbreak
\begin{iconcode}
StrLen(Arg0) = 7;\\
StrLoc(Arg0) = "integer";\\
Return;
\end{iconcode}
\noindent
Here, the returned qualifier points to a statically allocated C string.

There also are macros and support routines for constructing certain
kinds of descriptors. For example, the macro

\iconline{\color{red} Mkint(i, dp); }

\noindent
constructs an integer descriptor containing the integer \texttt{i} in the
descriptor pointed to by \texttt{dp}. The definition of \texttt{\color{red}Mkint}
depends on the word size of the computer. On 32-bit computers,
\texttt{\color{red}Mkint} simply produces assignments to the d-word and v-word of
descriptor pointed to by \texttt{dp}. On computers with 16-bit words, which
have both \texttt{T\_Integer} and \texttt{T\_Long} forms of integers,
\texttt{\color{red}Mkint} produces a call to a support routine.

\subsection[D.2.5 Default Values]{D.2.5 Default Values}

Many functions specify default values for null-valued arguments. There
are support routines for providing default values. For example,

\iconline{ \color{red}defstr(Arg3, sbuf, \&q); }


\noindent
changes \texttt{Arg3} to the string given by the qualifier \texttt{q} in
case \texttt{Arg3} is null-valued. If \texttt{Arg3} is not null-valued,
however, its value is converted to a string, if possible, by
\texttt{\color{red}defstr}. If this is not possible, \texttt{\color{red}defstr} terminates
execution with an error message.

\subsection[D.2.6 Storage Allocation]{D.2.6 Storage Allocation}

Functions that construct new data objects often need to allocate
storage. Allocation is done in the allocated string region or the
allocated block region, depending on the nature of the object. Support
routines are provided to perform the actual allocation.

As mentioned in Sec. 11.4, predictive need requests \textit{must} be made
before storage is actually allocated. The functions \texttt{strreq(i)} and
\texttt{blkreq(i)} request \texttt{i} bytes of storage in the allocated
string and block regions, respectively.

Such a request generally should be made as soon as an upper bound on the
amount of storage needed is known. It is not necessary to know the exact
amount, but the amount requested must be at least as large as the amount
that actually will be allocated. For example, the function
\texttt{reads(f. i)} requests \texttt{i} bytes of string storage, although
the string actually read may be shorter.

\textbf{String Allocation}. The function \texttt{alcstr(s, i)} copies
\texttt{i} bytes starting at \texttt{s} into the allocated string region
and returns a pointer to the beginning of the copy. For example, a function
\texttt{double(s)} that produces the concatenation of \texttt{s} with
itself is written as follows:
\goodbreak
\begin{iconcode}
\color{red}FncDcl(double. 1)\\
\{\\
\>register int slen;\\
\>char sbuf[MaxCvtLen];\\
\>extern char *alcstr();\\
\>if (cvstr(\&Arg1, sbuf) == NULL)\\
\>\>runerr(103, \&Arg1);\\
\>slen = StrLen(Arg1);\\
\>strreq(2 * slen);\\
\>StrLen(Arg0) = 2 * slen;\\
\>StrLoc(Arg0) = alcstr(StrLoc(Arg1). slen);\\
\>alcstr(StrLoc(Arg1), slen);\\
\>Return;\\
\}
\end{iconcode}

If the first argument of \texttt{alcstr} is \texttt{NULL}, instead of being
a pointer to a string, the space is allocated and a pointer to the
beginning of it is returned, but nothing is copied into the space. This
allows a function to construct a string directly in the allocated string
region.

If a string to be returned is in a buffer as a result of conversion
from another type, care must be taken to copy this string into the
allocated string region --- otherwise the string in the buffer will
be overwritten on subsequent calls. Copying such strings is
illustrated by the function \texttt{string(x)} given in Sec. 12.1.

\textbf{Block Allocation}. The routine \texttt{alcblk(i)} allocates
\texttt{i} bytes in the allocated block region and returns a pointer to the
beginning of the block. The argument of \texttt{alcblk} must correspond to
a whole number of words. There are run-time support routines for allocating
various kinds of blocks. These routines, in turn, call \texttt{alcblk}.
Such support routines generally fill in part of the block as well. For
example, \texttt{alccset(i)} allocates a block for a cset, fills in the
title and size words, and zeroes the bits for the cset:
\goodbreak
\begin{iconcode}
struct b\_cset *alccset(size)\\
int size;\\
\{\\
\>register struct b\_cset *blk;\\
\>register i;\\
\>extern union block *alcblk();\\
\>blk = (struct b\_cset *)alcblk((word)sizeof(struct b\_cset), T\_Cset);\\
\>blk->size = size;\\
\\
\>/*\\
\>\ * Zero the bit array.\\
\>\ */\\
\>for (i = 0; i < CsetSize; i++)\\
\>\>blk->bits[i] = 0; return blk;\\
\}
\end{iconcode}

\noindent
See Sec. D.5.5 for a complete list of block-allocation functions.

\subsection[D.2.7 Storage Management Considerations]%
           {D.2.7 Storage Management Considerations}

In addition to assuring that predictive need requests are made before
storage is allocated, it is essential to assure that all descriptors
contain valid data at any time a garbage collection may occur, that
all descriptors are accessible to the garbage collector, and that all
pointers to allocated data are in the v-words of descriptors.

Normally, all the descriptors that a function uses are on the interpreter
stack and are referenced as \texttt{Arg0}, \texttt{Arg1}, \ldots.  Such
descriptors are processed by the garbage collector. Occasionally,
additional descriptors are needed for intermediate computations. If such
descriptors contain pointers in their v-words, it is \textit{not} correct
to declare local descriptors, as in
\goodbreak
\begin{iconcode}
\color{red}FncDcl(mesh,2)\\
\{\\
\>struct descrip d1, d2;
\end{iconcode}

\noindent
The problem with this approach is that \texttt{d1} and \texttt{d2} are on
the C stack and the garbage collector has no way of knowing about them.

However, since all descriptors on the interpreter stack are accessible
to the garbage collector, intermediate computations can be performed
on descriptors on the interpreter stack. Extra descriptors for this
purpose can be provided by increasing the number of arguments
specified for the function. Thus,

\iconline{\color{red}FncDcl(mesh,4) }

\noindent
makes \texttt{Arg3} and \texttt{Arg4} available for intermediate
computations. The initial values of \texttt{Arg3} and \texttt{Arg4} will be
null because of argument adjustment performed by invoke unless mesh is
called with extra arguments.

Garbage collection can occur only during a predictive need
request. However, a predictive need request can occur between the time
a function suspends and the time it is resumed to produce another
result. Consequently, if a pointer is kept in a C variable in a loop
that is producing results by suspending, the pointer may be invalid
when the function is resumed. Instead, the pointer should be kept in
the v-word of a descriptor that is accessible to the garbage
collector.

\subsection[D.2.8 Error Termination]{D.2.8 Error Termination}

An Icon program may terminate abnormally for two reasons: as the
result of a source-language programming error (such as an invalid type
in a function call), or as a result of an error detected in the Icon
system itself (such as a descriptor that should have been dereferenced
but was not).

In case a source-language error is detected, execution is terminated
by a call of the form

\iconline{runerr(i, \&d); }

\noindent where \texttt{i} is an error message number and \texttt{d} is the
descriptor for the offending value. If there is no specific offending
value, the second argument is 0.

The array of error message numbers and corresponding messages is
contained in \textfn{runtime/data.r}. If there is no appropriate existing error
message, a new one can be added, following the guidelines given in
Appendix D of Griswold and Griswold 1983.

In theory, there should be no errors in the Icon system itself, but no
large, complex software system is totally free of errors. Some situations
are recognizable as being potential sources of problems in case data does
not have the expected values. In such situations, especially during program
development, it is advisable to insert calls of the function
\texttt{syserr}, which terminates execution, indicating that an error was
detected in the Icon system, and prints its argument as an indication of
the nature of the error. It is traditional to use calls of the form

\iconline{syserr("mesh: can't happen"); }

\noindent so that when, in fact, the ``impossible'' does happen, there is a
reminder of human frailty.  More informative messages are desirable, of
course.

\subsection[D.2.9 Header Files]{D.2.9 Header Files}

If a new function is added to an existing f file in \textfn{runtime}, the
necessary header files normally will be included automatically. If a
new function is placed in a new file, that file should begin with

\iconline{\#include "../h/rt.h" }

% The redefinition of arraystretch works as a way of reducing inter-row
% spacing in the table, but maybe we should find a more elegant solution.
{\renewcommand{\arraystretch}{0.9}%
\noindent
This header file includes ten other header files:\\[1ex]

\noindent\hspace{0.5cm}% Indent the entire table
% The magic spell ">{\textfn\bgroup} ... <{\egroup}" sets the font
% for the first column.  Just showing off for a three row table,
% but it will come in handy later on in the chapter.
\begin{xtabular}{>{\textfn\bgroup}l<{\egroup}@{\hspace{1cm}}p{11cm}}
../h/define.h    & platform specific definitions\\
../h/arch.h      & (Icon only) automatically generated word size definitions\\
../h/config.h    & general configuration information\\ 
../h/sys.h	     & system include files (\textfn{stdio.h} etc.)\\ 
../h/typedefs.h  & type definitions for the run-time system\\
../h/cstructs.h  & structure definitions and constants for the
                   functions in the \textfn{src/common} directory\\
%
../h/mproto.h	 & (Icon only) prototypes for functions common to several modules\\
%
{\color{blue}../h/proto.h} &%
% Without the \hspace{0cm} we get extra _vertical_ space between columns: go figure.
% It only happens when the \color{blue} is added.
    {\hspace{0cm}\color{blue}(Unicon only) prototypes for library functions}\\
%
../h/cpuconf.h   & definitions that depend on the computer word size\\
{\color{blue}../h/monitor.h} &%
    {\hspace{0cm}\color{blue}  (Unicon only) event code definitions}\\
../h/rmacros.h   & macros and manifest constants\\ 
../h/rstructs.h  & runtime data structures\\
\end{xtabular}
}\\[1ex]
% arraystretch is now back to normal

\noindent
All of these files contain appropriate information for the local
installation, and no changes in them should be needed.


In rare cases, it may be necessary to include other header files.
% [DonW] gc.h does not exist anymore. After looking at the Icon V6.7 gc.h,
% most of the definitions have been absorbed into rmacros.h. If we can think of
% another example then I suggest we reinstate the next sentence with it.
%
%  For example, a function that deals directly with garbage collection might
%  need to include \textfn{iconx/gc.h}.

\subsection[D.2.10 Installing a New Function]{D.2.10 Installing a New Function}

Both the linker and the run-time system must know the names of all
functions. This information is provided in the header file \textfn{h/fdefs.h}.


In order to add a function, a line of the form
\iconline{FncDef(name) }

\noindent must be inserted in \textfn{h/fdefs.h} in proper alphabetical order.

Once this insertion is made, the Icon system must be recompiled to
take into account the code for the new function. The steps involved in
recompilation vary from system to system. Information concerning
recompilation is available in system-specific installation documents.

\section[D.3 Adding Data Types]{D.3 Adding Data Types}

Adding a new data type is comparatively simple, although there are
several places where changes need to be made. Failure to make all the
required changes can produce mysterious bugs.

\subsection[D.3.1 Type Codes]{D.3.1 Type Codes}

At present, type codes range from 0 to 25 {\color{blue} (31 for
Unicon)}. Every type must have a distinct type code and corresponding
definitions.  These additions are made in \textfn{h/rmacros.h}. First, a
\texttt{T\_}definition is needed. For example, if a Boolean type is added,
a definition such as

\iconline{\#define T\_Boolean\ \ \ 26 }

\noindent is needed. The value of \texttt{MaxType}, which immediately follows
the type code definitions, must be increased to 26 accordingly.
Failure to set \texttt{MaxType} to the maximum type code may result in
program malfunction during garbage collection. See Sec. 11.3.2.

Next a \texttt{D\_} definition is needed for the d-word of the new type.
For a Boolean type, this definition might be

\iconline{\#define D\_Boolean\ \ (T\_Boolean | F\_Nqual) }

\noindent
All nonstring types have the \texttt{F\_Nqual} flag and their \texttt{T\_}
type code. Types whose v-words contain pointers also have the
\texttt{F\_Ptr} flag.

\subsection[D.3.2 Structures]{D.3.2 Structures}

A value of a Boolean type such as the one suggested previously can be
stored in the d-word of its descriptor. However, most types contain
pointers to blocks in their v-words. In this case, a declaration of a
structure corresponding to the block must be added to \textfn{h/rstructs.h}.
For example, a new rational number data type, with the type code
\texttt{T\_Rational}, might be represented by a block containing two
descriptors, one for the numerator and one for the denominator. An
appropriate structure declaration for such a block is

\goodbreak
\begin{iconcode}
struct b\_rational \{\\
\>int title;\\
\>struct descrip numerator;\\
\>struct descrip denominator;\\
\};
\end{iconcode}

Since rational blocks are fixed in size, no size field is needed. However,
a vector type with code \texttt{T\_Vector} in which different vectors have
different lengths needs a size field. The declaration for such a block
might be

\goodbreak
\begin{iconcode}
struct b\_vector \{\\
\>int title;\\
\>int blksize;\\
\>struct descrip velems[1];\\
\};
\end{iconcode}

\noindent
As mentioned in Sec. 4.4.2, the size of one for the array of
descriptors is needed to avoid problems with C compilers.  In
practice, this structure conceptually overlays the allocated block
region, and the number of elements varies from block to block.

Any new structure declaration for a block must be added to the declaration
union \texttt{block} in \textfn{h/rstructs.h}. The order of the fields in a
new structure must follow the rules (discussed next) imposed by the garbage
collector.

\subsection[D.3.3 Information Needed for Storage Management]%
           {D.3.3 Information Needed for Storage Management}

In Icon version 8 and later, allocated data may be referred to in two ways:
either (as before) in the v-word of a descriptor, or in one of the pointers
in a structure.  All pointers to allocated data must be contained either in
the v-words of descriptors or in pointers, since this is the only way the
garbage collector can locate them. Furthermore, all non-descriptor data
(including pointers) must precede any descriptors in a block. The amount of
non-descriptor data, and hence the location of the first descriptor in a
block, must be the same for all blocks of a given type. Pointers add new
constraints: If any pointers are present, they must be in one contiguous
group and must precede the first descriptor in the block although they need
not be placed immediately before the descriptors.  If any block has a
variable number of pointers they must be placed at the end of the block
(and there can be no descriptors in the block).

As described in Sec. 11.3.2, the garbage collector uses the array
\texttt{bsizes} to determine the size of a block and the array
\texttt{firstd} to determine the offset of the first descriptor in the
block. These arrays are in \textfn{rmemmgt.r}. When a new data type is
added, appropriate entries must be made in these arrays. Failure to do so
may result in serious bugs that occur only in programs that perform garbage
collection, and the symptoms may be mysterious.

There is an entry in \texttt{bsizes} for each type code. If the type has no
block, the entry is -1. If the type has a block of constant size, the entry
is the size of the block. Otherwise, the entry is 0, indicating that the
size is in the second word of the block. Thus, the entry for
\texttt{T\_Boolean} would be -1, the entry for \texttt{T\_Rational} would
be \texttt{sizeof(struct b\_rational)}, and the size for \texttt{T\_Vector}
would be 0.

There is a corresponding entry in \texttt{firstd} for each type code that
gives the offset of the first descriptor in its corresponding block. If
there is no block, the entry is -1. If the block contains no descriptors,
the entry is 0. For example, the entry for \texttt{T\_Boolean} would be -1,
the entry for \texttt{T\_Rational} would be \texttt{WordSize}, and the
entry for \texttt{T\_Vector} would be \texttt{2*WordSize}, where
\texttt{WordSize} is a defined constant that is the number of bytes in a
word.

Two further arrays are used to determine the placement of pointers.
\texttt{firstp} is analogous to \texttt{firstd}: there is one entry per
type. If there is no block, the entry is -1. If the type has no pointers
the value is 0, otherwise it is the offset of the first pointer in the
block.  It is possible to infer the number of descriptors from the position
of the first and the size of the block, but this inference is not always
possible for pointers so an array \texttt{ptrno} is used to indicate how
many pointers there are for each type. If there are no pointers, the value
is -1. A value of 0 means that the pointers occupy the rest of the block
(and the block has no descriptors). Otherwise the entry contains the number
of pointers in the block. As with \texttt{firstd} and \texttt{bsizes}, a
failure to initialize \texttt{firstp} and \texttt{ptrno} correctly may
result in serious (and mysterious) bugs.

A fifth array, \texttt{blknames}, provides string names for all block
types. These names are only used for debugging, and an entry should be
made in \texttt{blknames} for each new data type.

\subsection[D.3.4 Changes to Existing Code]{D.3.4 Changes to Existing Code}

In addition to any functions that may be needed for operating on
values of a new data type, there are several functions and operators
that apply to all data types and which may, therefore, need to be
changed for any new data type.

\noindent
These are\\[1ex]
{\renewcommand{\arraystretch}{0.9}%
\begin{xtabular}{>{\hspace{1cm}\texttt\bgroup}l<{\egroup}@{\hspace{1cm}}p{11cm}}
 *x      & size of x (in \textfn{runtime/omisc.r})\\
copy(x)  & copy of x (in \textfn{runtime/fmisc.r})\\
image(x) & string image of x (in \textfn{runtime/fmisc.r})\\
type(x)  & string name of type of x (in \textfn{runtime/fmisc.r})\\
\end{xtabular}
}\\[1ex]

There is not a concept of size for all data types. For example, a
Boolean value presumably does not have a size, but the size of a
vector presumably is the number of elements it contains. The size of a
rational number is problematical.  Modifications to \texttt{*x} are easy; see
Sec. 4.4.4.

There must be some provision for copying any value. For structures,
such as vectors, physical copies should be made so that they are
treated consonantly with other Icon structures. For other data types,
the ``copy'' consists of simply returning
the value and not making a physically distinct copy.  This should be
done for data types, such as Boolean, for which there are only
descriptors and no associated blocks.  Whether or not a copy of a
block for a rational value should be made is a more difficult decision
and depends on how such values are treated conceptually, at the
source-language level. It is, of course, easiest not to make a
physical copy.

Some image must be provided for every value. This image should contain
enough information to distinguish values of different types and, where
possible, to provide some useful additional information about the
specific value. The amount of detail that it is practical to provide
in the image of a value is limited by the fact that the image is a
string that must be placed in the allocated string region.


The type must be provided for all values and should consist of a simple
string name. For example, if \texttt{x} is a Boolean value, \texttt{type(x)}
should produce \texttt{"boolean"}. The coding for type is trivial; see
Sec. D.2.3.

There also are several run-time support routines that must be modified
for any new type:\\[1ex]
{\renewcommand{\arraystretch}{0.9}%
\begin{xtabular}{>{\hspace{1cm}\texttt\bgroup}l<{\egroup}@{\hspace{1cm}}p{11cm}}
outimage & image for tracing (in \textfn{runtime/rmisc.r)}\\
order    & order for sorting (in \textfn{runtime/rcomp.r)}\\
anycmp   & comparison for sorting (in \textfn{runtime/rcomp.r)}\\
equiv    & equivalence comparison (in \textfn{runtime/rcomp.r)}\\
\end{xtabular}
}\\[1ex]

The image produced for tracing purposes is similar to that produced by
image and must be provided for all data types.  However, outimage
produces output and is not restricted to constructing a string in
allocated storage. It therefore can be more elaborate and informative.

There must be some concept of sorting order for every Icon
value. There are two aspects to sorting: the relative order of
different data types and the ordering among values of the same
type. The routine order produces an integer that corresponds to the
order of the type. If the order of a type is important with respect to
other types, this matter must be given some consideration. For
example, a rational number probably belongs among the numeric types,
which, in Icon, sort before structure types. On the other hand, it
probably is not important whether vectors come before or after lists.


The routine \texttt{anycmp} compares two values; if they have the same
order, as defined previously, \texttt{anycmp} determines which is the
``smaller.''  For example, Boolean ``false'' might (or might not) come
before ``true,'' but some ordering between the two should be provided.
On the other hand, order among vectors probably is not important (or
well-defined), and they can be lumped with the other structures in
\texttt{anycmp}, for which ordering is arbitrary. Sometimes ordering can be
quite complicated; a correct ordering of rational numbers is nontrivial.


The routine \texttt{equiv} is used in situations, such as table
subscripting and case expressions, to determine whether two values are
equivalent in the Icon sense. Generally speaking, two structure values are
considered to be equivalent if and only if they are identical. This
comparison is included in \texttt{equiv} in a general way. For example,
\texttt{equiv} need not be modified for vectors. Similarly, for data types
that have no corresponding blocks, descriptor comparison suffices;
\texttt{equiv} need not be modified for Boolean values either. However,
determining the equivalence of numeric values, such as rational numbers,
requires some thought.

Although not strictly part of the language proper, there are several
routines in the Icon Programming Library and the Unicon class libraries
that operate on all types. The addition of new type may require that some
of these be modified. A good start to locating code that potentially
requires modification is to search for expressions of the form
\iconline{case \ldots type( \ldots ) \ldots of}

\section[D.4 Defined Constants and Macros]{D.4 Defined Constants and Macros}

Defined constants and macros are used heavily in Icon to parameterize
its code for different operating systems and computer architectures
and to provide simple, high-level constructions for commonly occurring
code sequences that otherwise would be complex and obscure.

These defined constants and macros should be used consistently when
making additions to Icon instead of using \textit{ad hoc}
constructions. This improves portability, readability, and
consistency.

Learning the meanings and appropriate use of the existing defined
constants and macro definitions requires investment of time and
energy. Once learned, however, coding is faster, simpler, and less
prone to error.

\subsection[D.4.1 Defined Constants]{D.4.1 Defined Constants}

The following defined constants are used frequently in the run-time
system. This list is by no means exhaustive; for specialized
constants, see existing functions.\\[1ex]

{\renewcommand{\arraystretch}{0.9}%
\begin{xtabular}{>{\hspace{1cm}\texttt\bgroup}l<{\egroup}@{\hspace{1cm}}p{11cm}}
CsetSize     & number of words needed for 256 bits\\
LogHuge      & one plus the maximum base-lO exponent of a C \texttt{double}\\
LoglntSize   & base-2 logarithm of number of bits in a C \texttt{int}\\
MaxCvtLen    & length of the longest possible string obtained by conversion\\
MaxLong      & largest C \texttt{long}\\
MaxShort     & largest C \texttt{short}\\
MaxStrLen    & longest possible string\\
MinListSlots & minimum number of slots in a list-element block\\
MinLong      & smallest C \texttt{long}\\
MinShort     & smallest C \texttt{short}\\
WordSize     & number of bytes in a \texttt{word}\\
\end{xtabular}
}\\[1ex]

\subsection[D.4.2 Macros]{D.4.2 Macros}

The following macros are used frequently in the run-time system.  See
\textfn{h/rmacros.h} for most of the definitions and the files included by
\textfn{h/rt.h} for the others; see existing routines for examples of
usages.\\[0.5cm]
{\renewcommand{\arraystretch}{0.9}%
\begin{xtabular}{>{\hspace{1cm}\texttt\bgroup}l<{\egroup}@{\hspace{1cm}}p{11cm}}
Arg(n)            & \hspace{0cm}\texttt{n}th argument to function\\
ArgType(n)        & d-word of \texttt{n}th argument to function\\
ArgVal(n)         & integer value of v-word of nth argument to function\\
BlkLoc(d)         & pointer to block from v-word of \texttt{d}\\
BlkSize(cp)       & size of block pointed to by \texttt{cp}\\
BlkType(cp)       & type code of block pointed to by \texttt{cp}\\
ChkNull(d)        & true if \texttt{d} is a null-valued descriptor\\
CsetOff(b)        & offset in a word of cset bit \texttt{b}\\
CsetPtr(b, c)     & address of word \texttt{c} containing cset bit \texttt{b}\\
DeRef( d)         & dereference \texttt{d}\\
EqlDesc(d1, d2)   & true if \texttt{d1} and \texttt{d2} are identical descriptors\\
GetReal(dp, r)    & get real number into \texttt{r} from
                    descriptor pointed to by \texttt{dp}\\
IntVal(d)         & integer value of v-word of \texttt{d}\\
Max(i, j)         & maximum of \texttt{i} and \texttt{j}\\
Min(i. j)         & minimum of \texttt{i} and \texttt{j}\\
Mkint(i, dp)      & make integer from \texttt{i} in descriptor
                    pointed to by \texttt{dp}\\
Offset(d)         & offset from d-word of variable descriptor \texttt{d}\\
Pointer(d)        & true if v-word of \texttt{d} is a pointer\\
Qual(d)           & true if \texttt{d} is a qualifier\\
Setb(b, c)        & set bit \texttt{b} in cset \texttt{c}\\
SlotNum(i, j)     & Slot for hash number \texttt{i} given \texttt{j} total slots\\
StrLen(q)         & length of string referenced by \texttt{q}\\
StrLoc(q)         & location of string referenced by \texttt{q}\\
Testb(b, c)       & true if bit \texttt{b} in cset \texttt{c} is one\\
Tvar(d)           & true if \texttt{d} is a trapped variable\\
TvarLoc(d)        & pointer to trapped variable from v-word of \texttt{d}\\
Type(d)           & type code in d-word of \texttt{d}\\
Var(d)            & true if \texttt{d} is a variable descriptor\\
VarLoc(d)         & pointer to value descriptor from v-word of \texttt{d}\\
Vsizeof(x)        & size of structure \texttt{x} less variable array at end\\
Vwsizeof(x)       & size of structure \texttt{x} in words less variable array at end\\
Wsizeof(x)        & size of structure \texttt{x} in words\\
\end{xtabular}
}\\[1ex]

\section[D.5 Support Routines]{D.5 Support Routines}

There are many support routines for performing tasks that occur frequently in
the Icon run-time system. Most of these routines are in files in \textfn{runtime}
that begin with the letter r. The uses of many of these support routines have
been illustrated earlier; what follows is a catalog for reference.

\subsection[D.5.1 Comparison]{D.5.1 Comparison}

The following routines in \textfn{runtime/rcomp.r} perform comparisons:

\begin{xtabular}{l@{\hspace{1cm}}p{11cm}}

\texttt{anycmp(dp1, dp2)} &
Compare the descriptors pointed to by \texttt{dp1} and \texttt{dp2} as Icon
values in sorting order, returning a value greater than 0, 0, or less than 0
depending on whether the descriptor pointed to by \texttt{dp1} is respectively greater
than, equal to, or less than the descriptor pointed to by \texttt{dp2}.\\

\texttt{equiv(dp1, dp2)} &
Test for equivalence of descriptors pointed to by \texttt{dp1} and \texttt{dp2},
returning 1 if equivalent and 0 otherwise\\

\texttt{lexcmp(dp1,dp2)} &
Compare string qualifiers pointed to by \texttt{dp1} and \texttt{dp2} returning
a value greater than 0, 0 or less than 0 depending on whether the string
referenced by \texttt{dp1} is respectively greater than, equal to or less than
the string referenced by \texttt{dp2}\\

\texttt{numcmp(dp1, dp2, dp3)} &
Compare the descriptors pointed to by \texttt{dp1} and \texttt{dp2} as numbers,
putting the converted value of the number referenced by \texttt{dp2} in the
descriptor pointed to by \texttt{dp3} and returning 0, 0 or less than 0
depending on whether the number referenced by \texttt{dp1} is respectively
greater than, equal to or less than the number referenced by \texttt{dp2}\\

\end{xtabular}

\subsection[D.5.2 Type Conversion]{D.5.2 Type Conversion}

The following routines in \textfn{\color{red}iconx/rconv.c} perform type conversions:

\begin{xtabular}{l@{\hspace{1cm}}p{11cm}}

\texttt{\color{red}cvcset(dp, csp, sbuf)} &
Convert the descriptor pointed to by \texttt{dp} to a cset and point
\texttt{csp} to it, using \texttt{sbuf} as a conversion buffer if necessary.\\

\texttt{\color{red}cvint(dp,ip)} &
Convert the descriptor pointed to by \texttt{dp} to an integer and store the
value in the location pointed to by \texttt{ip}, returning the type if the
conversion can be performed, but \texttt{NULL} otherwise.\\

\texttt{\color{red}cvnum(dp,xp)} &
Convert the descriptor pointed to by \texttt{dp} to a numeric value and store
the value in the location pointed to by \texttt{xp}, returning the type if the
conversion can be performed, but \texttt{NULL} otherwise.\\

\texttt{cvpos(i1, i2)} &
Convert \texttt{i1} to a positive value with respect to the length \texttt{i2},
returning 0 if the conversion is not possible.\\

\texttt{\color{red}cvreal(dp,rp)} &
Convert the descriptor pointed to by \texttt{dp} to a real number and store
the value in the location pointed to by \texttt{rp}, returning the type if the
conversion can be performed, but \texttt{NULL} otherwise.\\

\texttt{\color{red}cvstr(dp, sbuf)} &
Convert the descriptor pointed to \texttt{dp} to a string, using \texttt{sbuf}
as a buffer if necessary, returning \texttt{Cvt} if a conversion was performed,
\texttt{NoCvt} if a conversion was unneccessary, or \texttt{NULL} if the
conversion cannot be performed.\\

\texttt{\color{red}gcvt(n, i, sbuf)} &
Convert the number \texttt{n}to a string in \texttt{sbuf}, producing \texttt{i}
significant digits if possible, otherwise using exponent notation.\\

\texttt{\color{red}mkreal(r, dp)} &
Make a real number descriptor for \texttt{r} in the descriptor pointed to by
\texttt{dp}.\\

\texttt{strprc(dp, i)} &
Convert the qualifier pointed to by \texttt{dp} to a procedure descriptor if
possible, using \texttt{i} as the number of arguments in the case of a string
that represents an operator, returning 0 if the conversion cannot be performed.\\

\end{xtabular}

\subsection[D.5.3 Defaults]{D.5.3 Defaults}

The following routines in \textfn{\color{red}iconx/default.c} produce default values for
omitted arguments:

\begin{xtabular}{l@{\hspace{1cm}}p{11cm}}

\texttt{\color{red}defcset(dp, csp, sbuf, ip)} &
If the descriptor pointed to by \texttt{dp} is null, store the cset pointed at
by \texttt{ip} at the place pointed to by \texttt{csp} and return 1; otherwise
convert the descripro pointed to by \texttt{dp} and return 0, but terminate with
Error 104 if the conversion cannot be performed.\\

\texttt{\color{red}deffile(dp1, dp2)} &
If the descriptor pointed to by \texttt{dp} is null, replace it by the
descriptor pointed to by \texttt{dp2} and return 1; otherwise return 0 if
\texttt{dp1} points to a file descriptor but terminate with Error 105
otherwise.\\

\texttt{\color{red}defint(dp, ip, i)} &
If the descriptor pointed to by \texttt{dp} is null, store \texttt{i} at the
location pointed to by\texttt{ip} and return 1. Otherwise convert the descriptor
pointed to by \texttt{dp} to an integer, store it at the location pointed to by
\texttt{ip}, and return 0, but terminate with Error 101 if the conversion cannot
be performed.\\

\texttt{\color{red}defshort(dp,i)} &
If the descriptor pointed to by \texttt{dp} is null, convert it to \texttt{i}
and return 1. Otherwise convert the descriptor pointed to by \texttt{dp} to a
short integer and return 0, but terminate with Error 101 if the conversion cannot
be performed or Error 205 if the integer is too long.\\

\texttt{\color{red}defstr(dp1, sbuf, dp2)} &
If the descriptor pointed to by \texttt{dp1} is null,replace it by the
descriptor pointed to by \texttt{dp2} and return 0. Otherwise convert the
descriptor pointed by \texttt{dp1} to a string and return 0, but terminate with
Error 103 if the conversion cannot be performed.\\

\end{xtabular}


\subsection[D.5.4 Assignment]{D.5.4 Assignment}

The following routine in \textfn{\color{red}iconx/doasgn.c} is used for all source language
assignments:

\begin{xtabular}{l@{\hspace{1cm}}p{11cm}}
\texttt{\color{red}doasgn(dp1,dp2)} &
Assign the descriptor pointed to by \texttt{dp2} to the location referenced by
the variable descriptor pointed to by \texttt{dp1}.\\

\end{xtabular}

\subsection[D.5.4 Allocation]{D.5.4 Allocation}

The following routines in \textfn{runtime/ralc.r} all return pointers to the
objects they allocate:

{\renewcommand{\arraystretch}{1}%
\begin{xtabular}{>{\hspace{1cm}\texttt\bgroup}l<{\egroup}@{\hspace{1cm}}p{11cm}}

alcblk(i,j) &
Allocate a block of \texttt{i} bytes with title \texttt{j} in the allocated
block region.\\

alccoexp() &
Allocate a co-expression block.\\

alccset(i) &
Allocate a cset block for a cset, setting its size to \texttt{i}.\\

alcfile(fp, i, dp) &
Allocate a file block. setting its file pointer to \texttt{fp}, its status to
i, and its name to the qualifier pointed to by \texttt{dp}.\\

alclint(i) &
Allocate a long-integer block and place \texttt{i} in it.\\

alclist(i) &
Allocate a list-header block and set its size field to \texttt{i}.\\

alclstb(i1, i2, i3) &
Allocate a list element block for \texttt{i1} elements,setting its first field
to \texttt{i2} and its \texttt{nused} field to \texttt{i3}.\\

alcreal(r) &
Allocate a real-number block and place \texttt{r} in it.\\

alcrecd(i,dp) &
Allocate a record block with \texttt{i} fields setting its procedure descriptor
to the descriptor pointed to by \texttt{dp}.\\

alcrefresh(ip, i, j) &
Allocate a refresh block for a procedure with \texttt{i} arguments, \texttt{j}
local identifiers, and entry point \texttt{ip}.\\

alcselem(dp, i) &
Allocate a set-element block, setting its \texttt{member} field to the descriptor
pointed to by \texttt{dp} and its hash number field to \texttt{i}.\\

alcset() &
Allocate a set-header block.\\

alcstr(sbuf,i) &
Allocate a string of length \texttt{i}, and copy the string in \texttt{sbuf} into
it, provided \texttt{sbuf} is not \texttt{NULL}.\\

alcsubs(i, j, dp) &
Allocate a substring trapped-variable block, setting its length field to
i, its offset field to \texttt{j}, and its variable decriptor to the
descriptor pointed to by \texttt{dp}.\\

alctable(dp) &
Allocate a table-header block, setting its default descriptor to the descriptor
pointed to by \texttt{dp}.\\

alctelem() &
Allocate a table-element block.\\

alctvtbl(dp1, dp2, i) &
Allocate a table-element trapped-variable block, setting its link field to the
descriptor pointed to by \texttt{dp1}, its entry field to the descriptor pointed
to by \texttt{dp2}, and its hash number field to \texttt{i}.\\

blkreq(i) &
Request \texttt{i} bytes of free space in the allocated block region.\\

\texttt{strreq(i)} &
Request \texttt{i} bytes of space in the allocated string region.\\

\end{xtabular}
}

\subsection[D.5.6 Operations on Structures]{D.5.6 Operations on Structures}

The following routines in \textfn{runtime/rstruct.r} perform operations on structures:

\begin{xtabular}{l@{\hspace{1cm}}p{11cm}}

\texttt{addmem(sp, ep, dp)} &
Add the set-element block pointed to by \texttt{ep} to the set pointed to by
\texttt{sp} at the place pointed to by \texttt{dp}.\\

\texttt{cplist(dp1, dp2, i, j)} &
Copy the sublist from \texttt{i} to \texttt{j} of the list referenced by the
descriptor pointed to by \texttt{dp1}, and place the result in the descriptor
pointed to by \texttt{dp2}.\\

\texttt{{\color{red}locate}(sp1, sp2)} &
Return 1 if the set-element block pointed to by \texttt{sp2} is in the chain
that starts at \texttt{sp1}, but return 0 otherwise.\\

\texttt{memb(sp, dp, i, ip)} &
Set the value pointed to by \texttt{ip} to 1 if the descriptor pointed to by
\texttt{dp} is a member of the set pointed to by \texttt{dp} is a member of the
set pointed to by \texttt{sp} using \texttt{i} as the hash number but to 0
otherwise.\\

\end{xtabular}

\subsection[D.5.7 Input and Output]{D.5.7 Input and Output}

The following routines in \textfn{runtime/rsys.r} perform input and output
operations:

\begin{xtabular}{l@{\hspace{1cm}}p{11cm}}

% the replacement is probably getstrg
\texttt{\color{red}getstr(sbuf, i, fp)} &
Read a line of at most \texttt{i} characters from the file specified by
\texttt{fp}, putting the result in \texttt{sbuf}, returning the number of
characters read, but returning -1 on an end of file.\\

\texttt{putstr(fp, sbuf, i)} &
Write \texttt{i} characters from \texttt{sbuf} on the file specified by \texttt{fp}.\\

\end{xtabular}

\subsection[D.5.8 Error Termination]{D.5.8 Error Termination}

The following routines in \textfn{\hspace{0cm}\color{red}iconx/imain.c} cause error termination:

% [DonW] xtabular gives a really stupid page break at this point
%        so it is replaced by tabular.
\begin{tabular}{l@{\hspace{1cm}}p{11cm}}

\texttt{runerr(i, dp)} &
Terminate execution with error message \texttt{i} showing the offending value in
the dexcriptor pointed to by \texttt{dp}, but omit it if \texttt{dp} is
\texttt{NULL}.\\

\texttt{syserr(sbuf)} &
Terminate execution with system error message \texttt{sbuf}.\\

\end{tabular}

\subsection[D.5.9 Miscellaneous Operations]{D.5.9 Miscellaneous Operations}

The following miscellaneous operations are in \textfn{h/rmacros.h} or \textfn{runtime/rmisc.r}:

\begin{xtabular}{l@{\hspace{1cm}}p{11cm}}

\texttt{deref(dp)} &
Dereference the descriptor pointed to by \texttt{dp}.\\

\texttt{hash(dp)} &
Return a hash value for the descriptor pointed to by \texttt{dp}.\\

\texttt{outimage(fp, dp, i)} &
Write an image for the value of the descriptor pointed to by \texttt{dp} on the
file to by \texttt{fp}, but not callling recursively if \texttt{i} is nonzero.\\

\texttt{qtos(qp, sbuf)} &
Convert the string represented by the qualifier pointed to by \texttt{qp} to a
null-terminated C-style string in \texttt{sbuf}.\\

\end{xtabular}

\subsection[D.5.10 Diagnostics]{D.5.10 Diagnostics}

There are two routines in \textfn{runtime/rmemmgt.c} for producing diagnostic output:

%xtabular produces a really stupid page break
\begin{tabular}{l@{\hspace{1cm}}p{11cm}}

\texttt{descr(dp)} &
Write a diagnostic diagram of the descriptor pointed to by \texttt{dp} to
standard error output.\\

\texttt{blkdump} &
Write a diagnostic diagram of the allocated block region to standard error output.\\

\end{tabular}


\section[D.6 Debugging]{D.6 Debugging}

Debugging a modification to Icon can be difficult unless the overall structure of
the Icon system is understood. It is especially important to understand the way
Icon's data is organized and how storage management works. If an addition to
Icon does not work properly, or if Icon itself no longer functions correctly
after a modification, it is generally advisable to {\em think} about the
possible sources of problems, instead of immediately resorting to a debugger.

Print statements are a crude but often effective means of locating the source of
an error. When adding diagnostic output use
\iconline{fprintf(stderr, " \ldots ",\ \ldots );}
\noindent
instead of the corresponding \texttt{printf}. On some systems it may be useful
to follow such calls by \texttt{fflush(stderr)} to ensure that the diagnostoc
output appears as soon as it is produced.

Icon normally traps floating-point exception and illegal addressing
(segmentation violations) since these errors can result from source-language
progarmming errors, such as division by real zero and excessive recursion
resulting in C stack overflow. For example, Icon normally produces
\texttt{Error 302 ("C stack overflow")} in case of a segmentation violation.
This method of handling such an error is appropriate if Icon itself is free of
bugs, but it interferes with debugging in situations where there are likely to
be bugs in new code.

Assigning a value to the environment variable \texttt{ICONCORE} turns off the
trapping of such errors. In this case most systems produce meaningful diagnostic
messages and a core dump in the case of an exception.  If \texttt{ICONCORE} has
a value, a core dump is produced if an Icon program terminates as a result of
\texttt{runerr} or \texttt{syserr}.  It is therefore good practice to set
\texttt{ICONCORE} when making modifications to Icon. For an extended debugging
session, it may be convenient to set \texttt{dodump} in \textfn{runtime/init.r} to 1.

\section{D.7 Adding new types to the compiler}
\newcommand{\toklbra}{\texttt{~~(~~}}%
\newcommand{\tokrbra}{\texttt{~~)~~}}%
\newcommand{\toklcbra}{\texttt{~~\{~~}}%
\newcommand{\tokrcbra}{\texttt{~~\}~~}}%
\newcommand{\toklsbra}{\texttt{~~[~~}}%
\newcommand{\tokrsbra}{\texttt{~~]~~}}%
\newcommand{\tokcolon}{\texttt{~~:~~}}%
\newcommand{\cceq}{{\normalfont~~::=~~}}%
\newenvironment{ebnf}%
{\begin{specialcode}{\itshape}}%
{\end{specialcode}}%

\begin{quote}
Editorial Note: This section is derived from Appendix F of the RTL
reference document [.ipd261.]. The RTL manual is reproduced here as
appendix G.
\end{quote}

This section describes how to add new types to Icon. It deals with simple
types, aggregate types whose values contain other Icon values, and keywords
that are variables. These are the kinds of types that are most likely to be
added to the language. Executable types such as procedures and
co-expressions are beyond the scope of this section, as are types that
require special representations and variable types that have special
dereferencing semantics.

\subsection{D.7.1 The implementation of Icon Types (reprise)}

As discussed in part 1 and Appendix A of this compendium, an Icon value is
implemented as a two-word \textit{descriptor} containing type information
and value information. The first word of a descriptor is the {\em d-word}.
For the types discussed here, the d-word contains a unique code that
identifies the type. The second word of the descriptor is the {\em v-word};
it contains the actual value or a reference to the value.  Actual values
that are too large to fit in one word are usually put in the block
region. This region is controlled by a storage management system that
includes a garbage collector. The garbage collector is driven by
information in arrays indexed using the type codes associated with the
blocks. The block region contains values for both simple and aggregate
types.

There are several other parts of the run-time system besides the
garbage collector that need information about Icon types. Some are Icon
operations such as the \texttt{type()} function, while others are
automatically invoked features such as error trace back. These are
described in more detail below. Types, of course, typically have operations
associated with them that create and use values of the type.


\subsection{D.7.2 The Type Specification System}

Icon types are used in several places in RTL and new types must be added to
this language. These uses include type checking constructs,
\texttt{return}/\texttt{suspend} statements, and abstract type computations.
In addition, the Icon compiler needs information about types in order to
perform type inferencing. These requirements are satisfied with a type
specification system.

This system is a simple declarative language for naming types and
describing some of their properties. Information from the type
specification system is incorporated in \texttt{rtt} and in \texttt{iconc}
when they are built.

All types specified by the system may be used in the RTL \texttt{is} and
\texttt{type\_case} constructs. They may also be used in abstract type
computations. Aggregate types may be used in a \texttt{new} type expression
in an abstract type computation. A type specification may optionally
indicate that RTL supports a special form of
\texttt{return}/\texttt{suspend} statement that constructs a return value,
in the form of a full descriptor, from a C value for the v-word of the
descriptor.

Type specifications are in the file \textfn{common/typespec.txt}.  Comments
in the file start with \texttt{\#} and continue to the end of the line.

This file is translated into a C header file by the program \texttt{typespec}.
This is not part of the normal Icon build process; entries at the end of
\textfn{common/Makefile} must be uncommented if \textfn{typespec.txt} is
updated.

A type definition in the specification system has the form:

\begin{ebnf}
type-def \cceq identifier opt-abrv \tokcolon kind opt-return
\end{ebnf}

\noindent
where \textit{identifier} is the name of the type and \textit{opt-abrv} is an optional abbreviation for the type name. The abbreviation
has the form:

\begin{ebnf}
opt-abrv \cceq nil |\\
\>\>\>\> \toklcbra identifier \tokrcbra
\end{ebnf}

\noindent
The abbreviation is used in tracing type inferencing and other places where
a compact notation is desired. If no abbreviation is given, the full type
name is used.

There are three kinds of types: \texttt{simple}, \texttt{aggregate}, and
\texttt{variable}. Their syntax and usage are described in separate
sections below. \textit{opt-return} indicates optional RTL
\texttt{return}/\texttt{suspend} support for the type. The four types of
v-words supported by this construct are introduced below as needed. A
complete grammar for the specification language is given near the end of
this appendix.

\subsection{D.7.3 Simple Value Types}
Types with a kind clause of \texttt{simple} are simple in the sense that
values of the type do not have components that contain other Icon
values. These types may otherwise have sophisticated semantics.

There are three ways to implement the values of a type: encode them as C
integers (these are guaranteed to be at least 32 bits long), implement them
as blocks in the block region, or implement them in storage allocated using
\texttt{malloc()} (in theory values can also be put in the string region, but it is
only well suited for Icon strings; descriptors pointing into this region
must have a special form --- described in A.1.1). The choice of
implementation determines the type of C value stored in the v-word of the
descriptor representing the Icon value.

The d-word of a descriptor for one of these types contains a fixed code. It
consists of a small integer type code along with flags describing
characteristics of the descriptor. The necessary changes to
\textfn{rmacros.h} have been covered earlier in section D.3.1.



Three of the \textit{opt-return} type specification clauses are useful
for implementing value types (the fourth is used for variable types;
see below). These clauses add \texttt{return}/\texttt{suspend}
statements to RTL of the form
\begin{ebnf}
\>\texttt{return} \>\>\> type-name \toklbra expr \tokrbra\\
\>\texttt{suspend}\>\>\> type-name \toklbra expr \tokrbra
\end{ebnf}

\noindent

\textit{type-name} is the identifier naming the type. It determines
the \texttt{D\_} constant used for the d-word of the operation's
result descriptor. \textit{expr} is a C expression whose value is
placed in the v-word of the result. The particular \textit{opt-return}
clause chosen determines how the C value is stored in the v-word. The
clauses are

\begin{tabular}{>{\texttt\bgroup}l<{\egroup}%
@{\hspace{1cm}}p{11cm}}
return C\_integer     &%
indicates that the value is cast to a C integer; see the definition of
word in \textfn{h/typedefs.h} for the exact C type used.\\

return block\_pointer &%
indicates that the value is cast to \texttt{(union block *)}; this is
usually used for pointers to blocks in the block region.\\

return char\_pointer  &%
indicates that the value is cast to \texttt{(char *)}. Note, only
descriptors of a special form may point into the string region; the
storage used with \texttt{return char\_pointer} must reside elsewhere.\\
\end{tabular}
 
As an example, the type specification for the cset type is
\begin{iconcode}
cset\{c\}:\>\>\> simple\\
\>\>\>           return block\_pointer
\end{iconcode}

\noindent
Suppose a variable \texttt{cp} within an Icon operation written in RTL
points to a cset block. Then the statement
\iconline{return cset(cp);}
\noindent
constructs a result descriptor for the cset and returns it.

For a type with an associated block, a declaration for the block
structure must be added to \textfn{h/rstructs.h}. By convention, the
structure name is created by prepending \texttt{b\_} to the type name.
The first word of a block must contain its \texttt{T\_} type code. If
different instances of the block may vary in size, the second word of
the block must contain this size in bytes. The structure name of the
new block must be added to the \texttt{union block} declaration
in \textfn{h/rstructs.h}. An allocation routine for the block must be
added to \textfn{runtime/ralc.r}. The macros \texttt{AlcFixBlk()} and
\texttt{AlcVarBlk()} are useful in such routines; see other allocation
routines for guidelines.


There are five arrays in \textfn{runtime/rmemmgt.r} that must be
updated for all types (the details have been covered in section D.3.3).

Storage for the values of a type usually should be allocated in the
block region. However, for interfaces to packages written in C, it may
be necessary to use storage that is not relocated by garbage
collection. While it is possible to place types allocated with
\texttt{malloc()} under control of garbage collection, this is
complicated and beyond the scope of this appendix. See the
implementation of co-expressions for an example of how this can be
done. Short of modifying the garbage collector to handle a new type's
allocated storage, the best practice for any storage allocated with
\texttt{malloc()} is to provide an Icon function, along the lines of
\texttt{close()}, that explicitly frees storage associated with a
value when the program is finished using it.

The necessary changes to built-in functions and support routines have
been covered earlier in section D.3.4.

At the end of this appendix is a check list of files that must be
updated when a type is added to Icon.

\subsection{D.7.4 Aggregate Types}

Aggregate types have values with components that are other Icon values.
The aggregate type specification provides more sophisticated RTL abstract
type computations for the type. These in turn allow \texttt{iconc} to
produce code that is better optimized.

For interpreter-only implementations, abstract type computations are not
used and are optional in RTL code; the \texttt{simple} type specification
may be used in that case. However, the discussion later in this section on
block layout and on the storage management arrays still applies.


The \textit{kind} clause of an \texttt{aggregate} type specification
establishes and names abstract components for the type. The clause is of
the form

\begin{ebnf}
kind \cceq \texttt{aggregate} \toklbra component, \ldots \tokrbra\\
\\
component \cceq identifier |\\
\>\>\>\>\>       \texttt{var} identifier opt-abrv
\end{ebnf}

\noindent
Note, the opt-return clauses discussed in the previous section may be
also used with \texttt{aggregate} types.

The \texttt{aggregate} specification can be thought of as establishing a
sort of ``abstract type record'' whose fields, the abstract components,
summarize the type information of the actual components of values in the
type. Most types are given one abstract component. For example, the set
type has the specification
\begin{iconcode}
set\{S\}: \>\>\> aggregate(set\_elem)\\
\>\>\>           return block\_pointer
\end{iconcode}
\noindent
where \texttt{set\_elem} represents all the elements of a set.


Abstract components can be accessed using dot notation, and the
\texttt{new} abstract type computation can be used to establish a
new subtype of the type (subtypes only exist internally in the
compiler and have no existence at the Icon language level). A subtype
can be returned by the operation and has its own component types
independent of subtypes created elsewhere. The abstract type
computation for Icon set intersection, the \texttt{**} operator, uses
both dot notation and a \texttt{new} expression. It performs
intersection in the abstract type realm. \texttt{x} and \texttt{y} are
the parameters of the operation and may contain different subtypes of
the set type:

\iconline{
return new set(store[type(x).set\_elem] ** store[type(y).set\_elem])
}
\noindent
(Note that the components can be thought of as references to
information contained in a \textit{type store} --- thus the indexing notation.)

Components that represent Icon variables are preceded by \texttt{var} and may
be given abbreviations for use in tracing type inferencing. For
example, the list type has the specification
\begin{iconcode}
list\{L\}: \>\>\> aggregate(var lst\_elem\{LE\})\\
\>\>\>            return block\_pointer
\end{iconcode}

\noindent

These components may be returned from operations and represent the
component as a variable. For example, the abstract type computation
for element generation operator when applied to a list is

\iconline{
return type(dx).lst\_elem
}

\noindent
where \texttt{dx} is the parameter of the operation. When a value rather
than a variable is returned, the component must be ``dereferenced'' by
indexing into the store, as in the abstract type computations of
\texttt{get()}:

\iconline{
return store[type(x).lst\_elem]
}

\noindent
Non-variable components must always be dereferenced.

For types, such as tables, that contain Icon values serving different
purposes, it may be effective to establish several abstract components.

Aggregate types are implemented using blocks that contain descriptors, and
they may be implemented using several kinds of blocks, with some blocks
having pointers to others. When there are multiple blocks, there is always
a \textit{header} block that uses the \texttt{T\_} code of the type. Other
blocks are given internal type codes; these codes must be added to
\textfn{h/rmacros.h} and entries must be made in the storage management
arrays.

Any descriptors in a block must be at the end. The type's entry in the
\texttt{firstd} array is the location of the first descriptor. Any block
pointers in the block must be contiguous. The type's entry in the
\texttt{firstp} array is the location of the first pointer and its
entry in the \texttt{ptrno} array is the number of pointers.

\subsection{D.7.5 Keyword Variable Types}

Keyword variable types have a type specification with a \textit{kind}
clause of the form
\begin{ebnf}
kind \cceq \texttt{variable} var-type-spec\\
\\
var-type-spec \cceq \texttt{initially} type |\\
\>\>\>\>\>\>        \texttt{always} type\\
\\
type \cceq type-name |\\
\>\>\>     type \texttt{~~++~~} type-name\\
\\
type-name \cceq identifier
\end{ebnf}

The compiler must be able to infer the types of values stored in a
keyword variable. The \texttt{initially} option causes the keyword
variable type to be treated as a set of global variables, each
initialized to the given type specified by the \textit{type}
clause. The \texttt{always} option indicates that the keyword always
contains values of the given type and the compiler does no actual
inference on it. \textit{type} may be the union of several types; this
indicates that the type is uncertain and may be any of the ones
specified. A special \textit{type-name}, \texttt{any\_value},
indicates complete uncertainty.

\noindent The clause
\iconline{
always any-value
}
\noindent
is a correct, although entirely imprecise, description of any keyword variable.

This appendix assumes that keyword variables are implemented by global
descriptors (though other techniques are possible). The \textit{opt-return}
clause of the form
\iconline{
return descriptor\_pointer
}
\noindent
is useful for implementing keyword variables. The v-word of a result
descriptor from a corresponding \texttt{return}/\texttt{suspend}
expression is of \texttt{type struct descrip *}.


Some of the same files must be updated for variable types as for value
types. Type codes must be added to \textfn{h/rmacros.h}.
The \texttt{D\_} code must have the \texttt{F\_Var} flag set, for example:
\iconline{
\#define D\_Kywdint (T\_Kywdint | D\_Typecode | F\_Ptr | F\_Var)
}
\noindent
The storage management tables and the \texttt{outimage()} routine also must be
updated.


Other updates are unique to variable types. The global descriptor must be
established. \textfn{runtime/data.r} contains its declaration.
\texttt{icon\_init()} in \textfn{runtime/init.r} initializes the descriptor,
and \textfn{h/rexterns.h} contains an \texttt{extern} for it. Dereferencing
must be updated; it is performed by \texttt{deref()} in \textfn{runtime/cnv.r}.
Assignment must be updated; it is handled by the macro \texttt{GeneralAsgn()}
in \textfn{runtime/oasgn.r}. The \texttt{name()} function is updated by
changing the support routine \texttt{get\_name()} in \textfn{runtime/rdebug.r.}
The \texttt{variable()} function is updated by changing the support
routine \texttt{getvar()} in \textfn{runtime/rmisc.r}.

The keyword itself goes in \textfn{runtime/keyword.r}. For example,
\texttt{\&random} is of type \texttt{kywdint} and is implemented by the descriptor
\texttt{kywd\_ran}; its definition is
\begin{iconcode}
keyword\{1\} random\\
\>abstract \{\\
\>\>return kywdint\\
\>\}\\
\>inline \{\\
\>\>return kywdint(\&kywd\_ran);\\
\>\}\\
end
\end{iconcode}
\noindent
For the interpreter, the keyword name must be added to
\texttt{icont/key\_text.c}. These names are in alphabetical order.

If the descriptor may contain a value under control of garbage collection,
the support routine \texttt{collect()} in \textfn{runtime/rmemmgt.r} must be
updated. \texttt{postqual()} preserves references to the string region;
the macro \texttt{Qual()} is used to check for such references.
\texttt{markblock()} preserves references to blocks; the macro \texttt{Pointer()}
is used to check for such references.

\clearpage
\newpage
\subsection{D.7.6 The Complete Grammar for the Type Specification System}
\begin{ebnf}
type-def \>\>\>\> \cceq \>\> identifier opt-abrv \tokcolon kind opt-return\\
\\
kind \>\>\>\>     \cceq \>\> \texttt{simple} |\\
\>\>\>\>\>\>      \texttt{aggregate} \toklbra component, \ldots \tokrbra |\\
\>\>\>\>\>\>      \texttt{variable} var-type-spec\\
\\
component \>\>\>\> \cceq \>\> identifier |\\
\>\>\>\>\>\>                  \texttt{var} identifier opt-abrv\\
\\
var-type-spec \>\>\>\>\cceq \>\> \texttt{initially} type |\\
\>\>\>\>\>\>                     \texttt{always} type\\
\\
type \>\>\>\> \cceq \>\> type-name |\\
\>\>\>\>\>\>             type \texttt{~~++~~} type-name\\
\\
type-name \>\>\>\> \cceq \>\> identifier\\
\\
opt-abrv \>\>\>\>  \cceq \>\> nil |\\
\>\>\>\>\>\>                    \toklcbra identifier \tokrcbra\\
\\
opt-return \>\>\>\>  \cceq \>\>nil |\\
\>\>\>\>\>\>      \texttt{return  block\_pointer} |\\
\>\>\>\>\>\>      \texttt{return  descriptor\_pointer} |\\
\>\>\>\>\>\>      \texttt{return  char\_pointer} |\\
\>\>\>\>\>\>      \texttt{return  C\_integer}\\
\end{ebnf}

\clearpage
\newpage
\section[D.8 A check list for adding types]%
        {D.8 A check list for adding types%
        {\hfill\small\textit{(on one page for easy photocopying)}}}
%
{\renewcommand{\arraystretch}{0.8}% Squeeze the lines together
\begin{tabular}{%
@{$\square$\hspace{0.5cm}}>{\textfn\bgroup}l<{\egroup}%
@{\hspace{0.5cm}--\hspace{0.5cm}}l%
}
\multicolumn{2}{l}{\bf All Types}\\
\multicolumn{2}{l}{}\\
  common/typespec.txt & add type specification\\
  common/Makefile   & uncomment entries near the end of the file\\
  h/rmacros.h       & add T\_\textit{Type} macro\\
  h/rmacros.h       & add D\_\textit{Type} macro\\
  runtime/rmemmgt.r & \texttt{bsizes} table\\
  runtime/rmemmgt.r & \texttt{firstd} table\\
  runtime/rmemmgt.r & \texttt{firstp} table\\
  runtime/rmemmgt.r & \texttt{ptrno} table\\
  runtime/rmemmgt.r & \texttt{blkname} table\\
  runtime/rmisc.r   & update \texttt{outimage()}\\
\multicolumn{2}{l}{}\\
\multicolumn{2}{l}{\bf All Value Types\vspace{2ex}}\\
  runtime/fmisc.r   & update \texttt{copy()}\\
  runtime/fmisc.r   & update \texttt{type()}\\
  runtime/rcomp.r   & update \texttt{anycmp()}\\
  runtime/rcomp.r   & update \texttt{order()}\\
  runtime/rcomp.r   & update \texttt{equiv()}\\
  runtime/rmisc.r   & update \texttt{getimage()}\\
\multicolumn{2}{l}{}\\  
\multicolumn{2}{l}{\bf Types Implemented In The Block Region\vspace{2ex}}\\
  h/rstructs.h      & add declaration for the block structure\\
  h/rstructs.h      & update the \texttt{union block} declaration\\
  runtime/ralc.r    & add an allocation routine\\
\multicolumn{2}{l}{}\\  
\multicolumn{2}{l}{\bf Types With Sizes\vspace{2ex}}\\
  runtime/omisc.r   & update size operator\\
\multicolumn{2}{l}{}\\  
\multicolumn{2}{l}{\bf  All Keyword Variable Types\vspace{2ex}}\\
  h/rexterns.h      & extern for keyword descriptor\\
  runtime/cnv.r     & update \texttt{deref()}\\
  runtime/data.r    & declaration for keyword descriptor\\
  runtime/init.r    & initialize keyword descriptor\\
  runtime/keyword.r & add keyword declaration\\
  runtime/oasgn.r   & update \texttt{GeneralAsgn()} macro\\
  runtime/rdebug.r  & update \texttt{get\_name()}\\
  runtime/rmisc.r   & update \texttt{getvar()}\\
\multicolumn{2}{l}{}\\  
\multicolumn{2}{l}{\bf Keyword Variables That Must Be Garbage Collected\vspace{2ex}}\\
  runtime/rmemmgt.r & update \texttt{collect()}\\
\end{tabular}
}
\bigskip
