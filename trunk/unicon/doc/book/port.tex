\clearpage\section{Appendix E: Portability Considerations}

Unicon{\textquotesingle}s POSIX-based system interface is divided into
two pieces. The facilities presented in Chapter 5 are portable. You can
expect the portable system interface to be available on any
implementation of Unicon. Facilities will be implemented as completely
as is feasible on each platform. This appendix presents the rest of the
Unicon POSIX interface, as well as some notes on functionality specific
to Microsoft Windows.

\subsection[POSIX extensions]{POSIX extensions}

\index{POSIX extensions}Technically, the extensions presented in this
section may or may not be part of the POSIX standard. The important
issue is that they are a part of the Unicon language as implemented on
major POSIX-compliant UNIX platforms such as Solaris and Linux. Ports
of Unicon to non-POSIX or quasi-POSIX platforms may or may not
implement any of these facilities.

\subsubsection{Information from System Files}

There are four functions that read information from system files,
getpw() to read the password file, getgr() for the group file,
gethost() for hostnames, and getserv() for network services. Called
with an argument (usually a string), they perform a lookup in the
system file. When called with no arguments, these functions step
through the files one entry at a time.

The functions \texttt{setpwent()}, \texttt{setgrent()},
\texttt{sethostent()}, and \texttt{setservent()} do
the same things as their POSIX C language counterparts; they reset the
file position used by the get* routines to the beginning of the file.
These functions return records whose members are similar to the C
structures returned by the system functions \texttt{getpwuid(2)},
\texttt{gethostbyname(2)}, etc.

\subsubsection{Fork and Exec}

POSIX-compliant systems support an additional, lower-level interface
using the functions \index{fork()}\texttt{fork()} and \texttt{exec()}.
\texttt{fork()} causes the
system to make a copy of the current process. After the fork there will
be two identical processes that share all resources like open files,
and differ only in one respect: the return value they each got from the
call to \texttt{fork()}. One process gets a zero and is called the child; the
other gets the process id of the child and is called the parent.

Quite often \texttt{fork()} is used to run another program. In that case, the
child process uses the system call \texttt{exec()} which replaces the code of
the process with the code of a new program. This \texttt{fork()}/\texttt{exec()} pair is
comparable to calling \texttt{system()} and using the option to not wait for the
command to complete.

The first argument to \texttt{exec()} is the filename of the program to execute,
and the remaining arguments are the values of argv that the program
will get, starting with argv[0]. 

\iconcode{
exec({\textquotedbl}/bin/echo{\textquotedbl},
{\textquotedbl}echo{\textquotedbl},
{\textquotedbl}Hello,{\textquotedbl},
{\textquotedbl}world!{\textquotedbl})}

\subsubsection{POSIX Functions}

These functions are present in all Unicon binaries, but you can expect
them to fail on most non-UNIX platforms. Check the readme.txt file that
comes with your installation to ascertain whether it supports any of
these functions.

\paragraph[chown(f, u:{}-1, g:{}-1) : ?\ \ \ \ \ \ \ \ \ \ \ \ 
\ \ \ \ \ \ \ \ change owner]{chown(f, u:-1, g:-1) :
?\ \ \ \ \ \ \ \ \ \ \ \  \ \ \ \ \ \ \ \ change owner}
chown(f, u, g) sets the owner of a file (or string filename) f to owner
u and group g. The user and group arguments can be numeric
ID{\textquotesingle}s or names.

\paragraph[chroot(string) : ?\ \ \ \ \ \ \ \ \ \ \ \  \ \ \ \ \ \ change
filesystem root]{chroot(string) : ?\ \ \ \ \ \ \ \ \ \ \ \ 
\ \ \ \ \ \ change filesystem root}
chroot(f) changes the root of the filesystem to f. This may be of
assistance in implementing security measures.

\paragraph[crypt(string, string) : string\ \ \ \ \ \ \ \ \ \ \ \ 
\ \ \ encrypt password]{crypt(string, string) :
string\ \ \ \ \ \ \ \ \ \ \ \  \ \ \ encrypt password}
crypt(s1, s2) encrypts the password s1 with the salt s2. The first two
characters of the returned string will be the salt.

\paragraph[exec(string, string, ...) : null\ \ \ \ \ \ \ \ \ \ 
\ \ \ \ execute program]{exec(string, string, ...) :
null\ \ \ \ \ \ \ \ \ \  \ \ \ \ execute program}
exec(s, arg0, arg1, arg2, ...) \textit{replaces} the currently executing
Icon program with a new program named in s. The other arguments are
passed to the program. Consult the POSIX exec(2) manual pages for more
details. \texttt{s} must be a path to a binary executable program, not
to a shell script (or, on UNIX) an Icon program. If you want to run
such a script, the first argument to exec() should be the binary that
can execute them, such as /bin/sh.

\paragraph[fcntl(file, string, options)\ \ \ \ \ \ \ \ \ \ \ \ \ \ 
\ \ \ \ file control]{fcntl(file, string,
options)\ \ \ \ \ \ \ \ \ \ \ \ \ \  \ \ \ \ file control}
fcntl(file, cmd, arg) performs miscellaneous operations on the open
file. See the fcntl(2) manual page for more details. Directories and
DBM files cannot be arguments to fcntl(). The following characters are
the possible values for cmd: 

\ \ \ \ \ f \ \ \ \ \ \ \ \ Get flags (F\_SETFL)

\ \ \ \ \ F \ \ \ \ \ \ \ Set flags (F\_GETFL)

\ \ \ \ \ x \ \ \ \ \ \ \ Get close-on-exec flags (F\_GETFD)

\ \ \ \ \ X \ \ \ \ \ \ \ Set close-on-exec flag (F\_SETFD)

\ \ \ \ \ l \ \ \ \ \ \ \ \ Get file lock (F\_GETLK)

\ \ \ \ \ L \ \ \ \ \ \ \ Set file lock (F\_SETLK)

\ \ \ \ \ W \ \ \ \ \ \ Set file lock and wait (F\_SETLKW)

\ \ \ \ \ o \ \ \ \ \ \ \ \ Get file owner or process group (F\_GETOWN)

\ \ \ \ \ O \ \ \ \ \ \ \ Set file owner or process group (F\_SETOWN)

In the case of L, the arg value should be a string that describes the
lock, otherwise arg is an integer. A record will be returned by
F\_GETLK: 

\iconcode{
\ \ record posix\_lock(value, pid)
}

The lock string consists of three parts separated by commas: the type of
lock (r, w or u), the starting position, and the length. The starting
position can be an offset from the beginning of the file (e.g. 23), end
of the file (e.g. -50), or from the current position in the file (e.g.
+200). A length of 0 means lock till EOF. These characters represent
the file flags set by F\_SETFL and accessed by F\_GETFL: 

\ \ \ \ \ d \ \ \ \ \ \ \ FNDELAY 

\ \ \ \ \ s \ \ \ \ \ \ \ FASYNC 

\ \ \ \ \ a \ \ \ \ \ \ \ FAPPEND 

\paragraph[fdup(file, file) : ?\ \ \ \ \ \ \ \ \ \ \ \ 
\ \ \ \ \ duplicate file descriptor]{fdup(file, file) :
?\ \ \ \ \ \ \ \ \ \ \ \  \ \ \ \ \ duplicate file descriptor}
\texttt{fdup(src, dest)} is based on the POSIX \texttt{dup2(2)} system
call. It is used to
modify a specific UNIX file descriptor, such as just before calling
\texttt{exec()}. The dest file is closed; src is made to have its Unix file
descriptor; and the second file is replaced by the first. 

\paragraph[filepair() : list\ \ \ \ \ \ \ \ \ \ \ \ \ \ 
\ \ \ \ \ \ \ create connected files]{filepair() :
list\ \ \ \ \ \ \ \ \ \ \ \ \ \  \ \ \ \ \ \ \ create connected files}
filepair() creates a bi-directional pair of connected files. This is
analogous to the POSIX \texttt{socketpair(2)} function. It returns a list of two
files, such that writes on one will be available on the other. The
connection is bi-directional, unlike that returned by function \texttt{pipe()}.
The two files are indistinguishable. Caution: typically, the pair is
created just before a \texttt{fork()}; after it, one process should close L[1]
and the other should close L[2] or you will not get proper end-of-file
notification. 

\paragraph[fork() : integer\ \ \ \ \ \ \ \ \ \ \ \ \ \ \ \  fork
process]{fork() : integer\ \ \ \ \ \ \ \ \ \ \ \ \ \ \ \  fork process}
\texttt{fork()} creates a new process that is identical to the current process
except in the return value. The parent gets a return value that is the
PID of the child, and the child gets 0.

\paragraph[getegid() : string\ \ \ \ \ \ \ \ \ \ 
\ \ \ \ \ \ \ \ \ \ get effective group identity]{getegid() :
string\ \ \ \ \ \ \ \ \ \  \ \ \ \ \ \ \ \ \ \ get effective group
identity}
\texttt{getegid()} produces the effective group identity (gid) of the current
process. The name is returned if it is available, otherwise the numeric
code is returned.

\paragraph[geteuid() : string\ \ \ \ \ \ \ \ \ \ 
\ \ \ \ \ \ \ \ \ \ \ \ \ get effective user identity]{geteuid() :
string\ \ \ \ \ \ \ \ \ \  \ \ \ \ \ \ \ \ \ \ \ \ \ get effective user
identity}
\texttt{geteuid()} produces the effective user identity (uid) of the current
process. The name is returned if it is available, otherwise the numeric
code is returned.

\paragraph[getgid() : string\ \ \ \ \ \ \ \ \ \ \ \ \ \  get group
identity]{getgid() : string\ \ \ \ \ \ \ \ \ \ \ \ \ \  get group
identity}
\texttt{getgid()} produces the real group identity (gid) of the current process.
The name is returned if it is available, otherwise the numeric code is
returned.

\paragraph[getgr(g) : record\ \ \ \ \ \ \ \ \ \ \ \  \ \ \ \ get group
information]{getgr(g) : record\ \ \ \ \ \ \ \ \ \ \ \  \ \ \ \ get
group information}
\texttt{getgr(g)} returns a record that contains group file
information for group
g, a string group name or an integer group code. If g is null, each
successive call to getgr() returns the next entry. setgrent() resets
the sequence to the beginning. Return type:\\
\ \ \texttt{record posix\_group(name, passwd, gid, members)}

\paragraph[gethost(x) : record\ \ \ \ \ \ \ \ \ \ 
\ \ \ \ \ \ \ get host information]{gethost(x) :
record\ \ \ \ \ \ \ \ \ \  \ \ \ \ \ \ \ get host information}
\texttt{gethost(n)} for network connection n returns a string containing
the IP# and port this machine is using for a network connection.
\texttt{gethost(s)} returns a record that contains host information for the
machine named s. If s is null, each successive call to \texttt{gethost()}
returns the next entry. sethostent() resets the sequence to the
beginning. The aliases and addresses are comma separated lists of
aliases and addresses (in a.b.c.d format) respectively. Return
type:\\
\ \ \texttt{record posix\_hostent(name, aliases, addresses)}

\paragraph[getpgrp() : integer\ \ \ \ \ \ \ \ \ \ \ \ \ \  \ \ get
process group]{getpgrp() : integer\ \ \ \ \ \ \ \ \ \ \ \ \ \  \ \ get
process group}
\texttt{getpgrp()} returns the process group of the current process. 

\paragraph[getpid() : integer\ \ \ \ \ \ \ \ \ \ \ \  \ \ get process
identification]{getpid() : integer\ \ \ \ \ \ \ \ \ \ \ \  \ \ get
process identification}
getpid() produces the process identification (pid) of the current
process.

\paragraph[getppid() : integer?\ \ \ \ \ \ \ \ \ \  \ \ get parent
process identification]{getppid() : integer?\ \ \ \ \ \ \ \ \ \ 
\ \ get parent process identification}
\texttt{getppid()} produces the pid of the parent process.

\paragraph[getpw(u) : posix\_password\ \ \ \ \ \ \ \  \ get password
information]{getpw(u) : posix\_password\ \ \ \ \ \ \ \  \ get password
information}
\texttt{getpw(u)} returns a record that contains password file
information. \texttt{u} can
be a numeric uid or a user name. If u is null, each successive call to
\texttt{getpw()} returns the next entry and \texttt{setpwent()} resets
the sequence to the beginning. Return type:\\
\ \ \texttt{record posix\_password(name, passwd, uid, gid, age, comment, gecos,
dir, shell)}

\paragraph[getserv(string, string) : posix\_servent\ \ \ \ \ \ 
\ \ \ \ \ get service information]{getserv(string, string) :
posix\_servent\ \ \ \ \ \  \ \ \ \ \ get service information}
\texttt{getserv(s, proto)} returns a record that contains service information for
the service s using protocol proto. If s is null, each successive call
to \texttt{getserv()} returns the next entry. \texttt{setservent()} resets the sequence
to the beginning. Return type:\\
\ \ record posix\_servent(name, aliases, port, proto) 

If proto is defaulted, it will return the first matching entry. 

\paragraph[getuid() : string \ \ \ \ \ \ \ \ \ \ \ \ \ \  \ \ \ \ \ get
user identity]{getuid() : string \ \ \ \ \ \ \ \ \ \ \ \ \ \ 
\ \ \ \ \ get user identity}
\texttt{getuid()} produces the real user identity (uid) of the current process. 

\paragraph[kill(integer, x) : ?\ \ \ \ \ \ \ \ \ \ \ \ \ \ \ \  \ \ kill
process]{kill(integer, x) : ?\ \ \ \ \ \ \ \ \ \ \ \ \ \ \ \  \ \ kill
process}
\texttt{kill(pid, signal)} sends a signal to the process specified by pid. The
second parameter can be the string name or the integer code of the
signal to be sent.

\paragraph[link(string, string) : ?\ \ \ \ \ \ \ \ \ \ \ \ \ \ 
\ \ \ \ \ \ \ link files]{link(string, string) :
?\ \ \ \ \ \ \ \ \ \ \ \ \ \  \ \ \ \ \ \ \ link files}
\index{link!file system}\texttt{link(src,dest)} creates a (hard) link dest that
points to src.

\paragraph[readlink(string) : string?\ \ \ \ \ \ \ \ \ \ \ \  \ \ read
symbolic link]{readlink(string) : string?\ \ \ \ \ \ \ \ \ \ \ \ 
\ \ read symbolic link}
\texttt{readlink(s)} produces the filename referred to in a symbolic link at path
s.

\paragraph[setgid(integer) : ?\ \ \ \ \ \ \ \ \ \ \ \  \ \ \ \ \ \ set
group identification]{setgid(integer) : ?\ \ \ \ \ \ \ \ \ \ \ \ 
\ \ \ \ \ \ set group identification}
\texttt{setgid(g)} sets the group id of the current process to g. See the UNIX
\texttt{setgid(2)} man page. 

\paragraph[setgrent() : null\ \ \ \ \ \ \ \ \ \  \ \ \ \ reset group
information cursor]{setgrent() : null\ \ \ \ \ \ \ \ \ \  \ \ \ \ reset
group information cursor}
\texttt{setgrent()} resets and rewinds the pointer to the group file used by
\texttt{getgr()} when \texttt{getgr()} is called with no arguments. 

\paragraph[sethostent(integer:1) : null\ \ \ \ \ \ \ \ 
\ \ \ \ \ \ \ reset host information cursor]{sethostent(integer:1) :
null\ \ \ \ \ \ \ \  \ \ \ \ \ \ \ reset host information cursor}
\texttt{sethostent(stayopen)} resets and rewinds the pointer to the host file
used by \texttt{gethost()}. The argument defines whether the file should be kept
open between calls to \texttt{gethost()}; a nonzero value (the default) keeps it
open. 

\paragraph[setpgrp() : ?\ \ \ \ \ \ \ \ \ \ \ \ \ \ \ \  \ \ set process
group]{setpgrp() : ?\ \ \ \ \ \ \ \ \ \ \ \ \ \ \ \  \ \ set process
group}
\texttt{setpgrp()} sets the process group. This is equivalent to
\texttt{setpgrp(0, 0)} on BSD systems. 

\paragraph[setpwent() : null\ \ \ \ \ \ \ \  \ \ \ \ \ \ \ \ \ reset
password information cursor]{setpwent() : null\ \ \ \ \ \ \ \ 
\ \ \ \ \ \ \ \ \ reset password information cursor}
\texttt{setpwent()} resets and rewinds the pointer to the password file used by
\texttt{getpw()} when \texttt{getpw()} is called with no arguments.

\paragraph[setservent(integer:1) : null \ \ \ \ \ \  \ \ reset service
information cursor]{setservent(integer:1) : null \ \ \ \ \ \  \ \ reset
service information cursor}
\texttt{setservent(stayopen)} resets and rewinds the pointer to the
services file used by \texttt{getserv()}. The argument defines whether
the file should be kept open between calls to \texttt{getserv()}; a
nonzero value (the default) keeps it open. 

\paragraph[setuid(integer) : ?\ \ \ \ \ \ \ \ \ \ \ \ \ \ 
\ \ \ \ \ \ set user identity]{setuid(integer) :
?\ \ \ \ \ \ \ \ \ \ \ \ \ \  \ \ \ \ \ \ set user identity}
\texttt{setuid(u)} sets the user id of the current process to
\texttt{u}. See the UNIX \texttt{setuid(2)} man page. 

\paragraph[symlink(string, string) : ?\ \ \ \ \ \ \ \ \ \ \ \ 
\ \ \ \ \ symbolic file link]{symlink(string, string) :
?\ \ \ \ \ \ \ \ \ \ \ \  \ \ \ \ \ symbolic file link}
\index{link!symbolic file}symlink(src, dest) makes a symbolic link dest
that points to src.

\paragraph[umask(integer) : integer\ \ \ \ \ \ \ \ \ \ 
\ \ \ \ \ \ \ \ \ file permission mask]{umask(integer) :
integer\ \ \ \ \ \ \ \ \ \  \ \ \ \ \ \ \ \ \ file permission mask}
\texttt{umask(u)} sets the umask of the process to \texttt{u}. umask integers
encode nine bits for the read, write, and execute permissions of user,
group, and world access. See also \texttt{chmod()}. Each bit in the umask turns
\textit{off} that access, by default, for newly created files. The old
value of the umask is returned.

\paragraph[wait(integer:{}-1, integer:0) : string\ \ \ \ \ \ \ \ \ \ 
\ \ \ \ \ \ wait for process]{wait(integer:-1, integer:0) :
string\ \ \ \ \ \ \ \ \ \  \ \ \ \ \ \ wait for process}
\texttt{wait(pid, options)} waits for a process given by pid to terminate or
stop. The default pid value causes the program to wait for all the
current process{\textquotesingle} children. The options parameter is an
OR of the values 1 (return if no child has exited) and 2 (return for
children that are stopped, not just for those that exit). The returned
string represents the pid and the exit status as defined in this table:


\ \ \ \ \ UNIX equivalent\ \ \ \ \ \ example of returned string\\
 \ \ \ \ WIFSTOPPED(status)\ \ \ \ {\textquotedbl}1234
stopped:SIGTSTP{\textquotedbl} \\
 \ \ \ \ WIFSIGNALLED(status)\ \ \ \ {\textquotedbl}1234
terminated:SIGHUP{\textquotedbl} \\
 \ \ \ \ WIFEXITED(status)\ \ \ \ {\textquotedbl}1234
exit:1{\textquotedbl} \\
 \ \ \ \ WIFCORE(status)\ \ \ \ \ \ {\textquotedbl}1234
terminated:SIGSEGV:core{\textquotedbl} 

Currently the \textsf{rusage} facility is unimplemented.

\subsection[Microsoft Windows]{Microsoft Windows}
Windows versions of Unicon support certain non-portable extensions to
the system interfaces. Consult the readme.txt file and the on-line
reference (Unicon Technical Report 7) for details.

\subsubsection{Partial Support for POSIX}

Windows supports the \texttt{getpid()} function, but not many of the other
process-related functions such as \texttt{getppid()}. Windows supports the
\texttt{exec()} function in addition to Unicon{\textquotesingle}s extended
flavor of \texttt{system()}, but these functions may only launch Windows 32-bit
.EXE binaries.

Windows Unicon supports the following signals in functions such as
\texttt{kill()}: SIGABRT, SIGBREAK, SIGFPE, SIGILL, SIGINT, SIGSEGV, and
SIGTERM.

Windows Unicon supports the \texttt{umask()} function, but ignores execute
permission and treats user/group/world identically, using the most
permissive access specified in the umask.

\subsubsection{Native User Interface Components}

Windows Icon supports limited access to platform-native user interface
components and multimedia controls.

\texttt{WinButton(w,s,x,y,wd,ht)} installs a pushbutton with label s on window w.

\texttt{WinColorDialog(w, s)} allows the user to choose a color for a
window{\textquotesingle}s context.

\texttt{WinEditRegion(w, s, s2, x, y, wd, ht)} installs an edit box with label s.

\texttt{WinFontDialog(w, s)} allows the user to choose a font for a
window{\textquotesingle}s context.

\texttt{WinMenuBar(w, L1, L2,...)} installs a set of top-level menus.

\texttt{WinOpenDialog(w, s1, s2, i, s3, j)} allows the user to choose a file to
open.

\texttt{WinPlayMedia(w, x[])} plays a multimedia resource.

\texttt{WinSaveDialog(w, s1, s2, i, s3, j)} allows the user to choose a file to
save.

\texttt{WinScrollBar(w, s, i1, i2, i3, x, y, wd, ht)} installs a scrollbar.

\texttt{WinSelectDialog(w, s1, buttons)} allows the user to select from a set of
choices.

