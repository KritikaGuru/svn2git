\chapter{Portability Considerations}

Unicon's POSIX-based system interface facilities presented in Chapter
5 are portable.  You can expect the portable system interface to be
available on any implementation of Unicon. This appendix presents
additional, non-portable elements of the Unicon POSIX interface, as
well as some notes on functionality specific to Microsoft Windows.

\section{POSIX extensions}

\index{POSIX extensions}The extensions presented in this section may
or may not be part of the POSIX standard, but they are a part of the
Unicon language as implemented on major POSIX-compliant UNIX platforms
such as Solaris and Linux. Ports of Unicon to non-POSIX or quasi-POSIX
platforms may or may not implement any of these facilities.

\subsection*{Information from system files}

There are four functions that read information from the system:
\texttt{getpw()} to read the password file, \texttt{getgr()} for the
group file, \texttt{gethost()} for hostnames, and \texttt{getserv()}
for network services. Called with an argument (usually a string), they
perform a lookup in the system file; called with no arguments,
these functions step through the files one entry at a time.

The functions \texttt{setpwent()}, \texttt{setgrent()},
\texttt{sethostent()}, and \texttt{setservent()} do the same things as
their POSIX C language counterparts; they reset the file position used
by the get* routines to the beginning of the file.  These functions
return records whose members are similar to the C structures returned
by the system functions \texttt{getpwuid(2)},
\texttt{gethostbyname(2)}, etc.

\subsection*{\texttt{fork} and \texttt{exec}}

POSIX-compliant systems support a process-launch interface
using the functions \index{fork()}\texttt{fork()} and \texttt{exec()}.
\texttt{fork()} makes a copy of the current
process. After the fork there are two identical processes that
share resources such as open files, and differ only in one respect:
the return value they received from the call to \texttt{fork()}. One
process gets a zero and is called the child; the other gets the
process id of the child and is called the parent.

Usually \texttt{fork()} is used to run another program. In that
case, the child process uses the system call \texttt{exec()} which
replaces the code of the process with the code of a new program. This
\texttt{fork()}/\texttt{exec()} pair is comparable to calling
\texttt{system()} and using the option to not wait for the command to
complete.

The first argument to \texttt{exec()} is the filename of the program
to execute, and the remaining arguments are the values of argv that
the program will get, starting with \texttt{argv[0]}.

\iconcode{
exec("/bin/echo",
"echo",
"Hello,",
"world!")}

\subsection*{POSIX functions}

These functions are present in all Unicon binaries, but you can expect them to
fail on most non-UNIX platforms. Check the readme.txt file that comes with
your installation to ascertain whether it supports any of these functions.

\bigskip\hrule\vspace{0.1cm}
\noindent {\bf chown(chown(f, u:{}-1, g:{}-1) : ? \hfill change owner}

\noindent
\texttt{chown(f, u, g)} sets the owner of a file (or string filename) f to
owner u and group g. The user and group arguments can be numeric
ID's or names.

\bigskip\hrule\vspace{0.1cm}
\noindent {\bf chroot(string) : ? \hfill change filesystem root}

\noindent
\texttt{chroot(f)} changes the root directory of the filesystem to f.

\bigskip\hrule\vspace{0.1cm}
\noindent {\bf crypt(string, string) : string \hfill encrypt password}

\noindent
\texttt{crypt(s1, s2)} encrypts the password s1 with the salt s2. The first two
characters of the returned string will be the salt.

\bigskip\hrule\vspace{0.1cm}
\noindent {\bf exec(string, string, ...) : null \hfill execute program}

\noindent
\texttt{exec(s, arg0, arg1, arg2, ...)}
\textit{replaces} the currently executing
Icon program with a new program named in \texttt{s}. The other arguments are
passed to the program. Consult the POSIX exec(2) manual pages for more
details. \texttt{s} must be a path to a binary executable program, not
to a shell script (or, on UNIX) an Icon program. If you want to run
such a script, the first argument to \texttt{exec()} should be the binary that
can execute them, such as /bin/sh.

\bigskip\hrule\vspace{0.1cm}
\noindent {\bf fcntl(file, string, options) \hfill file control}

\noindent
\texttt{fcntl(file, cmd, arg)} performs miscellaneous operations on the open
file. See the fcntl(2) manual page for more details. Directories and
DBM files cannot be arguments to \texttt{fcntl()}. The following characters are
the possible values for cmd: 

\ \ \ \ \ f \ \ \ \ \ \ \ \ Get flags (F\_SETFL)

\ \ \ \ \ F \ \ \ \ \ \ \ Set flags (F\_GETFL)

\ \ \ \ \ x \ \ \ \ \ \ \ Get close-on-exec flags (F\_GETFD)

\ \ \ \ \ X \ \ \ \ \ \ \ Set close-on-exec flag (F\_SETFD)

\ \ \ \ \ l \ \ \ \ \ \ \ \ Get file lock (F\_GETLK)

\ \ \ \ \ L \ \ \ \ \ \ \ Set file lock (F\_SETLK)

\ \ \ \ \ W \ \ \ \ \ \ Set file lock and wait (F\_SETLKW)

\ \ \ \ \ o \ \ \ \ \ \ \ \ Get file owner or process group (F\_GETOWN)

\ \ \ \ \ O \ \ \ \ \ \ \ Set file owner or process group (F\_SETOWN)

In the case of L, the arg value should be a string that describes the
lock, otherwise arg is an integer. A \texttt{record posix\_lock(value,
pid)} will be returned by F\_GETLK.

The lock string consists of three parts separated by commas: the type of
lock (r, w or u), the starting position, and the length. The starting
position can be an offset from the beginning of the file (e.g. 23), end
of the file (e.g. -50), or from the current position in the file (e.g.
+200). A length of 0 means lock till EOF. These characters represent
the file flags set by F\_SETFL and accessed by F\_GETFL: 

\ \ \ \ \ d \ \ \ \ \ \ \ FNDELAY 

\ \ \ \ \ s \ \ \ \ \ \ \ FASYNC 

\ \ \ \ \ a \ \ \ \ \ \ \ FAPPEND 

\bigskip\hrule\vspace{0.1cm}
\noindent {\bf fdup(file, file) : ? \hfill duplicate file descriptor}

\noindent
\texttt{fdup(src, dest)} is based on the POSIX \texttt{dup2(2)} system
call. It is used to
modify a specific UNIX file descriptor, such as just before calling
\texttt{exec()}. The dest file is closed; src is made to have its Unix file
descriptor; and the second file is replaced by the first. 

\bigskip\hrule\vspace{0.1cm}
\noindent {\bf filepair() : list \hfill create connected files}

\noindent
\texttt{filepair()} creates a bi-directional pair of files
analogous to the POSIX \texttt{socketpair(2)} function. It returns a list of two
indistinguishable files; writes on one will be available on the other. The
connection is bi-directional, unlike that of function \texttt{pipe()}.
Caution: typically, the pair is
created just before a \texttt{fork()}; after it, one process should close L[1]
and the other should close L[2] or you will not get proper end-of-file
notification. 

\bigskip\hrule\vspace{0.1cm}
\noindent {\bf fork() : integer \hfill fork process}

\noindent
\texttt{fork()} creates a new process that is identical to the current process
except in the return value. The parent gets a return value that is the
PID of the child, and the child gets 0.

\bigskip\hrule\vspace{0.1cm}
\noindent {\bf getegid() : string \hfill get effective group identity}

\noindent
\texttt{getegid()} produces the effective group identity (gid) of the current
process. The name is returned if it is available, otherwise the numeric
code is returned.

\bigskip\hrule\vspace{0.1cm}
\noindent {\bf geteuid() : string \hfill get effective user identity}

\noindent
\texttt{geteuid()} produces the effective user identity (uid) of the current
process. The name is returned if it is available, otherwise the numeric
code is returned.

\bigskip\hrule\vspace{0.1cm}
\noindent {\bf getgid() : string \hfill get group identity}

\noindent
\texttt{getgid()} produces the real group identity (gid) of the current process.
The name is returned if it is available, otherwise the numeric code is
returned.

\bigskip\hrule\vspace{0.1cm}
\noindent {\bf getgr(g) : record \hfill get group information}

\noindent
\texttt{getgr(g)} returns a record that contains group file
information for group
g, a string group name or an integer group code. If g is null, each
successive call to getgr() returns the next entry. setgrent() resets
the sequence to the beginning. Return type:\\
\ \ \texttt{record posix\_group(name, passwd, gid, members)}

\bigskip\hrule\vspace{0.1cm}
\noindent {\bf gethost(x) : record|string \hfill get host information}

\noindent
\texttt{gethost(n)} for network connection n returns a string containing
the IP number and port this machine is using for a network connection.
\texttt{gethost(s)} returns a record that contains host information for the
machine named s. If s is null, each successive call to \texttt{gethost()}
returns the next entry. \texttt{sethostent()} resets the sequence to the
beginning. The aliases and addresses are comma separated lists of
aliases and addresses (in a.b.c.d format) respectively. Its return
type is \texttt{record posix\_hostent(name, aliases, addresses)}

\bigskip\hrule\vspace{0.1cm}
\noindent {\bf getpgrp() : integer \hfill get process group}

\noindent
\texttt{getpgrp()} returns the process group of the current process. 

\bigskip\hrule\vspace{0.1cm}
\noindent {\bf getpid() : integer \hfill get process identification}

\noindent
\texttt{getpid()} produces the process identification (pid) of the current
process.

\bigskip\hrule\vspace{0.1cm}
\noindent {\bf getppid() : integer? \hfill get parent process identification}

\noindent
\texttt{getppid()} produces the pid of the parent process.

\bigskip\hrule\vspace{0.1cm}
\noindent {\bf getpw(u) : posix\_password \hfill get password information}

\noindent
\texttt{getpw(u)} returns a record that contains password file
information. \texttt{u} can
be a numeric uid or a user name. If \texttt{u} is null, each successive call to
\texttt{getpw()} returns the next entry and \texttt{setpwent()} resets
the sequence to the beginning. Return type:

\noindent
\texttt{record posix\_password(name, passwd, uid, gid, age, comment, gecos,
dir, shell)}

\bigskip\hrule\vspace{0.1cm}
\noindent {\bf getserv(string, string) : posix\_servent \hfill
	get service information}

\noindent
\texttt{getserv(s, proto)} returns a record that contains service information
for the service s using protocol proto. If s is null, each successive call
to \texttt{getserv()} returns the next entry. \texttt{setservent()}
resets the sequence to the beginning.
If proto is defaulted, it will return the first matching entry. Its
return type is \texttt{record posix\_servent(name, aliases, port, proto)}


\bigskip\hrule\vspace{0.1cm}
\noindent {\bf getuid() : string \hfill get user identity}

\noindent
\texttt{getuid()} produces the real user identity (uid) of the current process. 

\bigskip\hrule\vspace{0.1cm}
\noindent {\bf kill(integer, x) : ? \hfill kill process}

\noindent
\texttt{kill(pid, signal)} sends a signal to the process specified by pid. The
second parameter can be the string name or the integer code of the
signal to be sent.

\bigskip\hrule\vspace{0.1cm}
\noindent {\bf hardlink(string, string) : ? \hfill link files}

\noindent
\index{link!file system}\texttt{hardlink(src,dest)} creates a link
named dest that points to src.

\bigskip\hrule\vspace{0.1cm}
\noindent {\bf readlink(string) : string? \hfill read symbolic link}

\noindent
\texttt{readlink(s)} produces the filename referred to in a symbolic link
at path \texttt{s}.

\bigskip\hrule\vspace{0.1cm}
\noindent {\bf setgid(integer) : ? \hfill set group identification}

\noindent
\texttt{setgid(g)} sets the group id of the current process to g. See the UNIX
\texttt{setgid(2)} man page. 

\bigskip\hrule\vspace{0.1cm}
\noindent {\bf setgrent() : null \hfill reset group information cursor}

\noindent
\texttt{setgrent()} resets and rewinds the pointer to the group file used by
\texttt{getgr()} when \texttt{getgr()} is called with no arguments. 

\bigskip\hrule\vspace{0.1cm}
\noindent
{\bf sethostent(integer:1) : null \hfill reset host information cursor}

\noindent
\texttt{sethostent(stayopen)} resets and rewinds the pointer to the host file
used by \texttt{gethost()}. The argument defines whether the file should be kept
open between calls to \texttt{gethost()}; a nonzero value (the default) keeps it
open. 

\bigskip\hrule\vspace{0.1cm}
\noindent {\bf setpgrp() : ? \hfill set process group}

\noindent
\texttt{setpgrp()} sets the process group. This is equivalent to
\texttt{setpgrp(0, 0)} on BSD systems. 

\bigskip\hrule\vspace{0.1cm}
\noindent {\bf setpwent() : null \hfill reset password information cursor}

\noindent
\texttt{setpwent()} resets and rewinds the pointer to the password file used by
\texttt{getpw()} when \texttt{getpw()} is called with no arguments.

\bigskip\hrule\vspace{0.1cm}
\noindent {\bf setservent(integer:1) : null \hfill
	reset service information cursor}

\noindent
\texttt{setservent(stayopen)} resets and rewinds the pointer to the
services file used by \texttt{getserv()}. The argument defines whether
the file should be kept open between calls to \texttt{getserv()}; a
nonzero value (the default) keeps it open. 

\bigskip\hrule\vspace{0.1cm}
\noindent {\bf setuid(integer) : ? \hfill set user identity}

\noindent
\texttt{setuid(u)} sets the user id of the current process to
\texttt{u}. See the \texttt{setuid(2)} man page. 

\bigskip\hrule\vspace{0.1cm}
\noindent {\bf symlink(string, string) : ? \hfill symbolic file link}

\noindent
\index{link!symbolic file}symlink(src, dest) makes a symbolic link dest
that points to src.

\bigskip\hrule\vspace{0.1cm}
\noindent {\bf umask(integer) : integer \hfill file permission mask}

\noindent
\texttt{umask(u)} sets the umask of the process to \texttt{u}, an nine-bit
encoding of the read, write, and execute permissions of user,
group, and world access. See also \texttt{chmod()}. Each bit in the umask turns
\textit{off} that access, by default, for newly created files. The old
value of the umask is returned.

\bigskip\hrule\vspace{0.1cm}
\noindent {\bf wait(integer:{}-1, integer:0) : string \hfill wait for process}

\noindent
\texttt{wait(pid, options)} waits for a process given by pid to terminate or
stop. The default pid value causes the program to wait for all the
current process' children. The options parameter is an
OR of the values 1 (return if no child has exited) and 2 (return for
children that are stopped, not just for those that exit). The returned
string represents the pid and the exit status as defined in this table:

\vspace{0.05in}
\begin{center}
\begin{xtabular}{m{2.4in} m{2.6in}}
 UNIX equivalent & example of returned string \\
 WIFSTOPPED(status) & "1234 stopped:SIGTSTP" \\
 WIFSIGNALLED(status) & "1234 terminated:SIGHUP" \\
 WIFEXITED(status) & "1234 exit:1" \\
 WIFCORE(status) & "1234 terminated:SIGSEGV:core" \\
\end{xtabular}
\end{center}
\vspace{0.05in}

\noindent
Currently the \textsf{rusage} facility is unimplemented.


\section{Microsoft Windows}

Windows versions of Unicon support certain non-portable extensions to
the system interfaces. Consult Unicon Technical Report 7 for details.

\subsection*{Partial support for POSIX}

Windows supports \texttt{getpid()}, but omits other
process-related functions such as \texttt{getppid()}. On Windows
\texttt{exec()} and \texttt{system()} may only launch Windows 32-bit
.EXE binaries.

Windows Unicon supports the following signals in functions such as
\texttt{kill()}: SIGABRT, SIGBREAK, SIGFPE, SIGILL, SIGINT, SIGSEGV, and
SIGTERM.

Windows Unicon supports the \texttt{umask()} function, but ignores execute
permission and treats user/group/world identically, using the most
permissive access specified.

\subsection*{Native user interface components}

Windows Unicon supports limited access to platform-native user interface
components and multimedia controls.

\texttt{WinButton(w,s,x,y,wd,ht)} installs a pushbutton with label s on window w.

\texttt{WinColorDialog(w, s)} allows the user to choose a color for a
window's context.

\texttt{WinEditRegion(w, s, s2, x, y, wd, ht)} installs an edit box with label s.

\texttt{WinFontDialog(w, s)} allows the user to choose a font for a
window's context.

\texttt{WinMenuBar(w, L1, L2,...)} installs a set of top-level menus.

\texttt{WinOpenDialog(w, s1, s2, i, s3, j, s4)} allows the user to
 choose a file to open.

\texttt{WinPlayMedia(w, x[])} plays a multimedia resource.

\texttt{WinSaveDialog(w, s1, s2, i, s3, j, s4)} allows the user to
 choose a file to save.

\texttt{WinScrollBar(w, s, i1, i2, i3, x, y, wd, ht)} installs a scrollbar.

\texttt{WinSelectDialog(w, s1, buttons)} allows the user to select from a set of
choices.

