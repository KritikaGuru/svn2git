\chapter{Internet Programs}

The \index{Internet}Internet is central to modern computing. Because
it is ubiquitous, programmers should be able to take it for granted.
Writing applications that use the Internet should be just as easy
as writing programs for a standalone desktop computer. In many respects
this ideal can be achieved in a modern programming language. The core
facilities for Internet programming were introduced with simple
examples as part of the system interface in Chapter 5. This chapter
expands on this important area of software development. This chapter
presents examples that show you how to
\begin{itemize}\itemsep0pt
\item Write Internet servers and clients
\item Build programs that maintain a common view of multiple
users' actions
\end{itemize}

\section{The Client-Server Model}

The Internet allows applications to run on multiple connected
computers using any topology, but the standard practice is to implement a
\index{client/server}client/server topology in which a
user's machine plays the role of a client, requesting
information or services from remote machines, each of which plays the
role of a server. The relationship between clients and servers is
many-to-many, since one client can connect to many servers and one
server typically handles requests from many clients.

Writing a \index{client}client can be easy. For simple read-only
access to a remote file, it is just as easy as opening a file on the
hard disk. Most clients are more involved, sending out requests
and receiving replies in some agreed-upon format called a
\index{protocol}\textit{protocol}. A protocol may be human readable
text or it may be binary, and can consist of any number of messages
back and forth between the client and server to transmit the required
information. The most common Internet protocols are built-in parts of
Unicon's messaging facilities, but some applications
define their own protocol.

Writing a server is more difficult. A server sits around in
an infinite loop, waiting for clients and servicing their requests.
When only one client is invoking a server, its job is simple enough,
but when many simultaneous clients wish to connect, the server program
must either be very efficient or else the clients will be kept waiting
for unacceptably long periods.

Although the following example programs emphasize how easy it is to
write Internet clients and servers in Unicon, writing
"industrial strength" applications requires
additional security considerations which are mostly beyond the scope of
this book. For example, user authentication and encryption are
essential in most systems, and many modern servers are carefully tuned
to maximize the number of simultaneous users they support, and minimize
their vulnerability to denial-of-service attacks.

\section{An Internet Scorecard Server}

Many games with numeric scoring systems feature a list of high scores.
This feature is interesting on an individual machine, but it is ten
times as interesting on a machine connected to the Internet! The
following simple \index{server}server program allows games to report
their high scores from around the world. This allows players to compete
globally. The scorecard server is called \texttt{scored}. By
convention, servers are often given names ending in
"d" to indicate that they are daemon
programs that run in the background.

\subsection*{The scorecard client procedure}

Before examining the server code, take a look at the client procedure
that a game calls to communicate with the \texttt{scored} server. To
use this client procedure in your programs, add the following
declaration to your program.

\iconcode{
link highscor
}

The procedure \texttt{highscore()} opens a network connection, writes
four lines consisting of the protocol name
"HSP", the name of the game, the
user's identification (which could be a nickname, a
number, an e-mail address, or anything else), and that
game's numeric score. Procedure \texttt{highscore()}
then reads the complete list of high scores from the server, and
returns the list. Most games write the list of high scores to a window
for the user to ponder.

\iconcode{
procedure highscore(game, userid, score, server) \\
\>   if not find(":", server) then server
{\textbar}{\textbar}:= ":4578" \\
\>   f := open(server, "n") {\textbar}
fail \\
\ \\
\>   \# Send in this game's score \\
\>   write(f, "HSP{\textbackslash}n", game,
"{\textbackslash}n", userid,
"{\textbackslash}n", score) {\textbar} \\
\>   \ \ \ stop("Couldn't write:
", \&errortext) \\
\ \\
\>   \# Get the high score list \\
\>   L := ["High Scores"] \\
\>   while line := read(f) do put(L, line) \\
\>   close(f) \\
\>   return L \\
end
}

\subsection*{The Scorecard server program}

The scorecard server program, \texttt{scored.icn} illustrates issues
inherent in all Internet servers. It must sit at a port, accepting
connection requests endlessly. For each connection, a call to
\texttt{score\_result()} handles the request. The \texttt{main()}
procedure given below allows the user to specify a port, or uses a
default port if none is supplied. If another server is using a given
port, it won't be available to this server, and the
client and server have to agree on which port the server is using.

\iconcode{
procedure main(av) \\
\>   port := 4578 \# a random user-level port \\
\>   if av[i := 1 to *av] == "-port" then
        port := integer(av[i+1]) \\
\ \\
\>   write("Internet Scorecard version
1.0") \\
\>   while net := open(":"
{\textbar}{\textbar} port, "na") \ do \{ \\
\>   \ \ \ score\_result(net) \\
\>   \ \ \ close(net) \\
\>   \ \ \ \} \\
\>   (\&errno = 0) {\textbar} stop("scored net accept
failed: ", \&errortext) \\
end
}

The procedure \texttt{score\_result()} does all the real work of the
server, and its implementation is of architectural significance. Any
delay in handling a request implies the server will be unable to
handle other simultaneous client requests. For this reason, many servers
immediately spawn a separate process to handle each request.
You could do that with \texttt{system()}, as illustrated in Chapter 5,
or launch a thread for it,
but for \texttt{scored} this is overkill. The server handles each
request almost instantaneously.

Some small concessions to security are in order, even in a trivial
example such as this. If a bogus Internet client connects by accident,
it will fail to identify our protocol and be rejected. More subtly, if
a rogue client opens a connection and writes nothing, we do not want to
block waiting for input or the client will deny service to others. A
call to \texttt{select()} is used to guarantee the server receives data within
the first 1000 milliseconds (1 second). A last security concern is to
ensure that the "game" filename supplied is
valid; it must be an existing file in the current directory, not
something like \texttt{/etc/passwd} for example. 

The \texttt{score\_result()} procedure maintains a static table of all
scores of all games that it knows about. The keys of the table are the
names of different games, and the values in the table are lists of
alternating user names and scores. The procedure starts by reading the
game, user, and score from the network connection, and loading the
game's score list from a local file, if it
isn't in the table already. Both the score lists
maintained in memory, and the high scores files on the server, are
sequences of pairs of text lines containing a userid followed by a
numeric score. The high score files have to be created and initialized
manually with some N available (userid,score) pairs of lines, prior to
their use by the server.

\iconcode{
procedure score\_result(net) \\
\>   local s := "" \\
\>   static t, gamenamechars \\
\>   initial \{ \\
\>\>    t := table() \\
\>\>    gamenamechars :=
\&letters++\&digits++'-\_' \\
\>\>    \} \\
\ \\
\>   select(net, 1000) {\textbar} \{ write(net,
"timeout"); fail \} \\

\>   (s ||:= ready(net)) ? \{ \\
\>\>     = "HSP{\textbackslash}n" {\textbar} \{
write(net, "wrong protocol"); fail \} \\
\>\>     game := tab(many(gamenamechars)) {\textbar} \{
                  write(net,"no game?"); fail \} \\
\>\>     = "{\textbackslash}n" \\
\>\>     owner := tab(many(gamenamechars)) {\textbar} \{
                  write(net,"no owner?"); fail \} \\
\>\>     = "{\textbackslash}n" \\
\>\>     score := tab(many(\&digits)) {\textbar} \{
                  write("no score?"); fail \} \\
\>\>     \} \\
\ \\
\>   if t[game] === \&null then \{ \\
\>\>    if not (f := open(game)) then \{ \\
\>\>\>     write(net, "No high scores here for ", game) \\
\>\>\>     fail \\
\>\>\>     \} \\
\>\>    t[game] := L := [] \\
\>\>    while put(L, read(f)) \\
\>\>    close(f) \\
\>\>    \} \\
\>   else \\
\>   \ \ \ L := t[game]
}

The central question is whether the new score makes an entry into the
high scores list or not. The new score is checked against the last
entry in the high score list, and if it is larger, it replaces that
entry. It is then "bubbled" up to the
correct place in the high score list by repeatedly comparing it with
the next higher score, and swapping entries if it is higher. If the new
score made the high score list, the list is written to its file on
disk.

\iconcode{
\>   if score {\textgreater} L[-1] then \{ \\
\>   \ \ \ L[-2] := owner \\
\>   \ \ \ L[-1] := score \\
\>   \ \ \ i := -1 \\
\>   \ \ \ while L[i] {\textgreater} L[i-2] do \{ \\
\>   \ \ \ \ \ \ L[i] :=: L[i-2] \\
\>   \ \ \ \ \ \ L[i-1] :=: L[i-3] \\
\>   \ \ \ \ \ \ i -:= 2 \\
\>   \ \ \ \ \ \ \} \\
\>   \ \ \ f := open(game,"w") \\
\>   \ \ \ every write(f, !L) \\
\>   \ \ \ close(f) \\
\>   \ \ \ \}
}

{\sffamily\bfseries
Note}

{\sffamily
List \textit{L} and \textit{t[game]} refer to the same list, so the
change to L here is seen by the next client that looks at
\textit{t[game]}.}

Lastly, whether the new score made the high score list or not, the high
score list is written out on the network connection so that the game
can display it.

\iconcode{
\>   every write(net, !L) \\
end
}

Is this high score application useful and fun? Yes! Is it secure and
reliable? No! It records any scores it is given for any game that has a
high score file on the server. It is utterly easy to supply false
scores. This is an honor system.

\section{A Simple ``Talk'' Program}

E-mail is the king of all Internet applications. After that, some of the
most popular Internet applications are real-time dialogues between
friends and strangers. Many on-line services rose to popularity because
of their "chat rooms," and Internet Relay
Chat (\index{IRC}IRC) is a ubiquitous form of free real-time
communication. These applications are evolving in multiple directions,
such as streaming multimedia, and textual and graphical forms of
interactive virtual reality. While it is possible to create arbitrarily
complex forms of real-time communication over the Internet, for many
purposes, a simple connection between two users'
displays, with each able to see what the other types, is all that is
needed.

The next example program, called \texttt{italk}, is styled after the
classic BSD UNIX \index{talk}\texttt{talk} program. The stuff you type
appears on the lower half of the window, and the remote
party's input is in the upper half. Unlike a chat
program, the characters appear as they are typed, instead of a line at
a time. In many cases this allows the communication to occur more
smoothly with fewer keystrokes.

The program starts out innocently enough, by linking in library
functions for graphics, defining symbolic constants for font and screen
size. Among global variables, \texttt{vs} stands for vertical space,
\texttt{cwidth} is column width, \texttt{wheight} and \texttt{wwidth}
are the window's dimensions, and \texttt{net} is the
Internet connection to the remote machine.

\iconcode{
link graphics
\ \\
\$define ps 10 \ \ \ \ \ \ \ \ \ \ \ \ \ \ \ \# The size of the font to
use \\
\$define lines 48 \ \ \ \ \ \ \ \ \ \ \ \ \# No. of text lines in the
window \\
\$define margin 3 \ \ \ \ \ \ \ \ \ \ \ \ \# Space to leave around the
margins \\
\$define START\_PORT 1234
\$define STOP\_PORT \ 1299
\ \\
global vs, cwidth, wheight, wwidth, net
}

The \texttt{main()} procedure starts by calling \texttt{win\_init()} and
\texttt{net\_init()} to open up a local window and then establish a
connection over the network, respectively. The first command line
argument is the user and/or machine to connect to.

\iconcode{
procedure main(args) \\
\>   win\_init() \\
\>   net\_init(args[1] {\textbar} "127.0.0.1")
}

Before describing the window interaction or subsequent handling of
network and window system events, consider how \texttt{italk}
establishes communication in procedure \texttt{net\_init()}. Unlike
many Internet applications, \texttt{italk} does not use a conventional
client/server architecture in which a server daemon is always running
in the background. To connect to someone on another machine, you name
him or her in the format \texttt{user@host} on the command line. The
code attempts to connect as a client to someone waiting on the other
machine, and if it fails, it acts as a server on the local machine and
waits for the remote party to connect to it.

\iconcode{
procedure net\_init(host) \\
\>   host ?:= \{ \\
\>   \ \ \ if user := tab(find("@")) then move(1) \\
\>   \ \ \ tab(0) \\
\>   \ \ \ \} \\
\>   net :=
(open\_client{\textbar}open\_server{\textbar}give\_up)(host, user) \\
end
}

The attempt to establish a network connection begins by attempting to
open a connection to an \texttt{italk} that is already running on the
remote machine. The \texttt{italk} program works with an arbitrary
user-level set of ports (defined above as the range 1234-1299). An
\texttt{italk} client wades through these ports on the remote machine,
trying to establish a connection with the desired party. For each port
at which \texttt{open()} succeeds, the client writes its user name,
reads the user name for the process on the remote machine, and returns
the connection if the desired party is found.

\iconcode{
procedure open\_client(host, user) \\
\>   port := START\_PORT \\
\>   if {\textbackslash}user then \{ \\
\>   \ \ \ while net := open(host {\textbar}{\textbar}
":" {\textbar}{\textbar} port,
"n") do \{ \\
\>   \ \ \ \ \ \ write(net, getenv("USER")
{\textbar} "anonymous") \\
\>   \ \ \ \ \ \ if user == read(net) then return net \\
\>   \ \ \ \ \ \ close(net) \\
\ \ \ \ \ \ \ \ \ port +:= 1 \\
\>   \ \ \ \ \ \ \} \\
\>   \ \ \ \} \\
\>   else \{ \\
\>   \ \ \ net := open(host {\textbar}{\textbar}
":" {\textbar}{\textbar} port,
"n") \\
\>   \ \ \ write(net, getenv("USER")
{\textbar} "anonymous") \\
\>   \ \ \ read(net) \# discard \\
\>   \ \ \ return net \\
\>   \ \ \ \} \\
end
}

The procedure \texttt{open\_server()} similarly cycles through the
ports, looking for an available one on which to wait. When it receives
a connection, it checks the client user and returns the connection if
the desired party is calling.

\iconcode{
procedure open\_server(host, user) \\
\>   repeat \{ \\
\ \ \ \ \ \ port := START\_PORT \\
\>   \ \ \ until net := open(":"
{\textbar}{\textbar} port, "na") do \{ \\
\>   \ \ \ \ \ \ port +:= 1 \\
\>   \ \ \ \ \ \ if port {\textgreater} STOP\_PORT then fail \\
\>   \ \ \ \ \ \ \} \\
\>   \ \ \ if not (them := read(net)) then \{ \\
\>   \ \ \ \ \ \ close(net) \\
\>   \ \ \ \ \ \ next \\
\>   \ \ \ \ \ \ \} \\
\>   \ \ \ if /user {\textbar} (them == user) then \{ \\
\>   \ \ \ \ \ \ write(net, getenv("USER")
{\textbar} "anonymous") \\
\>   \ \ \ \ \ \ WAttrib("label=talk: accepted call from
", them) \\
\>   \ \ \ \ \ \ return net \\
\>   \ \ \ \ \ \ \} \\
\>   \ \ \ WAttrib("label=talk: rejected call from
", them) \\
\>   \ \ \ write(net, getenv("USER")
{\textbar} "anonymous") \\
\>   \ \ \ close(net) \\
\>   \ \ \ \} \\
end
}

This connection protocol works in the common case, but is error prone.
For example, if both users typed commands at identical instants, both
would attempt to be clients, fail, and then become servers awaiting the
other's call. Perhaps worse from some
users' point of view would be the fact that there is
no real authentication of the identity of the users. The \texttt{italk}
program uses whatever is in the USER \index{environment
variable!USER}environment variable. The UNIX talk program solves both
of these problems by writing a separate talk server daemon that
performs the \textit{marshalling}. The daemons talk across the network
and negotiate a connection, check to see if the user is logged in and
if so, splash a message on the remote user's screen
inviting her to start up the talk program with the first
user's address.

The next part of \texttt{italk}'s code to consider is
the event handling. In reality, each \texttt{italk} program manages and
multiplexes asynchronous input from two connections: the window and the
network. The built-in function that is used for this purpose is
\texttt{select()}. The \texttt{select()} function will wait until some
(perhaps partial) input is available on one of its file, window, or
network arguments.

The main thing to remember when handling input from multiple
sources is that you must not block for I/O. This means: use listener
mode for new connections or a timeout parameter with \texttt{open()},
and when handling network connections, never use \texttt{read()},
only use \texttt{reads()} or better yet \texttt{ready()}.
For windows you must also avoid
\texttt{read()}'s library procedure counterpart,
\texttt{WRead()}. The code below checks which connection has input
available and calls \texttt{Event()} as events come in on the window,
and calls \texttt{reads()} on the network as input becomes available on
it. In either case the received input is echoed to the correct location
on the screen. Ctrl-D exits the program. To accept a command such as
"quit" would have meant collecting
characters till you have a complete line, which seems like overkill for
such a simple application.

\iconcode{
\>   repeat \{ \\
\>   \ \ \ *(L := select(net, \&window)){\textgreater}0 {\textbar}
stop("empty select?") \\
\>   \ \ \ if L[1] === \&window then \{ \\
\>   \ \ \ \ \ \ if \&lpress {\textgreater}= integer(e := Event())
{\textgreater}= \&rdrag then next \\
\>   \ \ \ \ \ \ if string(e) then \{ \\
\>   \ \ \ \ \ \ \ \ \ writes(net, e) {\textbar} break \\
\>   \ \ \ \ \ \ \ \ \ handle\_char(2, e) {\textbar} break \\
\>   \ \ \ \ \ \ \ \ \ WSync() \\
\>   \ \ \ \ \ \ \ \ \ \} \\
\>   \ \ \ \ \ \ \} \\
\>   \ \ \ else \{ \\
\>   \ \ \ \ \ \ s := reads(net) {\textbar} break \\
\>   \ \ \ \ \ \ handle\_char(1, s) {\textbar} break \\
\>   \ \ \ \ \ \ WSync() \\
\>   \ \ \ \ \ \ \} \\
\>   \ \ \ \} \\
\>   close(net) \\
end
}

After such a dramatic example of input processing, the rest of the
\texttt{italk} program is a bit anticlimactic, but it is presented
anyhow for completeness sake. The remaining procedures are all
concerned with managing the contents of the user's
window. Procedure \texttt{handle\_char(w, c)}, called from the input
processing code above, writes a character to the appropriate part of
the window. If \texttt{w = 1} the character is written to the upper
half of the window. Otherwise, it is written to the lower half. The two
halves of the window are scrolled separately, as needed.

\iconcode{
procedure handle\_char(w, c) \\
\>   \# Current horiz. position for each half of the window \\
\>   static xpos \\
\>   initial xpos := [margin, margin]
\ \\
\>   if c == "{\textbackslash}\^{}d" then
fail \ \ \ \ \# EOF \\
\ \\
\>   \# Find the half of the window to use \\
\>   y\_offset := (w - 1) * wheight/2 \\
\ \\
\>   if c ==
("{\textbackslash}r"{\textbar}'{\textbackslash}n')
{\textbar} xpos[w] {\textgreater} wwidth then \{ \\
\>   \ \ \ ScrollUp(y\_offset+1, wheight/2-1) \\
\>   \ \ \ xpos[w] := margin \\
\>   \ \ \ \} \\
\>   if c ==
("{\textbackslash}r"{\textbar}'{\textbackslash}n')
then return \\
\>   \#handles backspacing on the current line \\
\>   if c =="{\textbackslash}b" then \{ \\
\>   \ \ \ if xpos[w] {\textgreater}= margin + cwidth then \{ \\
\>   \ \ \ \ \ \ EraseArea(xpos[w]-cwidth, y\_offset+1+wheight/2-1-vs,cwidth,vs) \\
\>   \ \ \ \ \ \ xpos[w] -:= cwidth \\
\>   \ \ \ \ \ \ return \\
\>   \ \ \ \ \ \ \} \\
\>   \ \ \ \} \\
\>   DrawString(xpos[w], wheight/2 + y\_offset - margin, c) \\
\>   xpos[w] +:= cwidth \\
\>   return \\
end
}

Scrolling either half of the window is done a line at a time. The
graphics procedure \texttt{CopyArea()} is used to move the existing
contents up one line, after which \texttt{EraseArea()} clears the line
at the bottom.

\iconcode{
procedure ScrollUp(vpos, h) \\
\>   CopyArea(0, vpos + vs, wwidth, h-vs, 0, vpos) \\
\>   EraseArea(0, vpos + h - vs, wwidth, vs) \\
end
}

The window is initialized with a call to the library procedure,
\texttt{WOpen()}, which takes attribute
parameters for the window's size and font. These
values, supplied as defined symbols at the top of the program, are also
used to initialize several global variables such as \texttt{vs}, which
gives the vertical space in pixels between lines.

\iconcode{
procedure win\_init() \\
\>   WOpen("font=typewriter,"
{\textbar}{\textbar} ps, "lines="
{\textbar}{\textbar} lines, "columns=80") \\
\>   wwidth := WAttrib("width") \\
\>   wheight := WAttrib("height") \\
\>   vs := WAttrib("fheight") \\
\>   cwidth := WAttrib("fwidth") \\
\>   DrawLine(0, wheight/2, wwidth, wheight/2) \\
\>   Event() \\
end
}

Lastly, the procedure \texttt{give\_up()} writes a message and exits the
program, if no network connection is established. If \texttt{user} is
null and the non-null test (the backslash operator) fails, the
concatenation is not performed and \index{alternation operator (
{\textbar} )}alternation causes the empty string to be passed as the
second argument to \texttt{stop()}.

\iconcode{
procedure give\_up(host, user) \\
\>   stop("no connection to ",
({\textbackslash}user {\textbar}{\textbar}
"@") {\textbar}
"", host) \\
end
}

What enhancements would make \texttt{italk} more interesting? An
obvious extension would be to use a standard network protocol,
such as that of UNIX \texttt{talk}, so that \texttt{italk} could
communicate with other users that don't have
\texttt{italk}. UNIX \texttt{talk} also offers a more robust connection
and authentication model (although you are dependent on the
administrator of a remote machine to guarantee that its \texttt{talkd}
server is well behaved). Another feature of UNIX \texttt{talk} is
support for multiple simultaneously connected users.

One neat extension you might implement is support for
graphics, turning \texttt{italk} into a distributed whiteboard
application for computer-supported cooperative work. To support
graphics you would need to extend the window input processing to
include a simple drawing program, and then you would need to extend the
network protocol to include graphics commands, not just keystrokes. One
way to do this would be to represent each user action (a keystroke or a
graphics command) by a single line of text that is transmitted over the
network. Such lines might look like:

\iconcode{
key H \\
key i \\
key ! \\
circle 100,100,25
}

\noindent
and so forth. At the other end, the program deciphering these commands
translates them into appropriate output to the window, which would be
pretty easy, at least for simple graphics. The nice part about this
solution is that this particular collaborative whiteboard application
would work fine across differing platforms (Linux, Microsoft Windows,
and so on) and require only a couple hundred lines of code!

\section{Summary}

Writing Internet programs can be easy and fun, although it is easy to
underestimate the security needed. There are several
different ways to write Internet programs in Unicon. The
\index{database}database interface presented in Chapter 6 allows you to
develop client/server applications without \textit{any} explicit
network programming when the server is a database. A \index{SQL}SQL
server is overkill for many applications such as the high score server,
and it is not appropriate for other non-database network applications
such as the \texttt{italk} program.

For these kinds of programs, it is better to "roll your
own" network application protocol. Once a connection is
established (perhaps using a client/server paradigm), the actual
communication between programs is just as easy as file input and
output. If you do roll your own network application, keep the protocol
simple; it is easy enough to write yourself into deadlocks, race
conditions, and all the other classic situations that make parallel and
distributed programming perilous.

