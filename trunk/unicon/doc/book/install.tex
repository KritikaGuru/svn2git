\chapter{Installation}

The Downloads page of the Unicon web site (http://unicon.org) has
links to the binary distributions of Unicon and to the source code.
Unicon may be installed from a binary distribution for Intel based
Windows platforms. Users on other platforms will usually have to
download the source (using subversion) and build it from scratch.
The subversion web site http://subversion.apache.org has details on
how to get a subversion client if your system does not already have one.

\section{Building on Linux}

Your Linux software repository will probably have a
subversion client package as an easy install if you
do not have svn already.

\section{Building on OS X}

OS X comes with a subversion client but you will need to download the
Xcode development package from http://developer.apple.com/xcode/download 
to obtain a C compiler.  If you want access to the graphics
facilities of Unicon, you will also need to download and install the
XQuartz package from http://xquartz.macosforge.org.

Once the pre-requisites are in place, and any configuration has been
done, Unicon may be built by following the instructions in the top
level README file.

\section{Building on Windows}

If the prebuilt binary package is not suitable (for example, a feature
you need is not enabled) then Unicon can be built from scratch. You
will need a C compiler (gcc is the usual choice, although clang is also
available on Windows). The standard Unicon build scripts will not
work from the command prompt, so you will also need to install a
``unix-like'' command line environment: either Msys or cygwin are
known to work. The Unicon web site has a link
(http://www2.cs.uidaho.edu/~jeffery/win32) to some other utilities
that you may need.

Once the pre-requisites are in place, and any configuration has been
done, Unicon may be built by following the instructions in the top
level README file.
