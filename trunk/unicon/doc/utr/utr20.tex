\documentclass[letterpaper,12pt]{article}
\usepackage{utr}

\title{The Unicon JSON Library}
\author{Gigi Young, Clinton Jeffery}
\trnumber{20}
\date{\today}

\begin{document}
\abstract{
This report describes a library for supporting conversion to and from 
JSON (JavaScript Object Notation) and their equivalent Unicon structures.
}
\maketitle

\section{Introduction}

JSON is a widely used lightweight data-interchange format [Crawford]
that supports nested, sequential and associative data structures.
This report describes two Unicon library
functions that allow for convenient data conversion
between JSON and Unicon.

\section{An Overview of the JSON Grammar}

The JSON grammar is straightforward. A JSON data structure is composed
of one or more JSON values, which can be either a string, number,
boolean true, boolean false, null, array, or ``object''. An array contains
a list of JSON values, while an ``object'' is a set of string-value
pairs. In other words, an array is a list of values and an object is a
table (dictionary) that requires keys to be strings.
Arrays and objects may be arbitrarily nested within each other.

The manner in which the tokens of the JSON grammar are defined allows
a token's category to be determined from its first character. The JSON
tokens are string, number, true, false, null, and the operators used
for array, object, and their productions (\verb {}[]:, ). The
following table gives the possible first characters of each
non-operator JSON token.

\begin{center}
\begin{tabular}{ c | c }
 possible first characters  & JSON token \\
 \hline
 \{                         & object \\
 \verb [                    & array \\
 -0123456789                & number \\
 $\prime\prime$             & string \\
 t                          & true \\
 f                          & false \\
 n                          & null \\

\end{tabular}
\end{center}

\pagebreak

\section{The Unicon JSON API}

The JSON library contains two thread-safe functions, {\tt jtou()} and 
{\tt utoj()}, for converting from JSON to Unicon and Unicon to JSON, 
respectively. The data equivalencies are as follows: 

\begin{center}
{\large\bf JSON to Unicon}
\end{center}

\begin{center}
\begin{tabular}{ c | c }
 JSON      & Unicon \\
 \hline
 object    & table \\
 array     & list \\
 string    & string \\
 number    & int or real \\
 true      & string - {\tt "\_\_true\_\_"} \\
 false     & string - {\tt "\_\_false\_\_"} \\
 null      & null value - {\tt \&null} \\

\end{tabular}
\end{center}

\vspace{1em}

\begin{center}
\center{\large\bf Unicon to JSON}
\end{center}

\begin{center}
\begin{tabular}{ c | c }
 Unicon                   & JSON\\
 \hline
 table, record, or class  & object \\
 list or set              & array  \\
 string                   & string \\
 int or real              & number \\
 string - {\tt "\_\_true\_\_"}  & true   \\
 string - {\tt "\_\_false\_\_"} & false  \\
 null value - {\tt \&null}& null   \\

\end{tabular}
\end{center}

% Horizontal line spanning width of the page
\noindent\rule{\textwidth}{1pt}

\begin{description}

\item[{\tt\bf jtou(s)}] converts JSON to Unicon.
  If parameter {\tt s} is a legal JSON string, it is interpreted as such;
  otherwise, it is treated as a filename and its JSON contents are converted.

  It should noted that a JSON-formatted Unicon string must contain escaped
  double quotes due to the fact that double quotes define a string literal
  in the language. A sample string is as follows:

  \begin{verbatim}
    s := "{\"one\":1, \"two\":2, \"three\":3 }"
  \end{verbatim}
\vspace{-1em}

\item[{\tt\bf jtou(s,mode)}] forces conversion as a filename (mode {\tt "f"})
  or a string (mode {\tt "s"}).
  So for example {\tt jtou("x.json","f")} would read from file {\tt x.json}
  while {\tt jtou("[1,2,{$\backslash$"three$\backslash$":3}]", "s")} would 
  convert a string.

\item[{\tt\bf jtou(s,mode,error)}] specifies that in the case of a conversion
  error, the error is written to a file handle specifed by {\tt error}.
  This can be done with or without the optional second parameter as 
  {\tt jtou(s,,file)} or {\tt jtou(fname,"f",file)}, where {\tt file}
  can be any of the standard output streams or an open file
  handle.

\vspace{1em}

{\tt jtou()} is a generator that produces Unicon data one-by-one until
all JSON data is converted or an error occurs. If a JSON data contains 
one structure,
then one output will be generated. If a JSON data contains multiple 
structures, the outputs will appear in the order found in the JSON data.
To conform with other JSON libraries, {\tt jtou()} allows for single
trailing commas in JSON arrays and objects.

In the following program, if we convert a JSON data structure which
consists of an object and an array, the first result produced and
placed into {\tt u} is a table of size three and the second result
is a list of size four.

  \begin{verbatim}
  link json
  procedure main()
     u := [: jtou("{\"one\":1, \"two\":2, \"three\":3}_
         [true, false, null, 1.23e3]") :]
     x := u[1]
     y := u[2]
     write(x["one"])
     write(x["two"])
     write(x["three"])
     every i := 1 to *y do write(y[i])
  end
  \end{verbatim}

produces the output:

  \begin{verbatim}
  1
  2
  3
  __true__
  __false__

  1230.0
  \end{verbatim}


Another example is 

  \begin{verbatim}
  link json
  procedure main()
     write( jtou("\"this is a string\"") )
  end
  \end{verbatim}

for which the output is:

  \begin{verbatim}
  this is a string
  \end{verbatim}

\end{description}

\noindent\rule{\textwidth}{1pt}
\begin{description}

\item[{\tt\bf utoj(u)}] takes a Unicon value, {\tt u}, and returns a 
  corresponding JSON-formatted string. If no such equivalent structure exists,
  then the function fails. 

\item[{\tt\bf utoj(u,error)}] the optional second parameter, {\tt error} 
  takes a file handle which is written to in case of a conversion error and
  function failure, e.g. {\tt utoj(u,\&errout)} or {\tt utoj(u,file)}

\vspace{1em}

Here are some examples of primitive data conversions from Unicon to JSON

  \begin{verbatim}
  link json
  procedure main()
     write(utoj("this is a string"))
     write(utoj(100))
     write(utoj("__true__"))
     write(utoj("__false__"))
     write(utoj(&null))
  end
  \end{verbatim}

which produces the output:

  \begin{verbatim}
  "this is a string"
  100
  true
  false
  null 
  \end{verbatim}

More frequently, JSON will come into play when lists or tables are
involved. Here is an example of converting a Unicon table to a JSON object.

  \begin{verbatim}
  link json
  procedure main()
     T := table()
     T["one"] := 1
     T["two"] := 2
     T["list"] := [1,2,3]
     T["table"] := table()
     write("A table: ", utoj(T))
  end
  \end{verbatim}

This program writes out:

  \begin{verbatim}
  A table: {"two":2,"table":{},"one":1,"list":[1,2,3]}
  \end{verbatim}

The extraction of keys from Unicon tables occurs in a random manner,
resulting in JSON data that does not follow the order in which the Unicon
table was constructed. However, lists are sequential in Unicon, so the
resulting ordering of the JSON array is retained. 

A class conversion is as follows:

  \begin{verbatim}
  link json

  class Person(name, age, gender)
     #
     # some methods here
     #
  end

  procedure main()
     person := Person("Joe", 42, "Male")
     write("A Person: ", utoj(person))
  end
  \end{verbatim}

this outputs:

  \begin{verbatim}
  {"name":"Joe","age":42,"gender":"Male"}
  \end{verbatim}


\end{description}

\noindent\rule{\textwidth}{1pt}

%\section{Error Handling}

%The Unicon JSON library provides error handling in the form of a global
%class instance, \verb __jerror . To retrieve the latest error, the
%programmar may use the
%method \verb __jerror.get_err() . This returns
%the last error, if it exists, as a string. If no error has occurred, 
%\verb __jerror.get_err()  will return the empty string. 

%If the parameter, \verb s , passed to \verb jtou(s)  is a string containing 
%JSON data, the error message will contain the first error encountered. If
%\verb s  is a file, then the error message will contain the filename and
%the line number where the error was detected, followed by the error.

\section*{References}

[Crawford] anonymous, but owing to Douglas Crawford.
"Introducing JSON", json.org.

\end{document}
