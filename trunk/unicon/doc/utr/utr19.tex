\documentclass[letterpaper]{article}
\usepackage[ascii]{inputenc}
\usepackage[T1]{fontenc}
\usepackage[english]{babel}
\usepackage{utr}
\input icon.sty

\title{Adding New Diagnostic Syntax Error Messages to Unicon}
\author{Clinton Jeffery}
\trnumber{19}
\date{2017-05-26}

\begin{document}
\abstract{
Unicon uses a parser generated by iyacc, a modified version of
Berkeley YACC. Syntax error messages are produced using the same
two keys that YACC uses to determine what to do at each step:
the parse state and the current token.  Whenever the language
grammar is modified, these parse states change in unobvious ways.
To solve the problem of mapping changing parse states and tokens
onto diagnostic messages,
a tool named merr is used on a set of example error fragments to
associate messages with corresponding parse states and tokens.
This technical report describes the process of adding new diagnostic
messages.
}
\maketitle

\section{Introduction}

This technical report describes how to add new syntax error messages
to the Unicon translator. When it encounters a syntax error with which
no message is associated, Unicon emits a generic message along the
lines of the following.

\iconcode{
synerrfiletest.icn:32: \# "end" syntax error (232;272)
}

The two integers in parentheses, separated by a semi-colon, are the parse
state and the lexical token at the point of error.  It is these two
integers that YACC uses to determine that there is a syntax error, and
these two integers that Unicon uses to decide what message to write.


\section{A Brief History}

Unicon's grammar descends from Icon, and Icon's syntax error messages
are keyed off YACC parse states, which are integers internal and
specific to whichever YACC implementation is used.  For example, Icon
traditionally used AT\&T YACC. If its grammar were run through GNU
Bison, the integers would be different and the syntax errors emitted
would become nonsensical.

In addition to this portability problem, any change to the
Icon grammar meant a painful and mysterious task of updating a
manually-maintained table that mapped parse states to error messages.
The difficulty of this task was severe enough to reinforce the
project owner's reluctance to change the syntax, constraining research
directions albeit adding ``language stability through unmaintainability''.

A custom tool called {\em merr\/} was developed for Unicon in order to
solve the problem of updating the syntax error message table[1].  Along
the way, {\em merr\/} added the ability to disambiguate error
messages for a given parse state, depending on the current input token.
Compared with Icon, Merr also frees developers to use whichever
YACC implementation they wish (AT\&T YACC, Berkeley YACC, or Bison,
for example). In the case of Unicon, there was no option of sticking
with Icon's error message table even if it could have been updated to\
handle Unicon's additions, as AT\&T YACC is proprietary, and
iyacc is based on Berkeley YACC. It was essential that a new error
message table be created, and {\em merr\/} provided the missing piece.

\section{Running Merr During the Unicon Translator Build Process}

Merr is included in Unicon source distributions. It is simply named
merr.icn in the unicon translator directory, unicon/uni/unicon.
A Merr executable is built by saying "unicon merr" in that directory.

The makefile rules for running Merr during the Unicon translator build
are included in the unicon/uni/unicon makefile, but are commented out.
In order to run merr, the lines

\iconcode{
\#yyerror.icn: unigram.icn \\
\# \>\>\>	merr unicon
}

\noindent must be uncommented.

Adding these makefile rules will cause merr to run, and generate a new
yyerror.icn whenever unigram.icn is changed, generally due to iyacc
being run on unigram.y.  Unfortunately, the Unicon translator uses
some slight modifications to the generic yyerror.icn that {\em merr\/}
generates. As a result, the yyerror.icn should be hand-modified after
merr is run, to fix things up as follows.

\begin{itemize}
\item write() should be changed to iwrite()
\item the parse errors should be bundled onto global list
      parsingErrors instead of being written out directly.
\end{itemize}

These changes facilitate the use of the Unicon parser in the IDE, both
directly (as part of syntax checking) and indirectly, so that error
messages show up in the IDE error box and not just on standard error
output, which is easily missed in a GUI application.

\section{Adding New Error Messages}

Error fragments, and their associated error messages, are stored in a
file named meta.err in the unicon/uni/unicon directory.  The existing
file contains about 75 error fragments, and was created by starting
with all the Icon error message fragments, and adding messages
pertaining to Unicon-specific syntax such as class declarations.

The format of the fragments and messages is

\iconcode{
{\em fragment\/} \\
::: {\em message\/}
}

\noindent which is to say: one or more lines of source code that will
produce a syntax error, ending with a line consisting of three colons
followed by the syntax error message to use when that fragment (or any
source code resulting in the same parse state) occurs.  If multiple
fragments result in the same parse state, the first such fragment
gives the default syntax error message, but other fragments with that
parse state will give messages for specific input tokens encountered
at the point that the syntax error occurs.

\section{Example}

The following example, derived from bug \#168 on the Unicon bug
tracker on Source Forge, illustrates the process. Thanks to Charles
Evans for reporting and/or fixing many bugs such as this.  The error
fragment is just a test program, but you should strip it down to the
minimum required to produce the error. The test file can be named
whatever you like; the filename is not preserved when it is placed
into the merr meta.err file as a fragment. I typically use synerr.icn,
or synerrN.icn for some integer N when I have multiple errors in a
directory.  In this particular case the error code looks like

\iconcode{
procedure action\_12()\\
yyval := \\
end
}

\noindent The default diagnostic you get, before adding the error fragment
and running merr, looks like:

\iconcode{
unicon synerr\\
Parsing synerr.icn: \\
synerr.icn:3: \# "end" syntax error (232;272)\\
(use -M for assistance with this error)
}

What should the compiler have said?  Perhaps ``Assignment is missing
right operand value.'' Fine.  Add all this to the end of meta.err:

\iconcode{
procedure action\_12()\\
yyval := \\
end\\
::: Assignment is missing right operand value.
}

Run the commands ``unicon merr'' followed by ``merr unicon''. If all
went well, a new yyerror.icn was written, adding a table entry for
parse state 232 that we needed, and unfortunately losing many lines
that we will put back manually. A diff shows what Unicon's yyerror
customizes compared with the generic yyerror.icn that merr writes out.
Customization adds about 20 lines to be copied back into yyerror.icn
each time merr is run.

\iconcode{
--- yyerror.icn	(revision 5236)\\
+++ yyerror.icn	(working copy)\\
@@ -65,6 +65,7 @@\\
+   t[232] := table("Assignment is missing right operand value.")\\
@@ -91,43 +92,16 @@\\
-   if \_\_merr\_errors = 0 then iwrite(\&errout)\\
+   if \_\_merr\_errors = 0 then write(\&errout)\\
    else if map(s)== "stack underflow. aborting..." then return\\
    \_\_merr\_errors +:= 1\\
-   if \_\_merr\_errors $>$ 10 then \{\\
-      if *\textbackslash parsingErrors $>$ 0 then \{\\
-	 every pe := !parsingErrors do \{\\
-	    iwrite(\textbackslash\&errout, pe.errorMessage)\\
-	    \}\\
-	 \}\\
-      istop("too many errors, aborting")\\
-      \}\\
-   prefix := (\textbackslash yyfilename{\textbar}{\textbar}":") {\textbar} ""\\
+   if \_\_merr\_errors $>$ 10 then\\
+      write("too many errors, aborting") \& stop()\\
    if s == "syntax error" then\\
       s := t[(\textbackslash statestk)[1], yychar]\\
    if s == "syntax error" then \{\\
       s {\textbar}{\textbar}:= " (" {\textbar}{\textbar} (\textbackslash statestk)[1] {\textbar}{\textbar}";"{\textbar}{\textbar} yychar {\textbar}{\textbar} ")"\\
-\\
       \}\\
-   s := prefix {\textbar}{\textbar}yylineno{\textbar}{\textbar}": \# \textbackslash""{\textbar}{\textbar} yytext {\textbar}{\textbar} "\textbackslash" " {\textbar}{\textbar} s\\
-   if \textbackslash merrflag then \{\\
-      if ferr2 := open(\textbackslash yyfilename) then \{\\
-	 iwrite(\&errout, "Reporting (-M) your error to the Oracle (",\\
-	       merraddress, ") for assistance.")\\
-	 iwrite(\&errout, "Type any question you have on the lines below.",\\
-	       " Type a blank line to finish.")\\
-	 ferr := open("mail " {\textbar}{\textbar} merraddress, "pw")\\
-	 while iwrite(ferr, "" ~== read())\\
-	 iwrite(ferr)\\
-	 iwrite(ferr, s)\\
-	 iwrite(ferr)\\
-	 while iwrite(ferr, read(ferr2))\\
-	 close(ferr2)\\
-	 close(ferr)\\
-         \}\\
-      \}\\
-   /parsingErrors := []\\
-   errorObject := ParseError( yylineno, s )\\
-   put( parsingErrors, errorObject )\\
+   write(\&errout, (\textbackslash yyfilename{\textbar}"lambda.icn"),\\
":",yylineno, ": \# \textbackslash "", yytext, "\textbackslash ": ", s)
}

We keep the new line added at 65, but restore the rest of the Unicon
customizations to yyerror.icn so that the commit is just the one line
addition to the table.  Now the Unicon diagnostic for this input
looks like

\iconcode{
\$ unicon synerr\\
Parsing synerr.icn: \\
synerr.icn:3: \# "end": Assignment is missing right operand value.
}


\section{Conclusions}

Using Merr to update the Unicon translator with new messages is fairly
easy, but does involve some manual steps. It will be nice if there
someday exists a tool that will generate {\em all\/} error fragments
directly from the grammar, so that no undefined error messages could
occur unless new grammar rules were added. Until that day, we will add
error fragments as they are encountered.

\section*{References}

[1] Merr: an LR Syntax Error Message Tool. unicon.org/merr

\end{document}
