\documentclass[letterpaper,12pt]{article}
\usepackage{utr}

\title{How to Write a \\ Unicon Technical Report}
\author{Clinton Jeffery}
\trnumber{15}
\date{2013-07-04}

\begin{document}
\abstract{
Technical reports are an essential means by which technology and ideas are
disseminated. This report is for Unicon contributors and students
who are asked by their community or their research advisors to
produce a technical report that describes their work.
}
\maketitle

\section{Introduction}

This technical report serves as a template and guide to how to produce
a Unicon technical report as of July 2013. It is based on an earlier
"How to Write a CS Technical Report" document [1].

Unicon technical reports are the primary and permanent archival
technical documentation for an open-source software project. As such,
it is important that they are open documents (for example, public
domain or GNU Open Documentation License).  This particular report is
formatted in LaTeX. You may use any formatting system, font families
and sizes and so forth, but the preferred formats are human-readable
ASCII-based text, because it is amenable to long-term archival
survivability and revision control repository difference analysis. The
Unicon Project reserves the right to reformat any document submitted
for consideration as a Unicon Technical Report into an open ASCII
text-based format.

The only hard formatting requirement is that your cover page follow
the general format that this report observes, so that its title,
author(s), date, and technical report \# appear in the window on the
report cover. An abstract and the institutional address, if any,
should appear lower on the title page.  Unicon Citizens without an
institutional address may use "Unicon Project" in that location.

\section{Who Writes Unicon Technical Reports}

Unicon Technical Reports describe core contributions to the
Unicon Programming Language, its supporting tools, and/or its libraries.
Typically faculty members, thesis and dissertation students, research
assistants, and independent study students write technical reports
when they have a substantial piece of work of relevance to the Unicon
community.  Many reports serve as a mechanism to make research results
available quickly, before they see publication in a conference or
journal. Other reports present smaller results or material that may
not warrant external publication, such as reference materials that
accompany software developed during the course of Unicon-related
research and development. Basically, any document that a Unicon
Citizen feels should be a technical report can be made into one.

\section{The TR Production Process}

Once you have written the essential technical content of your report,
follow the steps below to introduce it electronically to the Unicon
Project files. A sponsoring Unicon Citizen should either be a coauthor
or should approve the content of the report before you start this
process. If you have made a contribution that warrants a Unicon TR,
you may well have become a Unicon Citizen yourself by the time you
are finished with the report.

\begin{itemize}
\item 1. TR Number Assignment.
Request a TR number only after you have a finished copy ready to
go. In your request for a number, send the UTR Tzar (currently
jeffery@uidaho.edu) a draft electronic copy of the report.
\item 2. You receive a number when the UTR czar updates the Unicon TR list.
\item 3. You add the number to your cover page, and give the UTR czar
electronic source code and letter-size PDF final copy of your document.
\item 4. The UTR czar posts the UTR PDF to the Unicon web site, and
enters the source code into the Unicon Project revision control system.
\end{itemize}

\section{Revisions}

Useful software has a long life during which it goes through many
updates and revisions. The accompanying technical reports get revised,
also. The steps to follow for a revision are similar to the steps for
a new report, except that the previous TR number should be in the
messages sent to the UTR czar. Typically an update does not require a
new TR number; instead the UTR number is appended with a revision
letter, and your revised TR should include the original date
followed by the most recent revision date. Large changes to a
technical report may warrant issuance of a new TR number.

\section{Advice to Junior Authors}

The quality of technical writing is important. If you haven't written
technical documentation before, allow yourself time for numerous draft
revisions with your sponsor, and your committee members if you
are writing a thesis. You will also need time to master the
publishing software or formatting language. Most document preparation
systems are full of quirks and obscure bugs. The best thing to do is
find someone who is already an expert and seek advice; your faculty
sponsor is such a person.
The abstract is extremely important. Many people will read it
who may never read the rest of your document, especially if the
abstract is poor. The abstract should be to the point, and
it must convey the central contributions of your work.

\section{Conclusions}

Writing technical reports is a lot of work but it is an important form
of participation in the local and global computer science community.
Whether research or development, if an open source contribution is
not understood and used by others its value is limited.
The list of Unicon Project technical reports is located at
http://unicon.org/reports/html

\section*{References}

[1] Clinton Jeffery. "How to Write a UTSA CS Division Technical
Report". University of Texas at San Antonio Computer Science Division
Technical Report 95-1, 1995.

\end{document}
